\documentstyle[12pt,a4wide,epsf]{article} 
% Specifies the document style.

\parindent0mm \parskip1.5ex plus 0.5ex minus 0.5ex
\pagestyle{plain}  % Vielleicht aber doch...
\setcounter{tocdepth}{4}
\setcounter{secnumdepth}{4}
\makeindex
\frenchspacing
\textheight 22cm % DIN_A4 = 22 cm

\setlength{\arraycolsep}{0.0mm}
\def\epsfsize#1#2{\epsfxsize}

%indexhack
\newcommand{\indexentry}[2]
{\item #1 #2}

\newcommand{\toindex}[1]
{\index{#1}#1}

\newcommand{\itindex}[1]
{\index{#1}{\em #1}}

\newcommand{\fig}[1]{fig. \ref{#1}}

\hyphenation{}

\bibliographystyle{pubsbib}

\begin{document}

\title{Extending {\sc Clig}: Interaction and User Defined Graphics}

\author{Karsten Konrad, konrad@coli.uni-sb.de}



\maketitle

\abstract{The {\sc Clig} grapher for linguistic data structures has
 been designed to be extendible for both user defined and interactive
 graphics. This paper describes how a user can add new graphical
 objects to the grapher by adding {\sc Tcl}-code to its sources and
 how interaction between the grapher and an underlying application can
 be programed.}


%%%%%%%%%%%%%%%%%%%%%%%%%%
%%% Inhaltsverzeichnis %%%
%%%%%%%%%%%%%%%%%%%%%%%%%%
\newpage
\tableofcontents
\newpage

\section{Introduction}

\cite{konrad 95a} describes the {\sc Clig} grapher and its use as a
general tool for visualizing common linguistic graphical notations. The
main purpose of the program is its use as a graphical interface for
applications in computational linguistics. These applications can
display their data by creating {\it description strings\/} and sending
them to the grapher. These strings are hierarchical textual
descriptions of the graphical structures. E.g., the following text is
the description for the graphical structure seen in Figure \ref{ex-1}:

\begin{quote}
\begin{verbatim}
{drs {plain-text B}
     {tree {plain-text s} 
           {plain-text B} 
           {tree {plain-text vp1} 
                 {tree {plain-text vp} 
                       {tree {plain-text v} 
                             {plain-text owns}} 
                       {tree {plain-text np} 
                              {tree {plain-text pn} 
                                    {plain-text ulysses}}}}}} 
      {plain-text jones(B)}}
\end{verbatim}
\end{quote}

\begin{figure}[ht]
\begin{center}
\leavevmode
\epsfbox{drs-tree.ps}
\caption{A DRS with a tree nested inside.}
\label{ex-1}
\end{center}
\end{figure}

With its description string language, {\sc Clig} provides a wide range
of possibilities to create the graphical representations one has in
mind. Objects can e.g. be ordered in stacks or sequences, can be put
in boxes or can each be assigned colors separately. The basic
notations like trees, texts or DRS boxes can be nested freely within
each other, producing complex graphical structures like. However, for
certain applications, the grapher's built-in facilities might not be
sufficient to produce the desired output. {\sc Clig} has been
designed in a way that makes it relatively easy to add new graphical
features. Section \ref{extend} describes how a user can add his or her
own graphical objects to the grapher.

By linking objects in the description string to pieces of code, an
application defines {\it interactive\/} graphics where a user can
perform application-dependent actions by clicking on objects with the
mouse cursor. Interactive graphics are under the control of the
application which creates the description strings. The application
decides which objects behave in what way by sending the appropriate
code with their descriptions. Interaction between grapher and
application is freely programmable and therefore very flexible, but
may also require some effort on the side of the programmer of the
application. {\sc Clig} is extendible in the sense that the application
programmer can add the amount of interactive behavior needed for the
specific application. Section \ref{interact} explains how interfaces
between applications and the grapher can be build.

\subsection{Tcl/Tk as the programming environment}

{\sc Clig} has been implemented using the software package {\sc
Tcl/Tk}, a programming system for graphical user interfaces. {\sc
Tcl}, the {\it tool command language\/}, is a simple interpreted
scripting language providing basic programming facilities like
variables, procedure definitions and loops.  {\sc Tk} is a toolkit for
the X Windows System that extends the core {\sc Tcl}-language with
commands to build Motif-like graphical user interfaces. The {\sc
Tcl/Tk} package can be used via the interpreter program {\sc Wish}
that contains the language interpreter for {\sc Tcl} and the {\sc Tk}
toolkit. {\sc Clig}'s source code is a set of {\sc Tcl} scripts which
gets interpreted by {\sc Wish}.

The use of Tcl/Tk has some obvious advantages and one minor drawback.
One advantage is portability: {\sc Tcl/Tk} is free and widely available
for different OS-platforms. The X-Windows functionality used by {\sc
Tk} gets emulated for foreign systems like Windows or Mac OS. Another
advantage is the {\it rapid prototyping\/} of code. Since {\sc Tcl}
gets interpreted, code extensions or modifications for the grapher can
be loaded on-the-fly without the need for recompilations. Modified code
simply overwrites old, while new code can be added following {\sc
Clig}'s simple code conventions (see section \ref{code}).  In contrast
to C or C++, {\sc Tcl/Tk} hides low-level details of the X-Windows
environment from its users, making it generally easier to write {\sc
Tcl/Tk} graphical user interfaces than C/C++ user interfaces.  The
minor disadvantage of {\sc Tcl/Tk} is the slower execution speed of the
interpreted {\sc Tcl} code which can be circumvented by writing C
routines for time-critical algorithms. Thanks to the speed of modern
work stations, this kind of optimization is rarely necessary. {\sc
Clig} currently does not contain any non-{\sc Tcl} code.

\section{Interaction}
\label{interact}

An interactive graphical structure contains parts which are mouse
sensitive; in {\sc Clig} these parts are linked to small {\sc Tcl}
programs called {\it scripts}. A {\sc Tcl} script in an interactive
graphical structure could e.g. open a pop-up menu or ask an application
for more information on whatever a user has clicked on. The linking of
code to graphical objects in the description strings can be done by the
special commands {\tt clickable} and {\tt active}.

\subsection{The {\tt clickable} command}

The {\tt clickable} command uses one single script which gets executed
when the user {\it double-clicks\/} on the associated object.  The
command takes two arguments, the object to be drawn and the {\sc Tcl}
script linked to it.  The following example creates a mouse sensitive
text box which produces a short message when clicked by the user:

\begin{quote}
\begin{verbatim}
{clickable {bigbox plain-text "click me!"} 
           {puts "YOU CLICKED ME!"}}
\end{verbatim}
\end{quote}

\subsection{The {\tt active} command}

The {\tt active} command can define multiple possibilities for the
same object or define special behaviors.  It allows the use of
several event-script pairs for the same object. An {\it event\/} can
be any X-Window event known in {\sc Tk} such as {\tt <Leave>}, {\tt
<1>}, {\tt <Double-3>} etc.

The general syntax for {\tt active} is 

\begin{quote}
\begin{verbatim}
{active <object> {<event> <script>} ... {<event> <script>}}
\end{verbatim}
\end{quote}
If you wanted the message from the last example to appear when the
middle mouse button gets clicked once, you would express it like this:

\begin{quote}
\begin{verbatim}
{active {bigbox plain-text "click me!"} 
        {<2> {puts "YOU CLICKED ME!"}}}
\end{verbatim}
\end{quote}

\subsubsection{Popup menues}

Popup menues are a standard feature of {\sc Tk} which have been
included into the {\tt active} command in recent {\sc Clig} versions
(Version 1.4 or greater). {\tt Active} accepts the keyword {\tt popup}
in a event-command-pair and uses the command part as a definition of a
popup menu as described in \cite{Ousterhout 94}. The following example
defines a box which shows a popup menu when clicked on. When you select
one of the buttons in it, it will print a short message.

\begin{quote}
\begin{verbatim}
{active {bigbox {color red {plain-text "CLICK ME!"}}}
             {popup <1> {{command -label "TEST1" 
                                -command "puts TEST1"}
                         {command -label "TEST2" 
                                -command "puts TEST2"}
                         {separator}
                         {command -label "TEST3" -command "puts
             TEST3"}}}}
\end{verbatim}
\end{quote}

\subsection{Application meets {\sc Clig}}

The concept of interfaces between applications and {\sc Clig} is shown
in Figure \ref{archy1}. The application ``talks'' to the grapher by
sending it description strings. Some systems like {\sc Sicstus3} have
their own {\sc Tcl/Tk} interface which can be used for this. Another
way to create an interface to the grapher is the use of the {\bf send}
mechanism which can be used by any {\sc Tk} application. We are also
currently working on a very simple interface which uses {\sc Clig}'s
standard input and output. In all these cases, the main drawing routine
{\tt clig} gets called by the application with an description string
containing {\tt clickable} or {\tt active} commands. The commands use
{\sc Tcl} scripts which in response to a mouse action force the
application to do something. The following example is taken from a
larger description string.

\begin{quote}
\begin{verbatim}
{bigbox clickable 
	{plain-text "app(1,2)"}
	{prolog_event "click(sem_node, d1, 3)"}}
\end{verbatim}
\end{quote}

The example structure is created by the educational tool developed in
the FraCaS project. The tool uses the {\sc Tcl/Tk} interface provided
by {\sc Sicstus3} Prolog. In the example, the text {\tt app(1,2)} is
linked to a script specifying that a
\verb'prolog_event' command gets executed when the text gets
double-clicked with the mouse. The \verb'prolog_event' command results
in an event which gets noticed by a loop in Prolog waiting for such an
event. The string \verb'click(sem_node, d1, 3)' informs Prolog that the
node {\tt 3} has been clicked on. The main loop then can take the
appropriate action and will display a new graph where node 1 has been
applied to node 2.

The concrete realization of the interface between an application and
{\sc Clig} depends on what the host system offers as a means of
communicating with {\sc Tcl/Tk}. We are currently experimenting with
different general approaches like using a TCP/IP channel or simple
standard input/output piping to allow a common interface.


\begin{figure}[H]
\begin{center}
\leavevmode
\epsfbox{architekt.eps}
\caption{Application-to-Clig interaction.}
\label{archy1}
\end{center}
\end{figure}

\section{Adding graphical objects}
\label{extend}

This section describes how user defined objects can be included into
the grapher. Subsection \ref{descr} explains how code extensions
appear in the description strings which are used for all
representation and interfacing purposes within the grapher. Subsection
\ref{code} explains the conventions necessary for code extensions.

\subsection{Description strings}
\label{descr}

The description of a graphical object in {\sc Clig} is done with a
hierarchical list in {\sc Tcl}-syntax, the {\em description
string}. The general syntax of a description string is

\begin{quote}
\begin{verbatim}
{<command> <parameter_1> ... <parameter_n>}
\end{verbatim}
\end{quote}

Most commands take description strings as their parameters. If a
command takes only one description as parameter, the command is said
to {\it modify\/} the description. The general syntax for such a
modifying command is

\begin{quote}
\begin{verbatim}
{<modify-command> <command> <parameter_1> ... <parameter_n>}
\end{verbatim}
\end{quote}

E.g. a description string like \verb'{neg plain-text "x"}'  is a
Tcl-list with 3 members, with the first two of them being
commands. The {\tt neg} command modifies the output of the {\tt
plain-text} command by putting a negation in front of the text.

Whenever {\sc Clig} draws a graph, it really interprets a
description string like this as a piece of {\sc Tcl}-code which lacks
some information, e.g. the exact positioning of the objects. The
description string gets executed by an internal interpreter which
recursively fills in the missing information after it has calculated
the correct positions of all objects in the drawing area. Each command
in the description string has an equivalent {\sc Tcl} procedure which
gets called with the missing information and all parameters. Extending
{\sc Clig} means writing such procedures and adding them to the source
code. 

As an example, take a look at the code which is responsible for drawing
negations:

\begin{quote}
\begin{verbatim}
proc neg {x y where mtags obj} { ;# draw negated drs
        seq $x $y $where $mtags \
        [list {sign neg} {plain-text " "} $obj]}
\end{verbatim}
\end{quote}

Without going into details, the code for {\tt neg} simply calls another
command used in {\sc Clig}, {\tt seq}, with three graphical objects,
the negation sign, a space character and the object which should be
negated. The {\tt seq} command gets some additional parameters it
inherits from the {\tt neg}-call, namely $x$ and $y$ for the exact
position of the object, $where$ for the canvas which should be used
({\sc Clig} uses two canvases, one for actual drawing and one for
calculating the size of an object) and $mtags$ which is used for active
(clickable) regions. The $mtags$ parameter contains a set of labels and
usually is only manipulated by the {\tt active} and {\tt clickable}
procedures for grouping together objects with the same scripts linked
to them. All other commands do not change $mtags$ but must inherit this
parameter to each of its daughters. By inheriting labels to their
daughters, clickable objects correctly execute their associated script
when one of their daughter objects are clicked on.

\subsection{Code conventions}
\label{code}

Each graphical command for {\sc Clig} must have exactly 5 parameters:
the $x$ and $y$ positions of the top left corner, the $where$ canvas,
the $mtags$ parameter and an additional parameter $obj$ containing
everything that was in the object description string except the command
name itself.

Each command must return the width and height of the drawn object as a
list. A good example of the use of the sizes would be the code for {\tt
underline} which simply underlines another object:

\begin{quote}
\begin{verbatim}
proc underline {x y where mtag obj} { 
        ;# drawing
        set size [execobj $obj $x $y $where $mtag]
        ;#
        set yline [expr $y+[lindex $size 1]+2]
        $where create line $x $yline \
                           [expr $x+[lindex $size 0]] $yline
        ;# returning the complete size of the underlined
        ;# object:
        list [lindex $size 0] [expr $yline-$y]}
\end{verbatim}
\end{quote}

The {\tt underline} procedure first draws the object $obj$ by calling
{\tt execobj}, the main drawing function which tries to execute the
description string $obj$ at position $x$,$y$. {\tt execobj}, the
interpreter procedure, returns the width and height of the object
which {\tt underline} uses to calculate where ($y$ position $+$ 2
points) and with which width to draw the line under the object. {\tt
underline} returns the width of the object and the height $+$ 2 points
as the size of the object drawn.

This example covers almost everything one has to know about writing
code for extending the grapher. A procedure has to have a unique name,
must use the 5 formal parameters described above and must return the
correct width and height of the object it draws---and then it is a
legal extension of the grapher. An application can simply load the
additional code using the methods described in section \ref{include}
and then use the new procedures in the same way as the ``factory''
commands {\sc Clig} provides. A user can even replace the built-in
commands by simply overwriting them with his own code since {\sc Tcl}
will not report an error for redefining procedures. In this way, an
application might modify the graphical output for its special needs,
e.g. by either optimizing for beautiful layout or speed, without the
need for actually changing the original code.

Admitedly, the simple convention used here has a few practical
drawbacks. The layout algorithm of an object can only use rectangular
regions for calculating the positions of its daughters. In the case of
objects with very irregular shape, this may result in a waste of space
and/or may result in an aesthetically unsatisfying result. Another
drawback ist the strict top-down approach of drawing the graphics.
Some graphical structures must calculate the size of their daughters
before they can actually draw them; in this case a bottom-up approach
would be better. However, the nasty side effect of having to draw
objects unecessarily because of repeated size calculations with {\sc
Clig}'s top-down algorithm can be circumvented by using the {\tt
calcsize} command described in the next section.

\subsection{Canvases and calculating sizes}

Sometimes a command has to know the size of an object {\em before\/} it
gets drawn. E.g., the {\tt stack} command stacks several objects on top
of each other; the command uses the size of the biggest box as the
maximum $x$ size of the whole object and centers all objects within
this maximum size. Therefore, before any object gets drawn, {\tt stack}
calculates all sizes and stores the maximum in the variable $xsize$. A
simplified version of this calculation routine would look like this:

\begin{quote}
\begin{verbatim}
foreach item $stack {
        set size [calcsize $item]
        set xsize [max $xsize [lindex $size 0]]}
\end{verbatim}
\end{quote}

The procedure which calculates the size of all objects, {\tt calcsize},
actually draws the object on an invisible canvas and returns the
size which every {\sc Clig} drawing procedure returns.

The {\tt calcsize} procedure is almost identical to {\tt execobj}
except that it does not need any coordinates nor $mtags$ which only
play a role in the visible drawing area. While {\tt execobj} uses the
visible {\it .graphics} canvas, {\tt calcsize} simply uses the canvas
{\it .invisible}. This special canvas never gets displayed by the
grapher; it only exists for evaluating sizes. Procedures can take
advantage of this distinction by minimizing what actually gets drawn
during size evaluation. Since drawing in an invisible area does not
have to look good or even complete, procedures may leave out any
actual drawing whenever they encounter {\tt .invisible} as their
$where$ parameter, as long as they can calculate the correct size of
the object they draw.

As an additional optimization, {\tt calcsize} uses a {\it
memoization\/} technique for storing sizes it already has calculated.


\subsection{Including your own code}
\label{include}

To include your own code into {\sc Clig}, you can simply change the
file {extend.tcl} in the {\sc Clig}-directory so that it loads your
files when you start the grapher. The only problem with this is that
the extensions get loaded whether they are needed or not and that you
have to modify {\tt extend.tcl} whenever you install the grapher. An
application using {\sc Clig} usually can do better by simply
performing the {\tt source} command of {\sc Tcl}. In this way, an
extension gets loaded only when needed by the specific application. 


\subsection{An example: Circles}

Supposed that you wanted to write a package for {\sc Clig} for
displaying e.g. {\em finite state machines\/} and therefore needed
circles to represent the states of the machine. In this case you would
have to write your own code, since circles are not supported in the
standard release of {\sc Clig}.

The first step would be to create a new file {\tt circle.tcl} and to
add this file to the {\tt extend.tcl} loading list as described in
section \ref{include}. The file {\tt circles.tcl} will be loaded
whenever you run {\sc Clig}. Every time you change your code, you can
include your changes simply by exiting your old {\sc Clig} and restarting
it again. An even easier way would be to use the {\bf Load...}  option
from the {\sc Clig} main menu. Although this option usually is used to
load object descriptions, it also is able to load arbitrary {\sc Tcl}
source code.

Now you can add the circle code. You want to use the commands {\tt
state} for a normal state and {\tt final} for a final state in your
finite state package. A state simply consists of a circle of a
certain size, while a final state will be displayed as two concentric
circles. Additionally, you want to have states to have arbitrary sizes,
so that \verb'{final 20}' will display a finite state with a radius of 
20 points.

The code for a normal state could look like this:

\begin{quote}
\begin{verbatim}
procedure state {x y where mtags rad} {
        set rad_size [expr $rad*2] 
        $where create oval $x $y \
                           [expr $x+$rad_size] \ 
                           [expr $y+$rad_size] \
                           -outline black -tags $mtags
        list $rad_size $rad_size}
\end{verbatim}
\end{quote}

Since you wanted the state to have a radius of $rad$, the size of the
complete object will be 2 $rad$ in x and 2 $rad$ in y direction. The
procedure draws an oval with a black outline which bears the labels
$mtags$. The labels are necessary for interactive graphs, see
section \ref{interact}.
The code does draw an oval whenever the procedure gets called, but this
may not always be necessary. Trees and other complex structures often
check for the size of their daughters just to see where they can be
placed. The following refinement of the code leaves out this unnecessary
drawing of the circle:  

\begin{quote}
\begin{verbatim}
proc state {x y where mtags rad} {
        set rad_size [expr $rad*2] 
        if {$where!=".invisible"} {
           $where create oval $x $y \
                              [expr $x+$rad_size] \ 
                              [expr $y+$rad_size] \
                              -outline black -tags $mtags}
        list $rad_size $rad_size}
\end{verbatim}
\end{quote}

This version only draws the circle when the destination canvas is not
equal to {\tt .invisible}.
The code for final states is very similar to the code for states; it
just puts a smaller circle inside the outer circle.

\begin{quote}
\begin{verbatim}
proc final {x y where mtags rad} {
        set rad_size [expr $rad*2] 
        set offset 3
        if {$where!=".invisible"} {
           $where create oval $x $y \
                              [expr $x+$rad_size] \ 
                              [expr $y+$rad_size] \
                              -outline black -tags $mtags
           $where create oval [expr $x+$offset] \
                              [expr $y+$offset] \
                              [expr $x+$rad_size-$offset] \ 
                              [expr $y+$rad_size-$offset] \
                              -outline black -tags $mtags}
        list $rad_size $rad_size}
\end{verbatim}
\end{quote}

After rerunning {\sc Clig}, you should have the possibility to display
states and final states as displayed. For a quick test, just type in the
following test file in a text editor and load it with {\sc Clig}:

\begin{quote}
\begin{verbatim}
# example: two states
clig {Seq {state 20}
          {final 20}}
\end{verbatim}
\end{quote}

\subsection{Colors}

The {\tt color} command provides an easy way to add color information
to graphs. User defined object code must include the following to be
compatible with the {\tt color} command:

\begin{itemize}
\item the command \verb'globals main_color' to make the global variable
\verb'main_color' known to the procedure
\item each drawing command (like e.g. drawing a line) must use
\verb'main_color' for the main drawing color. Most drawing commands,
e.g.  {\tt create line} use {\tt -fill} as the option for specifying
the foreground color.
\end{itemize}

The code for {\tt underline} contains only one drawing command, the
{\tt create line} code which draws the underlining line. With 
\verb'-fill $main_color', the line's color can be changed by a {\tt
color} command:

\begin{quote}
\begin{verbatim}
proc underline {x y where mtag obj} { ;# line under object
        global main_color
        set size [execobj $obj $x $y $where $mtag]
        set yline [expr $y+[lindex $size 1]+1]
        $where create line $x $yline \
                           [expr $x+[lindex $size 0]] $yline \
                           -tags $mtag -fill $main_color
        list [lindex $size 0] [expr $yline-$y]}
\end{verbatim}
\end{quote}
 

\newpage
\begin{thebibliography}{kako}
   \bibitem[Fracas 95]{Fracas 95}
        Robin Cooper, Dick Crouch, Jan van Eijck, Chris Fox,
        Josef van Genabith, Jan Jaspars, Hans Kamp,
        Manfed Pinkal, Massimo Poesio and Steve Pulman.
        {\em Evaluating the State of the Art}. FraCaS Deliverable D10.
        University of Edingburgh, Centre for Cognitive Sciences. 
   \bibitem[Kamp 93]{kamp 93} Hans Kamp and Uwe Reyle.
        {\em From Discourse to Logic}.  Kluwer Academic Press,
        Dordrecht - Boston - London, 1993 
   \bibitem[Konrad 95a]{konrad 95a} Karsten Konrad. {\em The {\sc Clig}
           grapher for linguistic data structures}. 
           unpublished, 1995.
   \bibitem[Ousterhout 94]{Ousterhout 94} John K. Ousterhout. {\em Tcl
           and the Tk Toolkit\/}.  Reading, Massachusetts - Menlo Park,
           California - New York, Addison-Wesley Professional Computing
           Series, 1994.  
   \bibitem[Pollard 87]{Pollard 87} Carl Pollard
           and Ivan A. Sag. {\em Information-Based Syntax and
           Semantics. Volume 1: Fundamentals}. CSLI Lecture Notes,
           Menlo Park - Stanford - Palo Alto, Center for the Study of 
           Language and Information (CSLI), 1987
   \bibitem[Sicstus 95]{Sicstus 95} Programming Systems Group of the
           Swedish Institute of Computer Science. {\em SICStus Prolog
           User's Manual}. Release 3. Swedish Institute of Computer 
           Science, 1995.
\end{thebibliography}

\end{document}