\documentstyle[12pt,a4wide,epsf]{article} 
% Specifies the document style.

\parindent0mm \parskip1.5ex plus 0.5ex minus 0.5ex
\pagestyle{plain}  % Vielleicht aber doch...
\setcounter{tocdepth}{4}
\setcounter{secnumdepth}{4}
\makeindex
\frenchspacing
\textheight 22cm % DIN_A4 = 22 cm

\setlength{\arraycolsep}{0.0mm}
\def\epsfsize#1#2{\epsfxsize}

%indexhack
\newcommand{\indexentry}[2]
{\item #1 #2}

\newcommand{\toindex}[1]
{\index{#1}#1}

\newcommand{\itindex}[1]
{\index{#1}{\em #1}}

\newcommand{\fig}[1]{fig. \ref{#1}}

\hyphenation{}

\bibliographystyle{pubsbib}

\begin{document}

\title{The {\sc Clig} Grapher for Linguistic Data Structures}

\author{Karsten Konrad, konrad@coli.uni-sb.de}



\maketitle

\abstract{This paper describes {\sc Clig}, an interactive, extendible
 grapher for visualizing linguistic data structures developed in the
 FraCaS project. The grapher has been designed both to be stand-alone
 and to be used as an add-on for linguistic applications which display
 their output in a graphical manner. The focus of this paper is to
 introduce the reader with {\sc Clig}'s graphical description language
 and its use as a general tool to display trees, feature structures,
 $\lambda$-expressions and semantic structures such as DRSs.}

%%%%%%%%%%%%%%%%%%%%%%%%%%
%%% Inhaltsverzeichnis %%%
%%%%%%%%%%%%%%%%%%%%%%%%%%
\newpage
\tableofcontents
\newpage

%%%%%%%%%%%%%%%%%%%%%%%%%%%%%
%%% Abbildungsverzeichnis %%%
%%%%%%%%%%%%%%%%%%%%%%%%%%%%%
%\listoftables
%\addcontentsline{toc}{section}{Abbildungsverzeichnis}
%\newpage

%%%definitionen
\newtheorem{definition}{Definition}
\newtheorem{lemma}{Lemma}

\section{Introduction: What is {\sc Clig}?}

A {\em grapher\/} is a program for displaying graphical information.
The program {\sc Clig}\footnote{{\sc Clig} stands for {\em
Computational Linguistics Interactive Grapher}} is a grapher which has
especially been designed to provide developers of applications in
computational linguistics with a way to display their data structures.
{\sc Clig} provides most graphical notations common in syntax and
semantics such as trees, AV-Matrices or {\it Discourse Representation
Structures}\footnote{see \cite{Kamp 93}}. A flexible layout algorithm
allows the use of arbitrarily nested or embedded notations like e.g. a
tree of feature structures.

Depending on the application, graphical structures can be assigned an
{\it interactive\/} behavior.  This means that a user can click on
mouse sensitive regions in the display area which will result in an
action performed by the underlying application. E.g. a user might click
on a node in a tree and get further information on that node.

{\sc Clig} is written in Tcl/Tk\footnote{see \cite{Ousterhout 94}}, the
widespread programming system for developing graphical user interfaces
(GUIs). One of the major properties {\sc Clig} has, thanks to the
Tcl/Tk-environment, is its extendibility. The source code of {\sc Clig}
can be seen as a basic set of ready-to-use tools which can be extended
with additional, more application specific modules if needed. A
developer may add his or her own graphical data structures to the
grapher (which is not very hard if you follow some simple conventions)
and provide these with their own interaction capabilities such as
clickable regions or edit windows. Communication between the {\sc Clig}
grapher and an application is possible when the application's host
system (e.g. {\sc Common Lisp}) provides a way to send a simple command
string to Tcl/Tk. {\sc Clig}'s interactive possibilities can be used
when the host system can give control to Tcl/Tk and read its
variables. Most modern programming environments provide these
communication channels or will so in the near future, since Tcl/Tk is
becoming the most popular extension language so far.

In conclusion, the {\sc Clig} grapher so far has the following
properties:

\begin{itemize}
\item It can display graphical data from simple text to nested
DRS, trees and typed feature structures
\item Missing graphical elements can be added by the user following
a simple convention
\item Description strings are used to describe the graphics and can
be sent to the grapher by any application
\item Graphical elements, such as a text or a whole DRS can be
specified to be clickable and can be associated with executable
Tcl-code
\item This code can specify any kind of interactive behavior via the
underlying application
\item {\sc Clig} provides standard facilities such as saving and loading
of graphical objects and printing to Postscript
\end{itemize}

The focus of this introductory paper is the use of {\sc Clig} as a
pure displaying tool for graphical information. {\sc Clig} uses a
human-readable ASCII-representation for the graphs it displays which
can be created and modified with common text editors like {\sc vi} or
{\sc Emacs}. Once you understand this description language, you can
simply enter graphs in your text editor and display them by loading
them into the grapher. Although I believe that very few people will use
{\sc Clig} in this way ({\sc Clig} usually gets its input from
programs, not human users), it is a good way to become familiar with
the grapher and its possibilities. More advanced users, who want to use
{\sf Clig} in interactive applications will find more detailed
information in \cite{Konrad 95b}.

\section{Using {\sc Clig}}

To get a quick overview of {\sc Clig}'s capabilities, the program comes
with a set of examples which can be loaded using the menu bar. The next
section describes how you can install the grapher on your machine.

\subsection{A quick installation guide}

In order to install {\sc Clig}, you need:

\begin{itemize}
\item a terminal with Tcl 7.4 and Tk 4.0 (or higher) running on it
\item the current release of the grapher ({\tt clig-release.tar.gz})
\end{itemize}

If you cannot run {\tt wish}, the Tcl/Tk-interpreter on your machine,
please ask your system operator for help or install the current version
of Tcl/Tk in your home directory (the current version can be ftp'ed via
{\tt <URL:ftp://ftp.smli.com/pub/tcl/>}).

You then can do the following step by step:

\begin{itemize}
\item[1.] Put {\tt clig-release.tar.gz} in a directory of its own

\item[2.] {\tt gunzip clig-release.tar.gz}

\item[3.] {\tt tar -xvf clig-release.tar}

\item[4.] {\tt clig}
\end{itemize}

I everything went well, you now should be able to load some examples
from the {\tt examples} directory. 

\subsection{The menu bar}

The menu bars provides the user with the standard facilities of the
grapher for loading and saving graphics, printing and setting some
preferences. It also contains some simple help-mechanisms where you can
get information about the basic functions hidden in the menu bar and
the program itself. 

The {\bf File} menu contains the following menu points:

\begin{description}
\item[New Window] opens a new {\sc Clig} process with a main
window of its own.

\item[Load] opens a file-selector box which lets you load a graphics

\item[Save] opens a file-selector box which lets you save a graphics

\item[Print] Postscript printing to a file

\item[Editor] Opens an editor window with which you can view and edit 
the description strings of the currently displayed graphics.

\item[Exit] closes main window and quits program.

\end{description}

The {\bf View} menu lets you change how things are displayed.

\begin{description}
\item[Redisplay] redisplays the current graph. This graph may not be visible
after you have selected some of the information in the Help-menu. This
option also is useful to see how fast a graph gets drawn and to
redisplay after changing font sizes, switching off colors etc.

\item[Fit Window] expands the main window to the exact size of current
graph after the window size has been changed. Note that you can set the
size of the main window with your window managers resize button. It
will stay the same size until you use Fit Window.

\item[View Top] display the topmost graph in the graph stack, which is
usually the last graph you loaded.

\item[Clear Stack] removes all graphs from memory and displays a copyright
message.

\item[Tree Hspace] lets you choose how much distance a node in a tree
at least should have from its neighbor sibling.

\item[Tree Vspace] lets you choose how much distance a node in a tree
should have from its parent node.

\item[Zoom] is used to adjust the size of the displayed graphics. 

\item[No Colors] switches color graphics on and off. This option should
be used if you either have no way of displaying colors on your screen
or if you want to make a hardcopy which should be printed on a
monochrome printing device.
\end{description}

The {\bf Help} menu contains general information and some basic tips
for using the program.

The menu bar also contains two buttons with arrows in them. These are
used to walk through the graph stack. {\sc Clig} stores all distinct
graphics it has to display in a stack, with the topmost being the newest
object. Thus, if you load three graphs, all three can be reviewed by
using the arrow keys. {\sc Clig} does not store objects twice in the
stack.

\section{The description string}

This section explains the {\it description strings\/} which are used to
describe the graphical information to be displayed.  A description
string is a list in Tcl-syntax containing a hierarchical description of
the object to be drawn without any exact positioning information. The
main function {\tt clig} takes this description as an argument and
performs a top-down drawing algorithm on the string. The keywords
(commands) contained in the string actually are Tcl-functions for
graphical objects which take care of the positioning of their
children. The following examples are description strings:

\begin{itemize}
\item \verb'{plain-text "NP"}'
\item \verb'{sign lambda}'
\item \verb'{drs {plain-text "x"} {plain-text "John(x)"}}'  
\end{itemize}

The first example describes the plain text {\tt NP} (a text with no
further graphical information which also is not clickable in any way).
The keyword for the description always is the first word, in this case
{\tt plain-text}. The second example describes the $\lambda$-sign. The
third describes a graphical structure with two children, the plain text
{\tt x} and the plain text {\tt John(x)}. These two children are to be
placed in a rectangle with the first child being boxed on top of the
remaining second; the children also are centered within the whole
structure. The result is a simple DRS as shown in \fig{drs1}.

In a loadable file, {\sf Clig} requires all descriptions to be
preceeded by the command {\sf clig}.  You therefore can try all
examples in this paper by yourself just by typing them into a text
file, add the word {\tt clig} in front of the example and loading them
with the {\bf Load...} option in the main menu of {\sc Clig}. Another
possibility is to edit the description in the editor window of {\sc
Clig} itself. Note that the editor only displays the actual descriptions
without any of the useful comments that may be in the example files.


\begin{figure}[h]
\begin{center}
\leavevmode
\epsfbox{drs1.ps}
\caption{A simple DRS realized with {\tt drs}}
\label{drs1}
\end{center}
\end{figure}

We already mentioned the fact that the description string is both a
description for a graphical object and a command sequence that realizes
this object on the screen. A description string actually is a piece of
Tcl-code with commands to draw certain objects; only the exact
positioning information is missing. Each keyword stands for one command
that has additional parameters which tell it where to draw its object,
e.g. the keyword {\tt drs} corresponds to the Tcl-command {\tt drs}
which has the parameters $x$, $y$, $where$, $mtags$ and $objs$. The $x$
and $y$ parameters are used to place the object when it is finally
drawn. The $where$ parameter chooses the canvas window where the object
is drawn ({\sc Clig} supports multiple windows open at the same
time). The $mtags$ parameter is used to inherit object indexes for
clickable regions\footnote{Tcl/Tk can logically group objects by
assigning labels to them. Each clickable object inherits a certain
label belonging to itself to all its daughter objects. The parameter
$mtags$ contains all labels inherited from a parent object.} while the
final $objs$ parameter consists of a list containing daughter object
descriptions.  The first one of these objects will be drawn in the case
of {\tt drs} in the head with a box around it, the rest below the top
box. All parameters except $objs$ are invisible in the description
string; they will be filled in when the grapher performs the
description as a Tcl-program.

The following sections explain the graphical commands allowed in the
description string along with their parameters and some simple
examples.

\subsection{Text}

Perhaps the most used command in {\sc Clig}'s description strings is
{\tt plain-text} which displays a string in {\sc Clig}'s standard
font. The text is not mouse-sensitive. {\tt Plain-text} takes a simple
string as parameter, e.g.

\begin{quote}
\verb'{plain-text "This is a plain text."}'
\end{quote}

To emphasize certain information, you can use {\tt bold-text}, which
does the same as {\tt plain-text} but with a bolder font.

\begin{quote}
\verb'{bold-text "This is a bold text."}'
\end{quote}

Text can also be set in {\it italics\/} by the {\tt slant-text}
command. The current version of {\sc Clig} uses the standard Helvetica
font family for its display area. These fonts should be installed for
your terminal.

By using either a {\tt small-} or {\tt big-} prefix, the font size used
can be decreased/increased by 2 points. The command {\tt big-bold-text}
for example will draw a bold text with a bigger font size than normal.

Some special characters are very common in computational linguistics.
The following commands produce the analogous signs in the {\em
symbol\/} font\footnote{which should be installed for your terminal.}.

\begin{quote}
\begin{tabular}{rr}
{\tt sign neg} & displays $\neg$ \\
{\tt sign disj} & displays $\vee$ \\
{\tt sign conj} & displays $\wedge$ \\
{\tt sign lambda}] & displays $\lambda$ \\
{\tt sign merge} & displays $\otimes$ \\
{\tt sign forall} & displays $\forall$ \\ 
{\tt sign exists} & displays $\exists$ \\
{\tt sign subseteq} & displays $\subseteq$ \\
{\tt sign emptyset} & displays $\emptyset$ \\
\end{description} 
\end{quote}

\subsection{Boxes and frames}

You can draw boxes around arbitrary objects with the {\tt smallbox} and
{\tt bigbox} commands. {\tt Smallbox} uses a border of 2 points, while
{\tt bigbox} will leave 6 points of border space. The thickness of the
border stays the same.

\begin{quote}
\verb'{smallbox {plain-text "This is a text in a small box."}}'
\end{quote}

\begin{quote}
\verb'{bigbox {plain-text "This is a text in a big box."}}'
\end{quote}

You can design boxes with arbitrary border-width with {\tt boxed}:

\begin{quote}
\verb'{boxed 3 {plain-text "This box has a 3 point border."}}'
\end{quote}

The command {\tt framed} does the same as {\tt boxed}, but without
actually drawing the border. {\tt framed} gets used when objects should
be drawn with additional space around them.

A special case of a box is the diamond. The command {\tt diamond}
takes an object and puts it in a square diamond (see fig. \ref{diamond}):

\begin{quote}
\verb'{diamond {plain-text "diamond"}}'
\end{quote}

\begin{figure}[hb]
\begin{center}
\leavevmode
\epsfbox{diamond.ps}
\caption{The diamond box}
\label{diamond}
\end{center}
\end{figure}

\subsection{Brackets}

Round and square brackets are drawn by the {\tt bracket} and {\tt
squarebracket} commands. The syntax is similar to the boxes. E.g.

\begin{quote}
\verb'{squarebracket {plain-text "Text in square brackets."}}'
\end{quote}

draws the text in square brackets.

\subsection{Lists}

Lists in angle brackets can be drawn with the {\tt anglelist} command.
The objects get separated by commas.

\begin{quote}
\verb'{anglelist {plain-text "first"} {plain-text "second"}}'
\end{quote}

\subsection{Sequences and stacks}

So far, you can only display one (possibly nested) object. Sometimes it
is necessary to display several objects adjacent to each other
(sequences) or on top of each other (stacks). A good example of a
sequence would be a complex logical formula containing variables (which
we would display as plain text) and logical connectives provided as
special commands. The following description string displays the formula
$\exists x (f(x)\vee v(x))$:

\begin{quote}
\verb'{seq {sign exists} {plain-text "x(f(x)"}' \newline
\verb'     {sign disj} {plain-text "v(x))"}}'
\end{quote}

{\tt seq} orders objects horizontally with minimum space between
the objects. If objects have different heights, they will be centered
vertically. The almost equivalent command {\tt Seq} does leave more
space between the objects.

{\em Stacks\/} are the same as sequences, but ordered vertically. The
description string below describes three boxes stacked on each other
(see fig. \ref{threebox}):

\begin{quote}
\verb'{stack {smallbox {plain-text "tiny"}}' \newline
\verb'       {smallbox {plain-text "bigger"}}' \newline
\verb'       {smallbox {plain-text "very large"}}}'
\end{quote}

Again, {\tt Stack} would do the same with a bit more space between the
boxes.
 
\begin{figure}[hb]
\begin{center}
\leavevmode
\epsfbox{threebox.ps}
\caption{Three boxes ``stacked'' on each other.}
\label{threebox}
\end{center}
\end{figure}

\subsection{Arranging space}

Stacks and sequences can be used to order objects in almost any way in
the display area. However, they do not allow you to arrange for
additional (or less) space around or between objects. With {\tt
vspace}, {\tt hspace} and {\tt space} you can add or subtract room
between objects. These commands basically draw ``invisible'' objects
which take some space in {\sc Clig}'s drawing area. Both {\tt hspace}
and {\tt vspace} consume one parameter, while {\tt space} needs two
integers as parameters. The following description string puts the tiny box
floating 6 points over the other two.

\begin{quote}
\verb'{stack {smallbox {plain-text "tiny"}}' \newline
\verb'       {vspace 6}' \newline
\verb'       {smallbox {plain-text "bigger"}}' \newline
\verb'       {smallbox {plain-text "very large"}}}'
\end{quote}   

\subsection{Sub- and Superpositioning}

The {\tt sub} and {\tt super} commands provide a general approach to
labeling objects with others. They can for example be used for
subscribing or superscribing with texts:

\begin{quote}
\begin{verbatim}
{sub {slant-plain-text X} {small-plain-text 1}}
{super {slant-plain-text X} {small-plain-text 2}}
\end{verbatim}
\end{quote}

The first description string will result in $X_1$ while the second will
be interpreted as $X^2$. Using {\tt sub} and {\tt super} can cause
problems when used together with {\tt seq} or other sequence building
commands. The centering {\tt seq} does will lead to objects being out
of position like in the following example:

\begin{quote}
\begin{verbatim}
{seq {sub {slant-plain-text A} {small-plain-text 1}} {plain-text B}}
\end{verbatim}
\end{quote}

In this example, the $A$ will come out too high compared with the $B$
because $A_1$ will be centered according to its complete size. To
circumvent this problem, you can use the {\tt subscript} and {\tt
superscript} commands which do exactly the same as {\tt sub} and {\tt
super} but will reserve extra space on top or bottom of the object to
force a correct centering. Note that you might waste some space in this
way.


\subsection{Using colors}

Generally, {\sc Clig} uses black as the main color for drawing objects.
The {\tt color} command encapsulates an object and forces it to be
drawn in some other color you choose. If you have a color display, you
could produce a red text within a blue box with the following example:
 
\begin{quote}
\verb'{color blue {bigbox color red  plain-text "colors"}}' \newline
\end{quote}   

The color name used can be any arbitrary color which is defined in 
the {\tt rgb.txt} in your X library directory.

If you want to change the background of any object, you can use
the {\tt color-area} command which draws the object on a colored
rectangular region:

\begin{quote}
\verb'{color-area blue {bigbox color red  plain-text "colors"}}' \newline
\end{quote}

\subsection{Trees}

{\sc Clig}'s trees are built with the {\tt tree} command. The command
takes one parameter---the parent node---and an arbitrary number of
children. Each child may either be another tree or a leaf consisting of
any object drawable by the grapher. The next example represents a
syntactical analysis expressed as a tree (see fig. \ref{postman}):



\begin{quote}
\begin{verbatim}
  {tree {plain-text S} 
        {tree {plain-text NP} 
              {tree {plain-text DET}
                    {plain-text The}} 
              {tree {plain-text N} 
                    {plain-text postman}}} 
        {tree {plain-text VP} 
              {tree {plain-text V} 
                    {plain-text bites}} 
              {tree {plain-text NP} 
                    {tree {plain-text DET} 
                          {plain-text the}} 
                          {tree {plain-text N} 
                          {plain-text dog}}}}}
\end{verbatim}
\end{quote}

\begin{figure}[h]
\begin{center}
\leavevmode
\epsfbox{postman.ps}
\caption{A sample syntactical tree.}
\label{postman}
\end{center}
\end{figure}

If you want horizontally oriented trees, just use {\tt htree} instead
of {\tt tree}.



\subsection{Feature structures}

{\em Feature Structures\/} are used mostly by unification based grammar
formalisms such as HPSG \footnote{see \cite{Pollard 87}}. Graphically
represented as an {\it Attribute Value Matrix}, a feature structure
consists of a square bracket around a set of label-value-pairs. A typed
feature structure may also have a type name in its brackets
or---depending on the notation---in front of it. To realize feature
structures, {\sc Clig} provides the {\tt fs} and {\tt feature}
commands. Figure \ref{fs1} illustrates the following nested typed
feature structure:

\begin{quote}
\begin{verbatim}
{fs {plain-text "word"}
         {feature "PHON" {plain-text <she>}}
         {feature "SYNSEM" 
           {fs {plain-text synsem}
               {feature "LOCAL"
                {fs {plain-text local}
                  {feature "CATEGORY" 
                    {fs {plain-text head}
                        {feature "HEAD" 
                          {fs {feature "CASE" 
                                       {plain-text nom}}}}
                        {feature  "SUBCAT" {plain-text <>}}}}
                  {feature "CONTENT" 
                    {fs {plain-text "ppro"}}}}}}}}
\end{verbatim}
\end{quote} 

\begin{figure}[H]
\begin{center}
\leavevmode
\epsfbox{fs1.ps}
\caption{A feature structure.}
\label{fs1}
\end{center}
\end{figure}

\subsection{Complex Boxes}

{\it Complex Boxes\/} are those graphical objects which are used
in the DRT and Situation Theory:

\begin{description}
\item[{\tt drs}:] draws an object $a$ above a set of objects $c_1 ...
c_n$ and draws a box around $a$ and a box around $c_1 ... c_n$. (see
figure \ref{drs-cb})

\begin{quote}
\begin{verbatim}
{drs {SEQ {plain-text "a"} {plain-text "x"}} 
     {plain-text "dog(a)"} 
     {plain-text "Jones(x)"}
     {plain-text "owns(x,a)"}}
\end{verbatim}
\end{quote} 

\begin{figure}[h]
\begin{center}
\leavevmode
\epsfbox{drs_cb.ps}
\caption{The {\tt drs} command}
\label{drs-cb}
\end{center}
\end{figure}

\item[{\tt cornerbox}:] is almost like {\tt drs}, but object $a$ gets
drawn in the northwest corner of the main box (figure \ref{corner-cb}).

\begin{quote}
\begin{verbatim}
{cornerbox {plain-text "This is in the corner..."} 
           {plain-text "...and this is below"}}
\end{verbatim}
\end{quote}

\begin{figure}[hb]
\begin{center}
\leavevmode
\epsfbox{corner_cb.ps}
\caption{A cornerbox}
\label{corner-cb}
\end{center}
\end{figure}

\item[{\tt followboxes}:] draws an object $a$ followed by a sequence of
objects $c_1 ... c_n$ where each $c_x$ is drawn in a box aligned to the
side of $a$ (see figure \ref{follow-cb}).

\begin{quote}
\begin{verbatim}
{followboxes {drs {plain-text "x"} {plain-text "dog(x)"}}
                  {plain-text "this follows the drs"}}
\end{verbatim}
\end{quote}

\begin{figure}[ht]
\begin{center}
\leavevmode
\epsfbox{follow_cb.ps}
\caption{Followboxes}
\label{follow-cb}
\end{center}
\end{figure}

\item[{\tt stackboxes}:] puts a set of objects $o_1 ... o_n$ each in a box
and draws them above an object $z$ (figure \ref{stack-cb}).

\begin{quote}
\begin{verbatim}
{stackboxes {plain-text "highest"} 
            {plain-text "higher"}
            {cornerbox {plain-text "cornerbox"} 
                       {plain-text "lowest"}}}
\end{verbatim}
\end{quote}

\begin{figure}[hb]
\begin{center}
\leavevmode
\epsfbox{stack_cb.ps}
\caption{The {\tt stackboxes} command}
\label{stack-cb}
\end{center}
\end{figure}

\item[{\tt stairboxes}:] draws boxes like {\tt stackboxes}, but also
indents the boxes on the left-hand side (figure \ref{stair-cb}).

\begin{quote}
\begin{verbatim}
{stairboxes {plain-text "highest"} 
            {plain-text "higher"}
            {cornerbox {plain-text "cornerbox"} 
                       {plain-text "lowest"}}}
\end{verbatim}
\end{quote}

\begin{figure}[hb]
\begin{center}
\leavevmode
\epsfbox{stair_cb.ps}
\caption{{\tt Stairboxes} indent boxes.}
\label{stair-cb}
\end{center}
\end{figure}

\end{description}

\subsection{Interaction}

Each object in the drawing area can be linked to a piece of {\sc Tcl}
code by using the commands {\tt clickable} and {\tt active}. While {\tt
clickable} simply performs one command which gets executed when the
user double-clicks on the associated object, {\tt active} can define
multiple possibilities for the same object or define special
behaviors. \cite{Konrad 95b} explains in a detailed way how you can
define interactive behavior within graphs. The following example
creates a mouse sensitive text-box which produces a short message with
the Tcl command {\tt puts} when clicked on:

\begin{quote}
\begin{verbatim}
{clickable {bigbox plain-text "click me!"} 
           {puts "YOU CLICKED ME!"}}
\end{verbatim}
\end{quote}

{\tt active} allows the use of several event-command pairs for the same
object. The general syntax for {\tt active} is 

\begin{quote}
\begin{verbatim}
{active <object> {<event> <command>} ... {<event> <command>}}
\end{verbatim}
\end{quote}

An event is an event description {\sc Tk} uses, e.g. \verb'<Double-2>'
for a double click on the middle mouse button. If you wanted the
message from the last example to appear when the middle mouse button
gets clicked once, you would express it like this: 

\begin{quote}
\begin{verbatim}
{active {bigbox plain-text "click me!"} 
        {<2> {puts "YOU CLICKED ME!"}}}
\end{verbatim}
\end{quote}

\subsubsection{Popup menues}

Popup menues are a standard feature of {\sc Tk} which have been
included into the {\tt active} command in recent {\sc Clig} versions
(Version 1.4 or greater). {\tt Active} accepts the keyword {\tt popup}
in a event-command-pair and uses the command part as a definition of a
popup menu as described in \cite{Ousterhout 94}. The following example
defines a box which shows a popup menu when clicked on. When you select
one of the buttons in it, it will print a short message.

\begin{quote}
\begin{verbatim}
{active {bigbox {color red {plain-text "CLICK ME!"}}}
             {popup <1> {{command -label "TEST1" 
                                -command "puts TEST1"}
                         {command -label "TEST2" 
                                -command "puts TEST2"}
                         {separator}
                         {command -label "TEST3" -command "puts
             TEST3"}}}}
\end{verbatim}
\end{quote}


\subsection{Other commands}

{\sc Clig} provides additional graphical objects for some special
applications, e.g. Intentional Logic. The table of commands in section
\ref{table} summarizes all commands.  

\subsection{Table of commands}
\label{table}

The following list summarizes the predefined graphical objects you can use
in the description string along with their parameters ($s$ stands for
strings, $d$ for object descriptions, $c$ for Tcl-scripts and $e$ for a
event-command pair):


\begin{center}
\begin{tabular}{l|l|l}
   Name & Description & Parameters \\ \hline
   {\tt active} & links object to events & $d$ $e_1$ ... $e_n$ \\
   {\tt anglelist} & draws objects as a list & $d_1$ ... $d_n$ \\ 
   {\tt bigbox} & draws box with large frame (6 pts) around object & $d$ \\
   {\tt bold-text} & draws text in bold font & $s$ \\
   {\tt boxed} & draws box with frame size $x$ around object & $x$ $d$ \\
   {\tt bracket} & draws round brackets around object & $d$ \\
   {\tt clickable} & links object to command & $d$ $c$ \\
   {\tt color} & changes foreground color for $d$ & $color$ $d$ \\
   {\tt color-area} & changes background color for $d$ & $color$ $d$ \\ 
   {\tt cornerbox} & like {\tt drs} with $d_{top}$ in the top left corner &
     $d_{top}$ $d_1$ ... $d_n$ \\
   {\tt diamond} & draws square diamond around object & $d$ \\
   {\tt drs} & DRS-like structure & $d_{top}$ $d_1$ ... $d_n$ \\  
   {\tt feature} & draws single feature-value pair & $string$ $d$ \\
   {\tt followboxes} & draws followboxes & $d_{main}$ $d_1$ ... $d_n$ \\
   {\tt framed} & leaves extra space $x$ around object & $x$ $d$ \\
   {\tt fs} & feature structure & $d_1$ ... $d_n$ \\
   {\tt htree} & like tree, but from left to right & $d_{parent}$
     $d_{child-1}$ ... $d_{child-n}$ \\
   {\tt imp} & implication & $d_1$ $d_2$ \\
   {\tt leftstack} & same as stack, but without centering & $d_1$ ...
     $d_n$ \\
   {\tt leftStack} & same as leftstack, with more space between objects
     & $d_1$ ... $d_n$ \\
   {\tt neg} & negates object & $d$ \\
   {\tt plain-text} & draws text & $s$ \\
   {\tt Seq} & draws horizontal sequence of objects & $d_1$ ... $d_n$ \\
   {\tt seq} & same as Seq, but with minimum space & $d_1$ ... $d_n$ \\
   {\tt SEQ} & same as Seq, with more space (see {\tt imp}) &
         $d_1$ ... $d_n$ \\ 
   {\tt sign} & displays some special signs & $name$ \\
   {\tt slant-text} & draws a text in {\tt italics} & $string$ \\ 
   {\tt smallbox} & draws box with small frame (2 pts) around object &
$d$ \\
   {\tt squarebracket} & draws square brackets around object & $d$ \\
   {\tt Stack} & stacks objects on top of each other & $d_1$ ... $d_n$ \\
   {\tt stack} & same as Stack, but with minimum space & $d_1$ ...
     $d_n$ \\
   {\tt stackboxes} & draws stackboxes & $d_1$ ... $d_n$ $d_{lowest}$
\\
   {\tt sub} & draws $d_2$ subpositioned to $d_1$ & $d_1$ $d_2$ \\
    {\tt subscrupt} &  draws $d_2$ superpositioned to $d_1$ & $d_1$ $d_2$
\\ & with forced correct centering & \\

   {\tt super} &  draws $d_2$ superpositioned to $d_1$ & $d_1$ $d_2$ \\
    {\tt super} &  draws $d_2$ superpositioned to $d_1$ & $d_1$ $d_2$
\\ & with forced correct centering & \\

   {\tt tree} & draws tree & $d_{parent}$ $d_{child-1}$ ...
     $d_{child-n}$ \\
   {\tt underline} & draws a line under the object & $d$ \\
   \hline
\end{tabular}
\end{center}

\section{What is not in this paper: Future works}

The purpose of this paper was to introduce {\sf Clig} as as a pure
displaying tool for graphical information and to show its graphical
capabilities. It therefore does not tell you in detail

\begin{itemize}
\item how you can extend {\sc Clig} with your own graphical objects
\item how you can make graphs interactive
\item how applications can use {\sc Clig} 
\end{itemize}

The topics of interaction and extendibility are dealt with in the paper
\cite{Konrad 95b} which will also give some detailed information about
the state of the interfaces.
  
\newpage
\begin{thebibliography}{kako}
   \bibitem[Fracas 95]{Fracas 95}
        Robin Cooper, Dick Crouch, Jan van Eijck, Chris Fox,
        Josef van Genabith, Jan Jaspars, Hans Kamp,
        Manfed Pinkal, Massimo Poesio and Steve Pulman.
        {\em Evaluating the State of the Art}. FraCaS Deliverable D10.
        University of Edingburgh, Centre for Cognitive Sciences. 
   \bibitem[Kamp 93]{kamp 93} Hans Kamp and Uwe Reyle.
        {\em From Discourse to Logic}.  Kluwer Academic Press,
        Dordrecht - Boston - London, 1993 
   \bibitem[Konrad 95b]{Konrad 95b} Karsten Konrad. {\em Extending {\sc Clig}:
           Interaction and User Defined Graphics}. 
           unpublished draft, 1995.
   \bibitem[Ousterhout 94]{Ousterhout 94} John K. Ousterhout. {\em Tcl
           and the Tk Toolkit\/}.  Reading, Massachusetts - Menlo Park,
           California - New York, Addison-Wesley Professional Computing
           Series, 1994.  
   \bibitem[Pollard 87]{Pollard 87} Carl Pollard
           and Ivan A. Sag. {\em Information-Based Syntax and
           Semantics. Volume 1: Fundamentals}. CSLI Lecture Notes,
           Menlo Park - Stanford - Palo Alto, Center for the Study of 
           Language and Information (CSLI), 1987
   \bibitem[Sicstus 95]{Sicstus 95} Programming Systems Group of the
           Swedish Institute of Computer Science. {\em SICStus Prolog
           User's Manual}. Release 3. Swedish Institute of Computer 
           Science, 1995.
\end{thebibliography}

\end{document} 
