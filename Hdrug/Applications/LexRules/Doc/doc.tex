\def\sxc#1{\sc #1}

\long\def\plcmdfm{$
\avm{
 \mbox{\sc sign}  \\ 
 \mbox{\it lex} : \mbox{\sc phrasal}  \\ 
 \mbox{\it cat} :\avm[{\mbox{H}}]{ \mbox{\sc vp}  \\ 
 \mbox{\it ipp} : \mbox{\sc +}  \\ 
 \mbox{\it vform} : \mbox{\sc fin} } \\ 
 \mbox{\it inv} : \mbox{\sc -} }$
}
\long\def\plcmdfn{$
\avm[{\mbox{L}}]{ \mbox{\sc sign}  \\ 
 \mbox{\it lex} :\avm{
 \mbox{\sc lexical}  \\ 
 \mbox{\it stem} : \mbox{jan}  \\ 
 \mbox{\it word} : \mbox{jan} } \\ 
 \mbox{\it cat} :\avm{
 \mbox{\sc np}  \\ 
 \mbox{\it case} : \mbox{\sc nom} }}$
}
\long\def\plcmdfo{$
\avm[{\mbox{P}}]{ \mbox{\sc sign}  \\ 
 \mbox{\it lex} :\avm{
 \mbox{\sc lexical}  \\ 
 \mbox{\it stem} : \mbox{marie}  \\ 
 \mbox{\it word} : \mbox{marie} } \\ 
 \mbox{\it cat} :\avm{
 \mbox{\sc np}  \\ 
 \mbox{\it case} : \mbox{\sc obj} }}$
}
\long\def\plcmdfp{$
\avm{
 \mbox{\sc sign}  \\ 
 \mbox{\it lex} :\avm{
 \mbox{\sc lexical}  \\ 
 \mbox{\it stem} : \mbox{wil}  \\ 
 \mbox{\it word} : \mbox{wil} } \\ 
 \mbox{\it sc} :\langle \avm[{\mbox{W}}]{ \mbox{\sc sign}  \\ 
 \mbox{\it lex} :\avm{
 \mbox{\sc lexical}  \\ 
 \mbox{\it stem} : \mbox{kan}  \\ 
 \mbox{\it word} : \mbox{kunnen} } \\ 
 \mbox{\it sc} :\langle \avm[{\mbox{Z}}]{ \mbox{\sc sign}  \\ 
 \mbox{\it lex} :\avm{
 \mbox{\sc lexical}  \\ 
 \mbox{\it stem} : \mbox{kus}  \\ 
 \mbox{\it word} : \mbox{kussen} } \\ 
 \mbox{\it sc} :\mbox{B}_{1}~\langle \mbox{P}\rangle  \\ 
 \mbox{\it cat} :\avm{
 \mbox{\sc vp}  \\ 
 \mbox{\it ipp} : \mbox{\sc -}  \\ 
 \mbox{\it vform} : \mbox{\sc inf} }} | \mbox{B}_{1}\rangle  \\ 
 \mbox{\it cat} :\avm{
 \mbox{\sc vp}  \\ 
 \mbox{\it ipp} : \mbox{\sc +}  \\ 
 \mbox{\it vform} : \mbox{\sc inf} }} , \mbox{Z} , \mbox{P} , \mbox{L}\rangle  \\ 
 \mbox{\it cat} :\mbox{H} \\ 
 \mbox{\it inv} : \mbox{\sc -} }$
}
\long\def\plcmdfq{$
\mbox{W}$
}
\long\def\plcmdfr{$
\mbox{Z}$
}
\long\def\plcmdfa{$
 \mbox{apply} $
}
\long\def\plcmdfb{$
 \mbox{jan1} $
}
\long\def\plcmdfc{$
 \mbox{marie1} $
}
\long\def\plcmdfd{$
 \mbox{wil2} $
}
\long\def\plcmdfe{$
 \mbox{kunnen0} $
}
\long\def\plcmdff{$
 \mbox{kussen2} $
}
\long\def\plcmdev{$
 \mbox{topicalize} $
}
\long\def\plcmdew{$
 \mbox{jan0} $
}
\long\def\plcmdex{$
 \mbox{verbfront} $
}
\long\def\plcmdey{$
 \mbox{kust0} $
}
\long\def\plcmdez{$
 \mbox{vrouwen1} $
}
\long\def\plcmdeq{$
\avm{
 \mbox{\sc sign}  \\ 
 \mbox{\it cat} :\avm[{\mbox{E}}]{ \mbox{\sc vp}  \\ 
 \mbox{\it ipp} : \mbox{\sc -}  \\ 
 \mbox{\it vform} : \mbox{\sc fin} } \\ 
 \mbox{\it inv} : \mbox{\sc +}  \\ 
 \mbox{\it f\_slash} :\mbox{H}~\langle \avm{
 \mbox{\sc sign}  \\ 
 \mbox{\it cat} :\avm[{\mbox{J}}]{ \mbox{\sc np}  \\ 
 \mbox{\it case} : \mbox{\sc nom} } \\ 
 \mbox{\it inv} :\mbox{B}}\rangle }$
}
\long\def\plcmder{$
\avm{
 \mbox{\sc sign}  \\ 
 \mbox{\it lex} :\avm{
 \mbox{\sc lexical}  \\ 
 \mbox{\it word} : \mbox{jan} } \\ 
 \mbox{\it cat} :\mbox{J} \\ 
 \mbox{\it inv} :\mbox{B}}$
}
\long\def\plcmdes{$
\avm{
 \mbox{\sc sign}  \\ 
 \mbox{\it cat} :\mbox{E} \\ 
 \mbox{\it inv} : \mbox{\sc +}  \\ 
 \mbox{\it slash} :\mbox{H}}$
}
\long\def\plcmdet{$
\avm{
 \mbox{\sc sign}  \\ 
 \mbox{\it lex} :\avm{
 \mbox{\sc lexical}  \\ 
 \mbox{\it word} : \mbox{kust} } \\ 
 \mbox{\it sc} :\langle \avm{
 \mbox{\sc sign}  \\ 
 \mbox{\it cat} :\avm[{\mbox{W}}]{ \mbox{\sc np}  \\ 
 \mbox{\it case} : \mbox{\sc obj} } \\ 
 \mbox{\it inv} :\mbox{A}}\rangle  \\ 
 \mbox{\it cat} :\mbox{E} \\ 
 \mbox{\it inv} : \mbox{\sc +}  \\ 
 \mbox{\it slash} :\mbox{H}}$
}
\long\def\plcmdeu{$
\avm{
 \mbox{\sc sign}  \\ 
 \mbox{\it lex} :\avm{
 \mbox{\sc lexical}  \\ 
 \mbox{\it word} : \mbox{vrouwen} } \\ 
 \mbox{\it cat} :\mbox{W} \\ 
 \mbox{\it inv} :\mbox{A}}$
}



\long\def\comment#1{}
\documentstyle[gj_bib_art,a4wide,matrix,exam,prolog,fleqn]{article}
\input{tmaker}
\setlength{\mathindent}{0ex}
\begin{document}
\bibliographystyle{named}
\title{A Head-driven Unification Grammar for Dutch}
\author{Gertjan van Noord}
\date{Vakgroep Alfa-informatica RUG\\
Groningen \\ vannoord@let.rug.nl}
\maketitle


\begin{abstract}
This paper documents a Head-driven Unification Grammar (HUG) for Dutch. The grammar covers phenomenon
such as verb-raising and clause union, topicalization, verb-second,
relative clauses, extraposition and modification.

The grammar heavily uses relational constraints to define lexical
rules.  
\end{abstract}

\section{Overview}

\subsection{Formalism}

cf. \shortcite{vannoord-diss}.

\subsection{Head and Foot feature principles}

Grammar rules are organised in a small inheritance hierarchy in order
to capture generalizations over rules. Rules are defined as
feature-structures of type {\sc rule} with the attributes {\it hd},
{\it mt}, {\it ls} and {\it rs}.
The value of the attribute {\it hd} consists of the
feature structure of the head of the rule, the value of 
{\it mt} is the result of the application, the value of {\it ls}
is a list of feature structures that precede the head (in
right-to-left order) and the value of {\it rs} gives a list of
feature-structures that are to follow the head (in left-to-right
order). 

In  rules we do not assume that the order is
determined by separate LP statements, but for the moment we assume
that, as in Categorial Grammar, elements have a feature {\it dir}
whose value determines whether an argument should precede the head or
be preceded by the head. 

The most important generalization of the current grammar are two
principles that take care of feature percolation: the head feature
principle and the foot feature principle.  The purpose of these
principles is as usual. The head-feature principle is defined as
follows.

\[
\mbox{ hfp}(\avm{
 \mbox{\sc rule}  \\ 
 \mbox{\it mt} :\avm{
 \mbox{\sc sign}  \\ 
 \mbox{\it cat} : \mbox{F}_{ 0 }  \\ 
 \mbox{\it inv} : \mbox{G}_{ 0 }  \\ 
 \mbox{\it subj} : \mbox{H}_{ 0 }  \\ 
 \mbox{\it sem} : \mbox{I}_{ 0 } } \\ 
 \mbox{\it hd} :\avm{
 \mbox{\sc sign}  \\ 
 \mbox{\it cat} : \mbox{F}_{ 0 }  \\ 
 \mbox{\it inv} : \mbox{G}_{ 0 }  \\ 
 \mbox{\it subj} : \mbox{H}_{ 0 }  \\ 
 \mbox{\it sem} : \mbox{I}_{ 0 } }}).
\]

In this schema the feature {\it sem} represents the semantic
structure, the feature {\it cat} represents syntactic structure,
the feature {\it subj} represents the subject and the feature
{\it inv} has a boolean value to indicate whether or not the structure
is inverted. This feature is used in the analysis of verb-second (cf.
below). 

The foot feature principle on the other hand is responsible for the
percolation of the {\it slash}, {\it extra} and {\it rel} features.
These features are used in the analysis of topicalization,
extraposition and relative clauses, and can be compared with the
respective `inherited' features in \shortcite{hpsg2}. The features {\it f\_slash},
{\it f\_extra} and {\it f\_rel} are the counterparts of the `to-bind'
features.

\[
\mbox{ ffp}(\avm{
 \mbox{\sc rule}  \\ 
 \mbox{\it mt} :\avm{
 \mbox{\sc sign}  \\ 
 \mbox{\it slash} :  \mbox{Y}_{ 0 } - \mbox{M}_{ 0 }  \\ 
 \mbox{\it extra} :  \mbox{Z}_{ 0 } - \mbox{N}_{ 0 }  \\ 
 \mbox{\it rel} :    \mbox{A}_{ 1 } - \mbox{O}_{ 0 }  \\ 
 \mbox{\it f\_slash} : \mbox{M}_{ 0 }  \\ 
 \mbox{\it f\_extra} : \mbox{N}_{ 0 }  \\ 
 \mbox{\it f\_rel} : \mbox{O}_{ 0 } } \\ 
 \mbox{\it ls} : \mbox{S}_{ 0 }  \\ 
 \mbox{\it hd} : \mbox{T}_{ 0 }  \\ 
 \mbox{\it rs} : \mbox{U}_{ 0 } }) \mbox{\tt :-}
~  \mbox{ collect\_ffp} ( \mbox{S}_{ 0 } \cdot \langle  \mbox{T}_{ 0 }  |  \mbox{U}_{ 0 } \rangle , \mbox{Y}_{ 0 } , \mbox{Z}_{ 0 } , \mbox{A}_{ 1 } ).
\]

Note that we often write $rel(A,B)$ for a value $C$ such that the
relation $rel(A,B,C)$ holds. Moreover, if the relation is the
concatenation relation, then we use the `dot' in infix notation. 
Furthermore, we write $\mbox{A} - \mbox{B}$ for the value $\mbox{C}$
such that { difference(A,B,C)}. This relation is true if $\mbox{C}$
is a list of which each of the elements occurs either in the list
$\mbox{A}$ or $\mbox{B}$.

In the foot-feature principle we simply collect the slashed elements
of all the daughters
and percolate all of them except the ones that
are found (and hence are represented at {\it f\_slash}). The same
holds for the {\it extra} and {\it rel} values.

\subsection{Five rules for Dutch}

Currently there are five rules in the grammar for Dutch. These rules
respectively deal with ordinary application, verb-fronting,
topicalization, relativization and clitics (seperable prefixes).
Each of the rules extend both the foot feature principle and the head
feature principle shown above. This is expressed by the relation ${
rule}$ which is called in every rule (i.e. every rule is of type
{\sc rule}).

\[
 \mbox{ rule} ( \mbox{B}_{ 0 } ) \mbox{\tt :-} 
~ \mbox{ ffp} ( \mbox{B}_{ 0 } ), 
~ \mbox{ hfp} ( \mbox{B}_{ 0 } ).
\]

\subsubsection{Application}
First consider the application rule that is defined as follows.

\[
\mbox{ r}(\avm[\mbox{A}_0]{
 \mbox{\sc apply}  \\ 
 \mbox{\it mt} :\avm{
 \mbox{\sc sign}  \\ 
 \mbox{\it lex} : \mbox{\sc phrasal}  \\ 
 \mbox{\it sc} : \mbox{E}_{ 0 }  \\ 
 \mbox{\it f\_slash} :[~] \\ 
 \mbox{\it f\_extra} :[~] \\ 
 \mbox{\it f\_rel} :[~]} \\ 
 \mbox{\it ls} : \mbox{S}_{ 0 }  \\ 
 \mbox{\it hd} :\avm{
 \mbox{\sc sign}  \\ 
 \mbox{\it lex} : \mbox{\sc lexical}  \\ 
 \mbox{\it sc} : \langle\mbox{N}_{ 1 }|\mbox{N}_{ 2 }\rangle \cdot \mbox{E}_{ 0 } \\ 
 \mbox{\it inv} : \mbox{\sc -} } \\ 
 \mbox{\it rs} : \mbox{L}_{ 1 } })
\mbox{\tt :-} ~ \mbox{ rule}(\mbox{A}_0), ~  \mbox{ extract\_args} ( \langle\mbox{N}_{ 1 }|\mbox{N}_{ 2 }\rangle , \mbox{S}_{ 0 } , \mbox{L}_{ 1 } ).
\]

The interesting information in this rule is the following. Firstly the
subcat list of the head is split in two parts. The first part consists
of the elements that are selected by this rule. The relation {
extract\_args} selects two lists of arguments on the basis of this subcat
list. Depending on the directionality each argument is placed in one
of these lists representing the daughters left and right of the head.
The elements in the subcat list that are to the left will eventually
occur closer to the head. 

This rule will almost always be used in such a way that all elements
are selected at once. Only in some special constructions to be
discussed later `partial' constituents are allowed. Note that the
application rule can not be applied recursively because the head
daughter is constrained to be of type lexical, whereas the mother is
of phrasal type. These types are inconsistent.

Also note that the rule may not apply vacuously as the list of
daughters is supposed to be non-empty. This implies that categories
from the lexicon are a supertype of both lexical and phrasal (of type
{\sc sign}).  As we will explain below this analysis gives rise to
partial vp-topicalizations along the lines of
\shortcite{non-head-movement} but without the problem Pollard has with
spurious ambiguities.

Finally note that the head daughter is marked as non-inverted. This is
the case because we use a special rule for sentences in which the verb
has been fronted.

\subsubsection{Verb-fronting}

The rule we use for verb-second structures in Dutch is a variant of
the previous rule. The rule is defined as follows:

\[
\mbox{ r}(\avm[\mbox{A}_0]{
 \mbox{\sc verbfront}  \\ 
 \mbox{\it mt} :\avm{
 \mbox{\sc sign}  \\ 
 \mbox{\it lex} : \mbox{\sc phrasal}  \\ 
 \mbox{\it sc} :[~] \\ 
 \mbox{\it f\_slash} :[~] \\ 
 \mbox{\it f\_extra} :[~] \\ 
 \mbox{\it f\_rel} :[~]} \\ 
 \mbox{\it ls} :[~] \\ 
 \mbox{\it hd} :\avm{
 \mbox{\sc sign}  \\ 
 \mbox{\it lex} : \mbox{\sc lexical}  \\ 
 \mbox{\it sc} : \langle\mbox{X}_{ 0 }|\mbox{X}_{1}\rangle\\ 
 \mbox{\it cat} :\avm{
 \mbox{\sc vp}  \\ 
 \mbox{\it vform} : \mbox{\sc fin} } \\ 
 \mbox{\it inv} : \mbox{\sc +} } \\ 
 \mbox{\it rs} : \mbox{ reverse}(\mbox{L}_0)\cdot\mbox{R} })
\mbox{\tt :-} 
~ \mbox{ rule}(\mbox{A}_0), ~ \mbox{ extract\_args}(\langle\mbox{X}_{ 0 }|\mbox{X}_{1}\rangle,\mbox{L}_0,\mbox{R} ).
\]

The relation {\it extract\_args} again is used to find the left- and
right-directed arguments from the subcat list of the head. Note that
the list of left-directed arguments is inverted and concatenated to
the list of right-directed arguments. As there is no reason for
partial inverted verb phrases that we are aware of, we require that
all elements from the subcat list are selected.

\subsubsection{Seperable Prefixes}

As we will explain below it is neccessary to have a version of the
application rule to derive seperable prefixes such as particles.  The
important difference with the ordinary application rule is that after
the selection of the particle the resulting category can be shown to
be `lexical'. The rule is defined as follows.

\[
\mbox{ r}(\avm[\mbox{A}_0]{
 \mbox{\sc cliticize}  \\ 
 \mbox{\it mt} :\avm{
 \mbox{\sc sign}  \\ 
 \mbox{\it lex} : \mbox{\sc lexical}  \\ 
 \mbox{\it sc} : \mbox{G}_{ 0 }  \\ 
 \mbox{\it f\_slash} :[~] \\ 
 \mbox{\it f\_extra} :[~] \\ 
 \mbox{\it f\_rel} :[~]} \\ 
 \mbox{\it ls} :\langle \avm[\mbox{V}_{ 0 }]{ \mbox{\sc
sign}\\\mbox{\it cat}: \mbox{\sc part}} \rangle\\ 
 \mbox{\it hd} :\avm{
 \mbox{\sc sign}  \\ 
 \mbox{\it lex} : \mbox{\sc lexical}  \\ 
 \mbox{\it sc} :\langle  \mbox{V}_{ 0 }  |  \mbox{G}_{ 0 } \rangle  \\ 
 \mbox{\it cat} : \mbox{\sc vp}  \\ 
 \mbox{\it inv} : \mbox{-} \\}\\ 
 \mbox{\it rs} :[~]})
 \mbox{\tt :-} 
~ \mbox{ rule}(\mbox{A}_0). \]

As we currently only use the rule for the selection of seperable
prefixes, the rule is not as general as it might be. 

\subsubsection{Topicalization}

The rule for topicalization consists of a topic and a verb-phrase that
misses that topic as an element of the slash list. The rule is defined
as follows:

\[
\mbox{ r}(\avm[\mbox{A}_0]{
 \mbox{\sc topicalize}  \\ 
 \mbox{\it mt} :\avm{
 \mbox{\sc sign}  \\ 
 \mbox{\it lex} : \mbox{\sc phrasal}  \\ 
 \mbox{\it sc} :[~] \\ 
 \mbox{\it slash} :[~] \\ 
 \mbox{\it extra} :[~] \\ 
 \mbox{\it rel} :[~] \\ 
 \mbox{\it f\_slash} :\mbox{M}_{ 0 }\langle \avm{
 \mbox{\sc sign}  \\ 
 \mbox{\it slash} :[~] \\ 
 \mbox{\it extra} :[~] \\ 
 \mbox{\it rel} : \mbox{W}_{ 0 } }\rangle  \\ 
 \mbox{\it f\_extra} :[~] \\ 
 \mbox{\it f\_rel} : \mbox{W}_{ 0 } } \\ 
 \mbox{\it ls} : \mbox{M}_{ 0 }  \\ 
 \mbox{\it hd} :\avm{
 \mbox{\sc sign}  \\ 
 \mbox{\it sc} :[~] \\ 
 \mbox{\it cat} :\avm{
 \mbox{\sc vp}  \\ 
 \mbox{\it vform} : \mbox{\sc fin} }} \\ 
 \mbox{\it rs} :[~]})
\mbox{\tt :-} 
~ \mbox{ rule}(\mbox{A}_0). \]

Note that the foot-feature principle should take care of the remaining
feature equations that are going on in this rule. 

The rule will be used both for topicalization in main clauses, and for
relative clauses. This is the reason that we only insist on a finite
verb-phrase --- and not on an inverted verb-phrase. Depending on the
availability of a non-empty {\it rel}-value we will instantiate the
to-bind feature for {\it rel}. Cf. below.

\subsubsection{Extraposition}

Finally we present the extraposition rule. Its intended use is to
generate verb-phrases with extraposed complements and adjuncts. Note
that the rule will also be used to obtain discontinuous constituents
as in the following cases:

\begin{exams}
\item dat jan het boek heeft gelezen van Vestdijk
\item dat jan het boek heeft gelezen dat Piet hem had aangeraden
\item dat jan over de verwoesting heeft gelezen van Carthago
\end{exams}

The rule is a mirror image of the topicalization rule. An important
difference is that this rule will often be applied successively as
there is no upper-bound on the number of extraposed elements.

\[
\mbox{ r}(\avm[\mbox{A}_0]{
 \mbox{\sc extrapose}  \\ 
 \mbox{\it mt} :\avm{
 \mbox{\sc sign}  \\ 
 \mbox{\it lex} : \mbox{\sc phrasal}  \\ 
 \mbox{\it sc} :[~] \\ 
 \mbox{\it f\_slash} :[~] \\ 
 \mbox{\it f\_extra} :\mbox{N}_{ 0 }\langle \avm{
 \mbox{\sc sign}  \\ 
 \mbox{\it extra} :[~] \\ 
 \mbox{\it dir} : \mbox{\sc extra} }\rangle  \\ 
 \mbox{\it f\_rel} :[~]} \\ 
 \mbox{\it ls} :[~] \\ 
 \mbox{\it hd} :\avm{
 \mbox{\sc sign}  \\ 
 \mbox{\it sc} :[~] \\ 
 \mbox{\it cat} : \mbox{\sc vp} } \\ 
 \mbox{\it rs} : \mbox{N}_{ 0 } })
\mbox{\tt :-} 
~ \mbox{ rule}(\mbox{A}_0). \]

Again note that the list of right-daughters will be identified with
the single element on the {\it f\_extra} list of the mother node by
virtue of the foot feature principle. 

\subsection{Simple lexical entries}

Lexical entries are defined by means of a multiple inheritance
hierarchy where some limited form of non-monotonicity is allowed. In
this paper we present simply the resulting category of the word --
sometimes leaving out details not relevant for the discussion at hand.
Also the structure of the network is often left implicit. 

The entries in the lexicon all heavily make use of lexical
constraints. For example, as will be discussed in the following, we
assume a lexical rule for adjunct-addition to the subcat list, a
lexical rule of topicalization which pushes an element from the subcat
list to slash, and a lexical rule of extraposition which moves an
element from the subcat list to the extraposition list.  The resulting
lexical entry therefore usually is an underspecified feature-structure
with some complex constraints attached to it. These constraints are
evaluated dynamically using delayed evaluation techniques from
constraint-logic-programming.

If we ignore such complex constraints for a minute then the 
lexical entry for the base form
of the Dutch verb `kussen' (`to kiss') is given as follows.
\[
\avm{
 \mbox{\sc sign}  \\ 
 \mbox{\it lex} :\avm{
 \mbox{\sc lexical}  \\ 
 \mbox{\it stem} : \mbox{kus}  \\ 
 \mbox{\it word} :\langle  \mbox{kussen} \rangle } \\ 
 \mbox{\it sc} :\langle \avm{
 \mbox{\sc sign}  \\ 
 \mbox{\it sc} :[~] \\ 
 \mbox{\it cat} :\avm{
 \mbox{\sc np}  \\ 
 \mbox{\it case} : \mbox{\sc obj} } \\ 
 \mbox{\it sem} : \mbox{M}_{ 0 }  \\ 
 \mbox{\it slash} :[~] \\ 
 \mbox{\it rel} :[~] \\ 
 \mbox{\it dir} : \mbox{\sc topic;left} }\rangle  \\ 
 \mbox{\it cat} :\avm{
 \mbox{\sc vp}  \\ 
 \mbox{\it ipp} : \mbox{\sc -}  \\ 
 \mbox{\it vform} : \mbox{\sc inf} } \\ 
 \mbox{\it subj} :\avm{
 \mbox{\sc sign}  \\ 
 \mbox{\it sem} : \mbox{G}_{ 1 } } \\ 
 \mbox{\it sem} :\avm{
 \mbox{\sc c2}  \\ 
 \mbox{\it fun} : \mbox{\sc prime(kus)}  \\ 
 \mbox{\it restr} :[~] \\ 
 \mbox{\it arg1} : \mbox{G}_{ 1 }  \\ 
 \mbox{\it arg2} : \mbox{M}_{ 0 } } \\ 
 \mbox{\it slash} :[~] \\ 
 \mbox{\it extra} :[~] \\ 
 \mbox{\it rel} :[~] \\ 
 \mbox{\it f\_slash} :[~] \\ 
 \mbox{\it f\_extra} :[~] \\ 
 \mbox{\it f\_rel} :[~]}
\]
\noindent
Note how this entry subcategorizes for a noun-prase (its object).
It is indicated that this argument is realized either
by selection to the left or via topicalization. Furthermore note how
the semantics is built lexically. Agreement is taken care of via an
index feature on the semantics. Note that subjects are added to the
subcat-list for finite verbs only.

Note however that the actual definition
of a transitive verb like `kust' is quite simple, because many
generalizations are expressed via the network. Hence the information
that is specific to `kussen' is very limited. This definition reads as
follows:

\[
 \mbox{ verb\_stem} (\avm[{ \mbox{B}_{ 0 } }]{ \mbox{\sc sign}  \\ 
 \mbox{\it lex} :\avm{
 \mbox{\sc lexical}  \\ 
 \mbox{\it stem} : \mbox{kus} }}) \mbox{\tt :-} 
~~~ \mbox{ np\_transitive} ( \mbox{B}_{ 0 } ).
\]
\noindent
Similarly, the definition of { np\_transitive} consists of:

\[
 \mbox{ np\_transitive} (\avm[{ \mbox{B}_{ 0 } }]{ \mbox{\sc sign}  \\ 
 \mbox{\it sc} :\langle \avm{
 \mbox{\sc sign}  \\ 
 \mbox{\it sc} :[~] \\ 
 \mbox{\it cat} :\avm{
 \mbox{\sc np}  \\ 
 \mbox{\it case} : \mbox{\sc obj} } \\ 
 \mbox{\it slash} :[~] \\ 
 \mbox{\it rel} :[~] \\ 
 \mbox{\it dir} : \mbox{\sc topic;left} } | \_\rangle }) \mbox{\tt :-} 
~~~ \mbox{ transitive} ( \mbox{B}_{ 0 } ).
\]
\noindent
which uses the information from `transitive'. This is defined as

\[
 \mbox{ transitive} (\avm[{ \mbox{B}_{ 0 } }]{ \mbox{\sc sign}  \\ 
 \mbox{\it sc} :\langle \avm{
 \mbox{\sc sign}  \\ 
 \mbox{\it sem} : \mbox{K}_{ 0 } }\rangle  \\ 
 \mbox{\it subj} :\avm{
 \mbox{\sc sign}  \\ 
 \mbox{\it sem} : \mbox{D}_{ 1 } } \\ 
 \mbox{\it sem} :\avm{
 \mbox{\sc c2}  \\ 
 \mbox{\it arg1} : \mbox{D}_{ 1 }  \\ 
 \mbox{\it arg2} : \mbox{K}_{ 0 } }}) \mbox{\tt :-} 
~~~ \mbox{ verb} ( \mbox{B}_{ 0 } ).
\]
\noindent
using the information from `verb':

\[
 \mbox{ verb} (\avm[{ \mbox{B}_{ 0 } }]{ \mbox{\sc sign}  \\ 
 \mbox{\it cat} :\avm{
 \mbox{\sc vp}  \\ 
 \mbox{\it ipp} : \mbox{\sc -} }}) \mbox{\tt :-} 
~~~ \mbox{ lexical} ( \mbox{B}_{ 0 } ).
\]
\noindent
Finally, the information from `lexical' simply is:
\[
\avm{
 \mbox{\sc sign}  \\ 
 \mbox{\it lex} : \avm{\mbox{\sc lexical}\\\mbox{\it stem}: A } \\ 
 \mbox{\it sem} :\avm{
 \mbox{\sc cx}  \\ 
 \mbox{\it fun} : \mbox{\sc prime(A)}  \\ 
 \mbox{\it restr} :[~]} \\ 
 \mbox{\it slash} :[~] \\ 
 \mbox{\it extra} :[~] \\ 
 \mbox{\it rel} :[~] \\ 
 \mbox{\it f\_slash} :[~] \\ 
 \mbox{\it f\_extra} :[~] \\ 
 \mbox{\it f\_rel} :[~]}
\]


This completes the definition of the `basic' form of the transitive
verb `kust'. However, as there are all kinds of lexical rules that
operate on this entry this is not the final entry. In fact, we have
the following. Up to now, we have been defining an instance of the
relation { verb\_stem}. However, to obtain a lexical entry, we have:

\[
 \mbox{verb\_entry} (\avm[{\mbox{U}}]{ \mbox{\sc sign}  \\ 
 \mbox{\it lex} :\avm{
 \mbox{\sc lexical}  \\ 
 \mbox{\it stem} :\mbox{O} \\ 
 \mbox{\it word} :\mbox{P}}}){\mbox{\tt :-}}\]
\vspace{-4.5ex}\par\[
~~~~ \mbox{verb\_stem} (\mbox{Q}), \]
\vspace{-4.5ex}\par\[
~~~~ \mbox{add\_mod} (\mbox{Q},\mbox{R},\mbox{U})
, \]
\vspace{-4.5ex}\par\[
~~~~ \mbox{add\_subject} (\mbox{R},\mbox{S})
, \]
\vspace{-4.5ex}\par\[
~~~~ \mbox{push\_to\_slash} (\mbox{S},\mbox{T})
, \]
\vspace{-4.5ex}\par\[
~~~~ \mbox{push\_to\_extra} (\mbox{T},\mbox{U})
, \]
\vspace{-4.5ex}\par\[
~~~~ \mbox{inflection} (\mbox{O},\mbox{P},\mbox{U})
. \]


\subsection{A note on the rules}

As is clear from the presentation above the grammar rules are a kind
of feature structure that have a special meaning, rather than in HPSG
where such rules are expressed as disjunctive principles on the way in
which complex feature structures may be composed. The following
motivations led to this design decision.

\begin{itemize}
\item This enables an ordinary interpreter/compiler for rules. Standard
parsing and generation algorithms can be applied. The current grammar
is interpreted by a head-corner parser and a head-driven bottom-up
generator \cite{kay-hd,cl,vannoord-diss,bouma-gertjan}.
\item We do not need a {\it dtrs} feature as in standard HPSG. This 
has two advantage. Firstly no (very) complex derivational histories
are built in the form of complex values for {\it dtrs}. Secondly we
have much less ambiguity. This second point probably needs some
further explanation.

If grammars are expressed in logical frameworks then we should not be
surprised to find that certain derivations (proofs that a sentence
is well-formed) give rise to the same instantiations. Although we may
regard this situation as a problem for processing, there is nothing
wrong from a linguistic viewpoint with the fact that certain sentences
can be shown to be well-formed in several ways. In such frameworks we
should not confuse `derivation' with `phrase structure'. However, if
we have a separate label of which the value might be instantiated by
these different derivations then, as far as the processor is
concerned, these ambiguities represent genuine ambiguities --- even
though the semantic structures might be identical.

\end{itemize}


\subsection{Example derivation}

For root sentences an appropriate top category for our Dutch grammar is
defines as follows:
\[
\avm{
 \mbox{\sc sign}  \\ 
 \mbox{\it sc} :[~] \\ 
 \mbox{\it cat} :\avm{
 \mbox{\sc vp}  \\ 
 \mbox{\it vform} : \mbox{\sc fin} } \\ 
 \mbox{\it inv} : \mbox{\sc +}  \\ 
 \mbox{\it slash} :[~] \\ 
 \mbox{\it extra} :[~] \\ 
 \mbox{\it rel} :[~] \\ 
 \mbox{\it f\_slash} :\langle \_\rangle }
\]

A sample derivation for the sentence:
\begin{exam}
jan kust vrouwen
\end{exam}
is given as follows. For reasons of space only some part of the information is
represented in the this tree diagram. The empty list value is generally suppressed,
as is the values of some other features. 

\begin{figure}[p]

\vspace{-30ex}

\par\input{fig0.tree}

\vspace{20ex}

\par\hspace{150pt}\input{fig1.tree}


\caption{\label{fig1}The phrase-structure tree and derivation tree for
the sentence `jan kust vrouwen'.}

\end{figure}


\nodeskip{150pt}
\outputquality{high}
\opentree{fig0.tree}
\tree{\noexpand\plcmdeq}
\leaf{\noexpand\plcmder}
\tree{\noexpand\plcmdes}
\leaf{\noexpand\plcmdet}
\leaf{\noexpand\plcmdeu}
\endtree
\endtree

\nodeskip{45pt}
\outputquality{low}
\opentree{fig1.tree}
\tree{\noexpand\plcmdev}
\leaf{\noexpand\plcmdew}
\tree{\noexpand\plcmdex}
\leaf{\noexpand\plcmdey}
\leaf{\noexpand\plcmdez}
\endtree
\endtree

\section{Basic Assumptions}

\subsection{Types and Features}

The HUG for Dutch uses a simplified type system. This type system is a
tree of types rooted by the type `top' (`unspecified'). The daughters
of the tree are subtypes. It is assumed that subtypes $s_1 \dots s_n$
of a type $t$ are mutually inconsistent. Furthermore, each type can
have an associated list of attributes, that are appropriate for this
type. No information concerning the possible values of these
attributes can be expressed. Finally note that an attribute can only
be associated with a single type. 

The current type-system is defined as a very flat system as follows.
Firstly the following subtypes of {\sc top} do not have any subtypes
or any attributes: 
{\sc part sbar n adv pp
prime(\_) + - phrasal fin inf te
pas om of dat left right extra topic nom obj obl mass sg pl neut nneut 
def indef fem masc nhuman}.  
Except for these, subtypes of {\sc top} are {\sc sign vp np adj sem rule lexical}.
Of these the following types have no subtypes, but do have appropriate
features:
\[ \mbox{\sc sign} : \mbox{\it lex sc cat inv subj sem slash extra rel f\_slash f\_extra f\_rel sep dir tree args} \]
\[  \mbox{\sc vp} : \mbox{\it ipp vform}\]
\[  \mbox{\sc np} : \mbox{\it case}\]
\[  \mbox{\sc lexical} : \mbox{\it stem word}\]

The subtype {\sc adj} is divided in subtypes {\sc att pred}.
Finally the following complex types are defined (figure~\ref{fig3}).

\begin{figure}[h,t,b,p]
\input{fig5.tree}\input{fig6.tree}
\caption{\label{fig3}Complex sub-types}
\end{figure}

\nodeskip{30pt}
\outputquality{high}
\opentree{fig5.tree}
\tree{\noexpand\sc sem : \noexpand\it  index}
\leaf{\noexpand\sc ix}
\tree{\noexpand\sc cx : \noexpand\it  fun restr}
\leaf{\noexpand\sc c0}
\tree{\noexpand\sc cxx : \noexpand\it  arg1}
\leaf{\noexpand\sc c1}
\tree{\noexpand\sc cxxx : \noexpand\it  arg2}
\leaf{\noexpand\sc c2}
\leaf{\noexpand\sc c3 : \noexpand\it  arg3}
\endtree
\endtree
\endtree
\endtree

\nodeskip{40pt}
\outputquality{high}
\opentree{fig6.tree}
\tree{\noexpand\sc rule : \noexpand\it  mt ls hd rs}
\leaf{\noexpand\sc apply}
\leaf{\noexpand\sc topicalize}
\leaf{\noexpand\sc extrapose}
\leaf{\noexpand\sc cliticize}
\leaf{\noexpand\sc verbfront}
\endtree
\subsection{Subjects}

For finite verbs there is a lexical rule that adds a subject as the
final element of the subcat list. This expresses the generalization
that only finite verbs take subjects. Note that all verbs have a
{\it subj} feature that we can refer to.

\subsection{Te}

The prefix `te' is not seen as a seperate word but is rather added to
an infinitive by the inflection rules. This simply gives rise to a
verb with the {\it vform} feature `te'.

\section{Verb clusters}
\comment{
 - inheritance of subcat list
 - vp-argument should be lexical
 - flat structure: better than categorial account
 - extraposition
 - third construction
 (- interaction with modification)
 - ipp
 - subj control
 - obj control
 - subj raising
 - aci
 - hebben/zijn \& participium inversion
 - modals \& modal inversion
 - passive ?
 }

The analysis of Dutch verb-clusters in HUG is based on proposals in
categorial grammar by Hoeksema and Moortgat and proposals in HPSG by
Hinrichs \& Nakasawa. The idea is that auxiliary verbs such as modals
select an unsaturated verb plus all the arguments that this verb
subcategorizes for.

\begin{exams}
\item \dots dat Jan wil slapen
\item * \dots dat Jan Marie wil slapen
\item \dots dat Jan Marie wil kussen
\item \dots dat Jan Marie boeken wil geven
\end{exams}

\[
 \mbox{verb\_stem} (\avm{
 \mbox{\sc sign}  \\ 
 \mbox{\it lex} :\avm{
 \mbox{\sc lexical}  \\ 
 \mbox{\it stem} : \mbox{wil} } \\ 
 \mbox{\it sc} :\langle \avm{
 \mbox{\sc sign}  \\ 
 \mbox{\it lex} : \mbox{\sc lexical}  \\ 
 \mbox{\it sc} :\mbox{A} \\ 
 \mbox{\it cat} :\avm{
 \mbox{\sc vp}  \\ 
 \mbox{\it vform} : \mbox{\sc inf} } \\ 
 \mbox{\it dir} : \mbox{\sc topic;right} } | \mbox{A}\rangle  \\ 
 \mbox{\it cat} :\avm{
 \mbox{\sc vp}  \\ 
 \mbox{\it ipp} : \mbox{\sc +} }})\]

This entry indicates that `wil' subcategorizes for a list of arguments
consisting of a lexical and infinitival verb-phrase and the list of
arguments that have not yet been selected by that verb-phrase,
indicated by the sharing of the variable A.

The control relation between the subject of the auxiliary verb and the
subject of the embedded verb is indicated by sharing the value of the
{\it index} attribute of the semantics.

Note that if there are more than two verbs the process applies
recursively. For example, in the subordinate sentence

\begin{exam}
\dots Jan Marie wil kunnen kussen
\end{exam}

we obtain the phrase structure given in figure~\ref{fig2}.
\begin{figure}[p]

 \[
 \avm{
  \mbox{\sc sign}  \\ 
  \mbox{\it lex} :\avm{
  \mbox{\sc lexical}  \\ 
  \mbox{\it stem} : \mbox{wil}  \\ 
  \mbox{\it word} : \mbox{wil} } \\ 
  \mbox{\it sc} :\langle \avm{
  \mbox{\sc sign}  \\ 
  \mbox{\it lex} :\avm{
  \mbox{\sc lexical}  \\ 
  \mbox{\it stem} : \mbox{kan}  \\ 
  \mbox{\it word} : \mbox{kunnen} } \\ 
  \mbox{\it sc} :\langle \avm[{\mbox{I}}]{ \mbox{\sc sign}  \\ 
  \mbox{\it lex} :\avm{
  \mbox{\sc lexical}  \\ 
  \mbox{\it stem} : \mbox{kus}  \\ 
  \mbox{\it word} : \mbox{kussen} } \\ 
  \mbox{\it sc} :\mbox{K}~\langle \avm[{\mbox{L}}]{ \mbox{\sc sign}  \\ 
  \mbox{\it lex} :\avm{
  \mbox{\sc lexical}  \\ 
  \mbox{\it stem} : \mbox{marie}  \\ 
  \mbox{\it word} : \mbox{marie} } \\ 
  \mbox{\it cat} :\avm{
  \mbox{\sc np}  \\ 
  \mbox{\it case} : \mbox{\sc obj} }}\rangle  \\ 
  \mbox{\it cat} :\avm{
  \mbox{\sc vp}  \\ 
  \mbox{\it ipp} : \mbox{\sc -}  \\ 
  \mbox{\it vform} : \mbox{\sc inf} }} | \mbox{K}\rangle  \\ 
  \mbox{\it cat} :\avm{
  \mbox{\sc vp}  \\ 
  \mbox{\it ipp} : \mbox{\sc +}  \\ 
  \mbox{\it vform} : \mbox{\sc inf} }} , \mbox{I} , \mbox{L} , \avm{
  \mbox{\sc sign}  \\ 
  \mbox{\it lex} :\avm{
  \mbox{\sc lexical}  \\ 
  \mbox{\it stem} : \mbox{jan}  \\ 
  \mbox{\it word} : \mbox{jan} } \\ 
  \mbox{\it cat} :\avm{
  \mbox{\sc np}  \\ 
  \mbox{\it case} : \mbox{\sc nom} }}\rangle  \\ 
  \mbox{\it cat} :\avm{
  \mbox{\sc vp}  \\ 
  \mbox{\it ipp} : \mbox{\sc +}  \\ 
  \mbox{\it vform} : \mbox{\sc fin} } \\ 
  \mbox{\it inv} : \mbox{\sc -} }\]

 \input{fig3.tree}

\caption{\label{fig2} Derivation tree
for `Jan Marie wil kunnen kussen', and the eventual instantiation of 
the category for `wil'.}
\end{figure}

This `flat' structure of the verb-cluster should be compared with
proposals in categorial grammar. In a CG account the verb-cluster
will be associated with a binary structure. However, it is then quite
complicated to define the appropriate feature percolation of the
`lexicality' feature: this feature should allow categories from the
lexicon, but also combinations of verbs. On the other hand, verbs that
have selected e.g. a noun-phrase argument should be marked `phrasal',
because:

\begin{exams}
\item * \dots dat Jan wil Marie kussen
\item * \dots dat Jan Marie wil boeken geven
\item * \dots dat Jan wil Marie boeken geven
\end{exams}

\nodeskip{40pt}
\outputquality{high}
\opentree{fig3.tree}
\tree{\noexpand\plcmdfa}
\leaf{\noexpand\plcmdfb}
\leaf{\noexpand\plcmdfc}
\leaf{\noexpand\plcmdfd}
\leaf{\noexpand\plcmdfe}
\leaf{\noexpand\plcmdff}
\endtree

Verb-clustering exhibits a feature known as the IPP effect: rather
than the participle form raising verbs that are selected by the
auxiliary `hebben' or `zijn' take the infinitival form:

\begin{exams}
\item \dots dat Jan heeft geslapen 
\item \dots dat Jan Marie heeft willen kussen
\item * \dots dat Jan Marie heeft gewild kussen
\end{exams}


For this purpose the boolean verb phrase feature {\it ipp} is used.
Verbs that select participles are systematically ambiguous between a
version that selects a participle if their verbal argument is {\it
-ipp}, and an infinitive if their verbal argument is {\it +ipp}.
Thus, the following two specifications for the verbal stem `heb' are
defined:

\[
 \mbox{verb\_stem} (\avm{
 \mbox{\sc sign}  \\ 
 \mbox{\it lex} :\avm{
 \mbox{\sc lexical}  \\ 
 \mbox{\it stem} : \mbox{heb} } \\ 
 \mbox{\it sc} :\langle \avm{
 \mbox{\sc sign}  \\ 
 \mbox{\it lex} : \mbox{\sc lexical}  \\ 
 \mbox{\it sc} :\mbox{A} \\ 
 \mbox{\it cat} :\avm{
 \mbox{\sc vp}  \\ 
 \mbox{\it ipp} : \mbox{\sc -}  \\ 
 \mbox{\it vform} : \mbox{\sc pas} } \\ 
 \mbox{\it dir} : \mbox{\sc right;left} } | \mbox{A}\rangle  \\ 
 \mbox{\it cat} :\avm{
 \mbox{\sc vp}  \\ 
 \mbox{\it ipp} : \mbox{\sc +} }})\]
\vspace{-4.5ex}\par\[
 \mbox{verb\_stem} (\avm{
 \mbox{\sc sign}  \\ 
 \mbox{\it lex} :\avm{
 \mbox{\sc lexical}  \\ 
 \mbox{\it stem} : \mbox{heb} } \\ 
 \mbox{\it sc} :\langle \avm{
 \mbox{\sc sign}  \\ 
 \mbox{\it lex} : \mbox{\sc lexical}  \\ 
 \mbox{\it sc} :\mbox{A} \\ 
 \mbox{\it cat} :\avm{
 \mbox{\sc vp}  \\ 
 \mbox{\it ipp} : \mbox{\sc +}  \\ 
 \mbox{\it vform} : \mbox{\sc inf} } \\ 
 \mbox{\it dir} : \mbox{\sc right} } | \mbox{A}\rangle  \\ 
 \mbox{\it cat} :\avm{
 \mbox{\sc vp}  \\ 
 \mbox{\it ipp} : \mbox{\sc +} }})\]

Note also that these two entries differ with respect to the
directionality of their verbal argument. Although raising verbs
normally select their verbal argument to the right, it is possible for
participles to show up at the beginning of the verb cluster:

\begin{exams}
\item \dots Jan Marie heeft willen hebben gekust
\item \dots Jan Marie gekust heeft willen hebben
\item *\dots Jan Marie heeft willen gekust hebben
\end{exams}

Note that because of the way verbal arguments are inherited this
analysis suffices to obtain the right orders. Eventually the
participle is inherited by the most prominent verb (the leftmost one) and then
this verb indeed selects the participle to the left --- as desired.

Another word-order phenomenon in the verb-cluster is the possibility
of modal inverstion. If the verb-cluster has only two verbs, one of
which is a modal verb, then the two verbs can come in both orders:

\begin{exams}
\item \dots Jan wil slapen
\item \dots Jan slapen wil
\item \dots Jan Marie wil kussen
\item \dots Jan Marie kussen wil
\item *\dots Jan heeft slapen willen
\item *\dots Jan hebben geslapen wil
\end{exams}

Therefore, modals are systematically ambiguous between the
feature-structure given above, and the following:

\[
 \mbox{verb\_stem} (\avm{
 \mbox{\sc sign}  \\ 
 \mbox{\it lex} :\avm{
 \mbox{\sc lexical}  \\ 
 \mbox{\it stem} : \mbox{wil} } \\ 
 \mbox{\it sc} :\langle \avm{
 \mbox{\sc sign}  \\ 
 \mbox{\it lex} : \mbox{\sc lexical}  \\ 
 \mbox{\it sc} :\mbox{A} \\ 
 \mbox{\it cat} :\avm{
 \mbox{\sc vp}  \\ 
 \mbox{\it ipp} : \mbox{\sc -}  \\ 
 \mbox{\it vform} : \mbox{\sc inf} } \\ 
 \mbox{\it dir} : \mbox{\sc left} } | \mbox{A}\rangle  \\ 
 \mbox{\it cat} :\avm{
 \mbox{\sc vp}  \\ 
 \mbox{\it ipp} : \mbox{\sc +}  \\ 
 \mbox{\it vform} : \mbox{\sc fin} } \\ 
 \mbox{\it inv} : \mbox{\sc -} })\]




\section{Modification}
Motivation: 
- narrow scope readings of verb-clusters.
- cf. Miller
- other syntactic generalizations (e.g. remnants of gapping).

\section{Topicalization}
- push subcat element to slash
- topic rule
- Partial vp-topicalization.
- allows for topicalization of modifiers that are on subcat.

\section{Extraposition}
- push subcat elements to extraposition list
- vp -> vp extras
- third construction

\section{Verb-second}
- Rule.

\section{Particle Verbs}
- Verb has a clitic
- v -> clitic v 
- Interaction with verb-clusters. 
  - niet: * dat jan piet gebeld op heeft
- Interaction with topicalization
  - partiele topicalization


\section{Relativization}
Relatives. Agreement. Case.

\bibliography{biblio}




\end{document}









