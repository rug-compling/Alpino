\documentstyle[gj_bib_art,a4wide,leqno,fleqn,prolog,matrix]{article}
\begin{document}
\bibliographystyle{named}

\title{The Dutch verbal complex: A Lexicalist Account}
\author{Gosse Bouma}

\maketitle

\tableofcontents

\section{Introduction}

This paper documents an implementation of a categorial analysis of the Dutch verbal complex. The analysis is formulated in terms of  a unification-based categorial grammar in which extensive use is made of lexical rules. The starting point of the analysis is the proposal of \shortcite{hoeksema-ms}. In section 3, we argue that a categorial account of `cross-serial dependencies' based on some form of division, must include both the disharmonic and harmonic versions of this rule. In section 4, we show how the overgeneration (both in terms of word order and in terms of spurious derivations) that results form including division  can be avoided. Several minor issues are briefly mentioned in section 5. In section 6, we show how the analysis of (partial) {\sc vp}-fronting proposed by \shortcite{Nerbonne93} can be incorporated easily. We also note that the  grammaticality judgements at this point are not very clear, and that some issues are left unexplained. Finally, we argue that both word order and scope of adjuncts in sentences headed by a verbal complex suggests that these must be treated as arguments. 

\section{Verb raising, Extraposition, and Partial Extraposition}

In Dutch, verbs selecting a {\sc vp}-complement come in at least three 
varieties. First of all, a verb, such as {\em weigeren}, may simply 
subcategorize for a full {\sc vp} to its right (\ref{extrap}). In most 
transformational accounts, it is assumed that the {\sc vp}-complement originates 
in the midfield, and is (obligatorily) extraposed to the right. This is why the 
verbs in this class are called {\em extraposition} verbs. 
\begin{equation}
\label{extrap}
\mbox{\ldots dat Jan Marie verbiedt [het boek te lezen]} 
\end{equation}

Second, a verb may form a `verbal complex' or `verb cluster' with the head of 
its
{\sc vp}-complement.  In this case, the non-verbal elements of the {\sc
vp}-complement normally occur to the left of the governing verb, whereas the
verbal head of the {\sc vp}-complement occurs to the right (\ref{vr}a).  This is
the so-called {\em verb-raising} (VR) construction, as transformational accounts
have assumed that in this case the head of the embedded {\sc vp}-complement is
raised from its original position left of the governing verb to a position on 
the
right.  Modals, such as {\em willen} or {\em kunnen} are typical VR-verbs, as 
well
as perception verbs such as {\em zien} or {\em horen} and causatives such as 
{\em
laten}.  The VR-construction gives rise to so-called {\em cross-serial
dependencies}, as the non-verbal arguments of a complex verb-cluster occur in an
order which is identical to the order of the verbs selecting for these arguments
(\ref{vr}b,c).


\begin{equation}
\label{vr}
\begin{array}[t]{ll}
a. & \mbox{\ldots dat Jan het boek wil lezen} \\
b.& \mbox{\ldots dat Jan Marie$_1$ het boek$_2$ laat$_1$ lezen$_2$} \\
c. & \mbox{\ldots dat Jan$_1$ Marie$_2$ het boek$_3$ wil$_1$ laten$_2$ 
lezen$_3$}
\end{array}
\end{equation}

Finally, there is a small class of verbs (containing {\em proberen} and {\em 
verzuimen}) that induce word orders that are a mixture of extraposition and 
verb-raising. Instances of such mixed word orders (\ref{p_extrap}) are known as 
`the 
third construction' \cite{denBestenRutten89}, or as cases of  
{\em partial} extraposition \cite{hoeksema-ms}. 

\begin{equation}
\label{p_extrap}
\begin{array}[t]{ll}
a. & \mbox{\ldots dat Jan het boek aan Marie verzuimt te geven} \\
b.& \mbox{\ldots dat Jan het boek verzuimt aan Marie  te geven} \\
c. & \mbox{\ldots dat Jan verzuimt het boek aan Marie te geven} \\
\end{array}
\end{equation}


\subsection{A Categorial Analysis}

In \shortcite{hoeksema-ms}  a categorial analysis of verb raising and 
related constructions is presented. 

Extraposition verbs are simply assigned a category which selects for a {\sc vp} 
complement on its right, thus eliminating the  need for a special  {\em 
extraposition}-operation:

\begin{equation}
\begin{array}[t]{ll}
\mbox{verbieden :} & (\mbox{\sc np}\backslash\mbox{\sc vp})/ \mbox{\sc vp}
\end{array}
\end{equation}


Verb raising verbs are assigned schematic (polymorphic) categories of the 
following 
type:
 
\begin{equation}
\begin{array}[t]{ll}
\mbox{willen :} & (\$\backslash \mbox{\sc vp})/(\$\backslash \mbox{\sc vp} ) \\ 
\mbox{laten :} & (\$\backslash (\mbox{\sc np}\backslash\mbox{\sc vp}))/
						(\$\backslash \mbox{\sc vp} )
\end{array}
\end{equation}

The $\$\backslash$-symbol in these category-schemata represents a variable which 
can be instantiated by an arbitrary number of arguments, all of which must be 
dominated by a $\backslash$. Thus, the category of {\em willen} subsumes
{\sc vp$/$vp , (np$\backslash$vp)$/$(np$\backslash $vp)}, and {\sc 
(np$\backslash$(np$\backslash$vp))$/$(np$\backslash$(np$\backslash $vp))}.
The following are derivations of (\ref{vr}b,c).

{\sc
\begin{equation}
\mbox{
\begin{tabular}[t]{ccccccc}
\ldots {\em Jan} & {\em Marie} & {\em het boek} & {\em laat} & {\em lezen} \\
\ldots np & np & np & $(\$\backslash$tv$)/(\$\backslash$vp$)$ & np$\backslash$vp  
\\
\cline{4-5}
&&& \multicolumn{2}{c}{np$\backslash$tv} \\
\cline{3-4} && tv \\
\cline{2-3} & \multicolumn{2}{c}{vp }\\
\cline{1-2} \multicolumn{2}{c}{s}
\end{tabular}
}
\end{equation}

\begin{equation}
\mbox{
\begin{tabular}[t]{cccc@{\hspace{2cm}}c}
\ldots {\em Jan} & {\em Marie} & {\em het boek} & {\em wil} & {\em laten lezen}  
\\
\cline{5-5}
\ldots np & np & np & $(\$\backslash$vp)$/(\$\backslash$vp) &
np$\backslash$np$\backslash$vp \\
\cline{4-5}
&&& \multicolumn{2}{c}{np$\backslash$np$\backslash$vp} \\
\cline{3-4} && np$\backslash$vp \\
\cline{2-3} & \multicolumn{2}{c}{vp }\\
\cline{1-2} \multicolumn{2}{c}{s}
\end{tabular}
}
\end{equation}
} 

Partial extraposition verbs, finally, are treated as VR verbs, that is 
they also receive a polymorphic category (\ref{pe-le}).  The difference between 
the 
two verb types is
that for VR verbs, it is obligatory that only the verbal head of its complement
follows the VR verb, whereas for PE verbs this constraint does not hold.  We 
will
explain below how this difference is accounted for.

\begin{equation}
\label{pe-le}
\begin{array}[t]{ll}
\mbox{verzuimen :} & (\$\backslash \mbox{\sc vp})/(\$\backslash \mbox{\sc vp} ) 
\\ 
\mbox{proberen :} & (\$\backslash\mbox{\sc vp})/(\$\backslash \mbox{\sc vp} )
\end{array}
\end{equation}

The analysis of \shortcite{hoeksema-ms} differs from most categorial analyses in
that it treats the possibility of verb raising (and the related cross-serial 
word
orders) as a consequence of the category of certain lexical items.  Other
treatmentss have accounted for cross-serial word order either by a combination 
of
a non-directional flexible CG with word order constraints
\cite{Houtman84,Steedman85} or by introducing disharmonic rules of composition
\cite{Cremers-diss}.  The advantage of using lexically assigned category 
schemata
is that it encapsulates the machinery required to derive cross-serial word 
orders
fully into the lexicon.  As a consequence, the interaction of these rules with
other (syntactic) rules is controlled more easily and overgeneration can be
avoided.

\section{Division}

\subsection{Implementation}
We have implemented an analysis of Dutch verb clusters in which verb-raising 
and partial extraposition verbs may be subject to a lexical rule of `division'.
We think of such lexical rules as recursive, relational, constraints on  
categories. 

The variable notation used by Hoeksema, and introduced originally in the work of
Steedman, cannot be translated directly into a unification-based setting, as 
there
is no corresponding feature structure for the class of categories which is
supposed to be subsumed by a \$-category.  That is, we cannot define a feature
structure, such that all feature structures corresponding to legal 
instantiations
of the \$-category are subsumed by it.\footnote{ An encoding of categories using
{\sc subcat}-lists, as has been proposed in \shortcite{Wittenburg-diss}, will 
not
solve the problem either.}  Therefore, we have chosen for a slightly different
approach.

The effect of the category schemata Hoeksema assumes is similar to that of the 
categorial rule {\em rightward disharmonic division}:

\begin{equation}
X/Y \rightarrow (Z\backslash X)/(Z\backslash Y) \mbox{\hspace{1cm} \bf Rightward 
Disharmonic Division}
\end{equation}
That is, all categories subsumed by the category schema assigned to {\em willen} 
could be derived by applying the rule disharmonic composition zero or more times 
to the category {\sc vp$/$vp}. 

Instead of using polymorphic categories, we therefore assume that VR and PE 
verbs
are subject to a lexical rule of division.  An illustrative lexical entry, as 
well
as a first approximation of the definition of {\em division} is presented below.

\pr
\pred
\head{\rel{lex}(\con{willen},\var{Sign}) \; \prologif }
\body{\var{Sign0} \Leftrightarrow \con{vp}/\con{vp},}
\body{\rel{division}(\var{Sign0},\var{Sign}).}
\epred
\pred
\head{\rel{division}(\var{In},\var{In}).}
\head{\rel{division}(\var{In},\var{Out}) \; \prologif}
\body{\var{Out} \Leftrightarrow 
(\var{Z}\backslash\var{X})/(\var{Z}\backslash\var{Y}),}
\body{\rel{division}(\var{In},\var{X}/\var{Y}).}
\epred
\epr
The {\em division}-relation succeeds if its second argument is the result of 
applying the categorial rule (disharmonic) division to the first argument 
an arbitrary number of times. 

A consequence of introducing recursive lexical rules such as division is that 
certain lexical entries will be infinitely ambiguous. This is not without 
consequences 
for the generative capacity of the grammar \cite{carpenter-lexical}. We hope to 
report 
on the possiblities of parsing with grammars containing recursive lexical rules 
elsewhere.

\subsection{Other cases of division}

The definition of division presented in the previous section only allows for
disharmonic division on right-directional functors.  However, once we take some
additional data into account, it becomes clear that other instances of division are
also needed.

The auxiliaries {\em hebben} en {\em zijn} are VR verbs which select a {\sc 
vp}-complement headed by a participle:

\begin{equation}
\begin{array}[t]{ll}
a. & \ldots\mbox{dat Jan Marie heeft gekust.} \\
b. & \ldots\mbox{dat Jan te laat is gekomen}.
\end{array}
\end{equation}
 
These constructions are somewhat exceptional, however, in that a word order in 
which the participle precedes the auxiliary is also possible:

\begin{equation}
\begin{array}[t]{ll}
a. & \ldots\mbox{dat Jan Marie gekust heeft.} \\
b. & \ldots\mbox{dat Jan te laat gekomen is}.
\end{array}
\end{equation}
 
One might try to account for these data by assigning auxiliaries not only the
category of VR verbs, but also the category {\sc vp$\backslash $vp}.  This will not
quite suffice, however.  Consider for instance the combination of an auxiliary with
an extraposition verb.  Again, the participle may either precede or follow the
auxiliary, but the extraposed complement must always follow the rightmost element in
the sentence.

\begin{equation}
\label{hebben}
\begin{array}[t]{ll}
a. & \ldots\mbox{dat Jan heeft [$_{\mbox{\sc vp}}$ geleerd zich aan te passen]}. \\
b. & \ldots\mbox{dat Jan [ geleerd heeft  ] zich aan te passen}.
\end{array}
\end{equation}

\noindent The most straightforward analysis of (\ref{hebben}b) is to assume that
{\em hebben}, when selecting a complement to its left also behaves as a VR verb,
that is, as a verb that may inherit the arguments of the head of its complement.
Given this assumption, {\em heeft} in (\ref{hebben}b) first combines with the
participle {\em geleerd} and next combines with the extraposed phrase {\em het
contract te tekenen}.  Note, however, that this requires a left-directional version
of disharmonic division:

\begin{equation}
Y\backslash X \rightarrow (Y/Z)\backslash(X/Z) \mbox{\bf {\hspace{2cm} Leftward 
Disharmonic Division}}
\end{equation}
The derivation of (\ref{hebben}b) is sketched below:

{\sc 
\begin{equation}
\label{gew-heeft-deriv}
\mbox{
\begin{tabular}[t]{cccc}
\ldots {\em Jan}&{\em geleerd} & {\em heeft} & {\em zich aan te passen} 
\\
\ldots np & vp$/$vp & vp$\backslash$vp& vp \\
&& $\Downarrow$ \\
&& (vp$/$vp)$\backslash$(vp$/$vp) \\
\cline{2-3}
& \multicolumn{2}{c}{vp$/$vp} \\
\cline{2-4} && \multicolumn{2}{c}{vp} 
\end{tabular}
}
\end{equation}
}

If an auxiliary is selected by a VR verb, the participle governed by the auxiliary
may appear either in final position, or it may precede both the VR verb and the
auxiliary:

\begin{equation}
\label{geprotesteerd}
\begin{array}[t]{ll}
a. & \ldots\mbox{dat Jan zou hebben geprotesteerd}. \\
b. & \ldots\mbox{dat Jan geprotesteerd zou hebben}.
\end{array}
\end{equation}
The derivation of (\ref{geprotesteerd}b) is as follows:

{\sc 
\begin{equation}
\mbox{
\begin{tabular}[t]{cccc}
\ldots {\em Jan}&{\em geprotesteerd} & {\em zou} & {\em hebben} \\
\ldots np & vp & vp$/$vp & vp$\backslash$vp \\
&& $\Downarrow$ \\
&& (vp$\backslash$vp)$/$(vp$\backslash$vp) \\
\cline{3-4}
&& \multicolumn{2}{c}{vp$\backslash$vp} \\
\cline{2-4} & \multicolumn{3}{c}{vp} 
\end{tabular}
}
\end{equation}
}

\noindent Next, consider a case in which the participle is actually an extraposition
verb.  Again, two word orders are possible:

\begin{equation}
\label{gew-zou-hebben}
\begin{array}[t]{ll}
a. & \ldots\mbox{dat Jan moet hebben [$_{\mbox{\sc vp}}$ geleerd zich aan te passen]}. \\
b. & \ldots\mbox{dat Jan [ geleerd  [ moet hebben ]] zich aan te passen}.
\end{array}
\end{equation}

\noindent The most likely analysis of (\ref{gew-zou-hebben}b) is that {\em hebben}
is assigned the category {\sc ((vp$/$vp)$\backslash$(vp$/$vp))} (as in
(\ref{gew-heeft-deriv})).  Next, {\em moet} has to combine with this complex
category.  This can be achieved, but implies that {\em harmonic} division of VR
verbs (\ref{hd}) must also be possible.  The stepwise derivation of the category for
{\em moet} required in this case is shown in (\ref{zou}).  It consists of one
instance of disharmonic division followed by an instance of harmonic division.

\begin{equation}
\label{hd}
X/Y \rightarrow (X/Z)/(Y/Z) \mbox{\bf \hspace{2cm} Rightward Harmonic Division}
\end{equation}

\begin{equation}
\label{zou}
\mbox{\sc
vp$/$vp $\rightarrow$ (vp$\backslash$vp)$/$(vp$\backslash$vp) $\rightarrow$
((vp$\backslash$vp)$/$(vp$\backslash$vp))$/
			$((vp$\backslash$vp)$/$(vp$\backslash$vp))
}
\end{equation}

Finally, consider cases in which an auxiliary selects a participle which is  
extrapostion verb also selecting an {\sc np}-object (control-verbs, such as {\em 
beloven} or {\em verbieden} are typical examples of the latter category). 

\begin{equation}
\label{beloven}
\begin{array}[t]{ll}
a. & \ldots\mbox{dat Jan Marie heeft beloofd de diskette mee te brengen}. \\
b. & \ldots\mbox{dat Jan Marie [ beloofd heeft] de diskette mee te brengen}.
\end{array}
\end{equation}

\noindent The category of {\em beloven} is {\sc (np$\backslash$vp)$/$vp}.  This
implies that {\em heeft} in (\ref{beloven}b) can only combine with the preceding
participle if it can undergo leftward harmonic division (\ref{hld}), as well as
disharmonic division.

\begin{equation}
\label{hld}
Y\backslash X \rightarrow (Z\backslash Y)\backslash (Z\backslash X) \mbox{\bf 
\hspace{2cm} Leftward Harmonic Division}
\end{equation}

\noindent The derivation of (\ref{beloven}b) is shown below.

\begin{equation}
\mbox{\sc
\begin{tabular}[t]{ccccc}
{\em J.}& {\em M.} & {\em beloofd}  & {\em heeft} & {\em de diskette 
mee 
te brengen}\\
\ldots np & np & (np$\backslash$vp)$/$vp & vp$\backslash$vp & vp \\
&&& $\Downarrow$ \\
&&& (np$\backslash$vp)$\backslash$(np$\backslash$vp) \\
&&& $\Downarrow$ \\
&&& ((np$\backslash$vp)$/$vp)$\backslash$((np$\backslash$vp)$/$vp) \\
\cline{3-4}
&& \multicolumn{2}{c}{(np$\backslash$vp)$/$vp} \\
\cline{3-5} && \multicolumn{3}{c}{np$\backslash$vp} 
\end{tabular}
}
\end{equation}

The upshot of the discussion in this section is that all four possibilities of
division are needed to account for the full range of word order facts associated
with `verb raising'.  While at first blush it may seem that VR requires only
rightward disharmonic division, participle inversion requires that leftward
disharmonic division is also available.  Combinations of VR verbs with 
auxiliaries
and extraposition verbs show that the harmonic instances of division are also
required.  Therefore, in the sequel we will assume that VR verbs are subject to 
a
lexical rule of division, which comes in the following four varieties:

\begin{equation}
\begin{array}[t]{ll}
\mbox{ Right Disharmonic} &X/Y \rightarrow (Z\backslash X)/(Z\backslash Y) \\
\mbox{ Left Disharmonic} & Y\backslash X \rightarrow (Y/Z)\backslash(X/Z) \\
\mbox{ Right Harmonic} & X/Y \rightarrow (X/Z)/(Y/Z) \\
\mbox{ Left Harmonic} & 
Y\backslash X \rightarrow (Z\backslash Y)\backslash (Z\backslash X) 
\end{array}
\end{equation}

The definition of {\em division} as a relational constraint is modified 
accordingly:

\pr
\pred
\head{\rel{division}(\var{In},\var{In}).}
\head{\rel{division}(\var{In},\var{Out}) \; \prologif 
	\mbox{\hspace{2cm} \% rightward disharmonic}}
\body{\var{Out} \Leftrightarrow 
(\var{Z}\backslash\var{X})/(\var{Z}\backslash\var{Y}),}
\body{\rel{division}(\var{In},\var{X}/ \var{Y}).}
\head{\rel{division}(\var{In},\var{Out}) \; \prologif
	\mbox{\hspace{2cm} \% leftward disharmonic}}
\body{\var{Out} \Leftrightarrow 
(\var{Y}/ \var{Z})\backslash(\var{X}/ \var{Z}),}
\body{\rel{division}(\var{In},\var{Y}/ \var{X}).}
\head{\rel{division}(\var{In},\var{Out}) \; \prologif
	\mbox{\hspace{2cm} \% rightward harmonic}}
\body{\var{Out} \Leftrightarrow 
(\var{X}/ \var{Z})/(\var{Y}/ \var{Z}),}
\body{\rel{division}(\var{In},\var{X}/ \var{Y}).}
\head{\rel{division}(\var{In},\var{Out}) \; \prologif
	\mbox{\hspace{2cm} \% leftward harmonic}}
\body{\var{Out} \Leftrightarrow 
(\var{Z}\backslash\var{Y})\backslash(\var{Z}\backslash\var{X}),}
\body{\rel{division}(\var{In},\var{Y}\backslash\var{X}).}
\epred
\epr

\section{Phrasal and subphrasal constituents}

In this section, we discuss the distribution of the feature {\sc phrase} (and 
its
companion {\sc rphrase}).  This feature is used to account for the difference
between VR and PE verbs, and also to eliminate the spurious ambiguity that 
results
form freely applying harmonic division.

\subsection{Avoiding overgeneration}

The analysis of VR as we have discussed it so far overgenerates in several ways.
The grammar makes it possible for both VR and PE verbs to combine
with the verbal head of their complement, before combining with the (non-verbal)
arguments of that head, but nothing in the analysis requires that this actually
must be so.  For PE verbs this is fine, as in these cases both the VR and the
extraposed word order is allowed, as well as cases that are a mixture of both 
((\ref{cremers}c) is taken from \shortcite{Cremers-diss}, the `partially 
extraposed' {\sc 
vp} appears is italic):

\begin{equation}
\label{cremers}
\begin{array}[t]{ll@{\hspace{0cm}}l}
a. && \ldots\mbox{dat hij {\em de chef voor zich} meende {\em te hebben 
gewonnen}}. \\
b. && \ldots\mbox{dat hij meende {\em de chef voor zich te hebben gewonnen}}. \\
c. && \ldots\mbox{dat hij  {\em de chef} meende {\em voor zich te hebben 
gewonnen}}. \\
\end{array}
\end{equation}

\noindent For VR verbs, however, only the VR word order is grammatical, and 
thus, 
the
following sentences need to be ruled out:

\begin{equation}
\label{ungramm}
\begin{array}[t]{ll@{\hspace{0cm}}l}
a. &^*& \ldots\mbox{dat Jan moet [een artikel lezen]}. \\
b. &^*& \ldots\mbox{dat Jan wil [aan zijn proefschrift werken]}. \\
c. &^*& \ldots\mbox{dat Jan [geleerd [zich aan te passen]] heeft}.
\end{array}
\end{equation}


\noindent In (\ref{ungramm}a,b), an {\sc np} or {\sc pp} argument occurs to the
right of a VR verb, leading to ungrammaticality.  In (\ref{ungramm}c), an extraposed
{\sc vp}-complement has not been `extraposed' fully, also leading to
ungrammaticality.  The derivation of these phrases is straightforward, however, and
involves no appeal to division whatsoever.


The observation we have failed to incorporate is that when a verb is a VR
trigger, the process of verb raising is in fact obligatory.  That is, a VR verb
must (and not only may) combine with the verbal head of its complement, before
this head combines with its own (non-verbal) complements.  In
\shortcite{hoeksema-ms} this is achieved by imposing the constraint that the
complement of a VR verb must be lexical (i.e.  {\sc [+lex]}).  We follow this
suggestion here, but, for reasons that will become apparent below, we have 
chosen
to introduce a feature {\sc phrase} instead of {\sc lex}.  Thus, we impose the
constraint that the verbal complement of a VR verb must be {\sc -phrase}.  This
is achieved as follows:

\pr
\pred
\head{\rel{lex}(\con{willen},\var{Sign}) \; \prologif }
\body{\var{Sign0} \Leftrightarrow \con{iv}/\con{vp},}
\body{\rel{division}(\var{Sign0},\var{Sign}),}
\body{\var{Sign0}\con{:arg:phrase} \Leftrightarrow \con{'-'}}. 
\head{\rel{lex}(\con{hebben},\var{Sign}) \; \prologif }
\body{\var{Sign0} \Leftrightarrow \con{vp}\backslash \con{vp},}
\body{\rel{division}(\var{Sign0},\var{Sign}),}
\body{\var{Sign0}\con{:arg:phrase} \Leftrightarrow \con{'-'}}.
\epred
\epr

\noindent Note that the relevant constraint is added to the argument of the
`output' of {\em division} and not to the argumentof the `input'.  On the
assumption that combinations of a verb and one or more of its {\sc pp} or {\sc
np} arguments as well as combinations of an extraposition verb and its {\sc
vp}-complement are marked {\sc [+phrase]}, this will rule out the ungrammatical
cases in (\ref{ungramm}).

The next, and more difficult, matter that needs to be settled is when exactly a
constituent is marked as {\sc [+phrase]}.  Below, we discuss an approach in 
which each functor specifies whether the result of combining this functor with 
its argument will be phrasal or not. We believe that this solution makes it 
reasonably straightforward to account for a wide range of facts. Some of 
alternative approaches are discussed in the section \ref{alternatives}.  

\subsubsection{Basic verb clusters}

The combination of a functor and an argument usually is {\sc +phrase}, but
there are some exceptions.  First of all, the combination of a verb with the
verbal head of its complement is not {\sc +phrase}:

\begin{equation}
\label{rb}
\ldots\mbox{(dat Jan het manuscript) zou [willen lezen]}. \\
\end{equation}

\noindent On the assumption that (\ref{rb}) has an analysis in which {\em
willen} combines with {\em lezen}, before it acts as argument of {\em zou}, it
must be the case that {\em willen lezen} is not {\sc +phrase}.\footnote{
\label{against-lb} It is difficult to find evidence for or against a particular
phrase structure inside verb clusters.  Conjunction facts show that pretty much
any structure is possible.  The fact that certain parts of the cluster can be
topicalized as a constituent is also not very solid evidence, as we will show
below.  One consideration that might be an argument in favor of 
right-bracnhing  is the fact that {\em willen} on its own cannot be 
topicalized. This follows if
we assume a right-branching structure.  Second, there may be a cross-linguistic
argument for the structure in (\ref{rb}).  In German, the phenomenon known as
{\em auxiliary-flip} orders (finite) auxiliaries in front of the verb cluster,
whereas they normally occur at the end of the cluster.  This is easily accounted
for if we assume that {\em haben} takes a verb cluster instead of a lexical unit
as argument.  Hinrichs and Nakazawa (\citeyear{HinrichsNakazawa93}) have
presented an analysis in terms of HPSG that is, in all relevant aspects,
compatible with the present categorial account.}

Second, combinations of a verb-prefix and a verb or verb cluster are not {\sc 
+phrase}:

\begin{equation}
\label{sep-prefix}
\begin{array}[t]{ll@{\hspace{0cm}}l}
a. && \ldots\mbox{dat Jan Marie heeft aangesproken}. \\
b. && \ldots\mbox{dat Jan Marie heeft [ aan durven spreken ]}. \\
\end{array}
\end{equation}

\noindent For (\ref{sep-prefix}a) it might be suggested that {\em aangesproken}
is a word, as far as syntax is concerned, but this cannot be the whole story,
given examples such as (\ref{sep-prefix}b), where the prefix has been seperated
from the verb to which it belongs.  If {\em heeft} as a VR verb takes {\em aan
durven spreken} as argument, the combination of {\em aan} and the verb cluster
{\em durven spreken} cannot be {\sc +phrase}.

To account for the rule (complex phrases are {\sc +phrase}) as well as the
exceptions\footnote{
Apart from verb clusters and seperable prefixes, there are a number of elements 
that may somtimes occur to the left of a VR verb. Together with the verb, these 
always form a more or less fixed and idiomatic expression. Some examples are 
listed below (see also \cite{ans}, p. 1012 ff.):

\[
\begin{array}[t]{ll@{\hspace{0cm}}l}
a. && \ldots\mbox{dat deze stichting in korte tijd veel goeds heeft tot stand 
gebracht.} \\
b. && \ldots\mbox{dat de man ontkent zich aan zwendel te hebben schuldig 
gemaakt} \\
c. && \ldots\mbox{dat Deng de Russen had duidelijk gemaakt, dat het om 
een beperkte actie ging.}
\end{array}
\]

The heterogenous nature of this class of elements and expressions is further 
support for an analysis in which the phrasal status of complex constituents is a 
lexical property of the verbs involved.
}
we introduce an auxiliary feature {\sc rphrase}, (for `resultant
phrase') which may be present on the argument category of functors.  The rule is
that a functor specifies its argument as {\sc +rphrase}, but there are some
exceptional cases where this specification is left out.  Sample lexical entries
are provided below.

\begin{equation}
\begin{array}[t]{ll}
\mbox{lezen :} & \mbox{\sc np[+rphrase]$\backslash$vp} \\ 
\mbox{denken :}& \mbox{\sc pp[+rphrase]$\backslash$vp} \\
\mbox{leren :} & \mbox{\sc vp}/ \mbox{\sc vp[+rphrase]} \\ 
\mbox{spreken :} & \mbox{\sc pref$\backslash($ np[+rphrase]$\backslash$vp)}
\end{array}
\end{equation}

The feature-specification {\sc [+rphrase]} specifies that the result of
combining a functor with this argument is a phrase.  Thus, application is
extended as follows:

\pr
\pred
\head{\rel{application}(\var{Value},\var{Functor},\var{Argument}) \; \prologif }
\body{\var{Functor}\con{:arg} \Leftrightarrow \con{Argument},}
\body{\var{Functor}\con{:val} \Leftrightarrow \con{Value},}
%\body{\var{Value}\con{:phrase} \Leftrightarrow \con{Functor}\con{:phrase},}
\body{\var{Value}\con{:phrase} \Leftrightarrow \con{Argument}\con{:rphrase}.}
\epred
\epr

Consider now the following examples, where {\sc p} and {\sc rp} abbreviate {\sc
phrase} and {\sc rphrase}.  For VR verbs, we supply the category as it can be
derived using division.  The effect of `argument inheritance' is made explicit
in the form of an index.

\begin{equation}
\label{boekwillezen}
\mbox{\sc
\begin{tabular}[t]{ccccc}
\ldots {\em een boek}& {\em wil} & {\em lezen} \\
\ldots np & (np$_i\backslash$vp)$/$(np$_i\backslash$vp)[-p] & 
np[+rp]$\backslash$vp \\
\cline{2-3}
& \multicolumn{2}{c}{np[+rp]$\backslash$vp} \\
\cline{1-3}  \multicolumn{3}{c}{vp[+p]} 
\end{tabular}
}
\end{equation}

\begin{equation}
\label{wilboeklezen}
\mbox{\sc
\begin{tabular}[t]{ccccc}
\ldots {\em wil} & {\em een boek} & {\em lezen} \\
\ldots vp$/$vp[-p] & np & np[+rp]$\backslash$vp \\
\cline{2-3}
& \multicolumn{2}{c}{vp[+p]} \\
\cline{1-3}  \multicolumn{3}{c}{***} 
\end{tabular}
}
\end{equation} 

\noindent Example (\ref{wilboeklezen}) is correctly ruled out (as a possible
word order in subordinate clauses), as {\em een boek lezen} is a phrase.

Note also that in (\ref{boekwillezen}) the whole constituent {\em een boek wil
lezen} is a phrase, thus accounting for the fact that {\em \ldots zou een boek
willen lezen} is not possible in subordinate clauses.  This follows from the
fact that {\sc rphrase} is located on arguments and the fact that VR verbs
inherit the argument specifications of their complement by division.  That is,
even if a functor does not directly combine with its argument, as is typically
the case in VR constructions, the information that once this argument is found a
phrase has been derived, is preserved.

\subsubsection{Participle Inversion}

Cases where an auxiliary selects to the left for an extraposition verb are the
mirror image of the examples above:

\begin{equation}
\label{boekheeftgew}
\mbox{\sc
\begin{tabular}[t]{ccccc}
\ldots {\em geleerd}& {\em heeft} & {\em zich aan te passen} \\
\ldots vp$/$vp[+rp] & (vp$/$vp$_i$)[-p]$\backslash$(vp$/$vp$_i$) & 
vp[+p] \\
\cline{1-2}
\multicolumn{2}{c}{vp$/$vp[+rp]} \\
\cline{1-3}  \multicolumn{3}{c}{vp[+p]} 
\end{tabular}
}
\end{equation}

\begin{equation}
\label{gewboekheeft}
\mbox{\sc
\begin{tabular}[t]{ccccc}
\ldots {\em geleerd}& {\em zich aan te passen} & {\em heeft}  \\
\ldots vp$/$vp[+rp] & vp[+p] & vp[-p]$\backslash$vp \\
\cline{1-2}
\multicolumn{2}{c}{vp[+p]} \\
\cline{1-3}  \multicolumn{3}{c}{***} 
\end{tabular}
}
\end{equation}

One additional assumption is needed for these cases, however.  Consider the
example in (\ref{zouetc}).  The verb cluster {\em geleerd hebben} is not a
phrase, which means that it can be the argument of {\em zou}.

\begin{equation}
\label{zouetc}
\mbox{\sc
\begin{tabular}[t]{lcccc}
$^*$\ldots {\em zou} & {\em geweigerd}& {\em hebben} & {\em zich aan te passen} 
\\
(vp$/$vp$_i$)/(vp$/$vp$_i$)[-p] & vp$/$vp[+rp] & 
(vp$/$vp$_j$)[-p]$\backslash$(vp$/$vp$_j$) & 
vp[+p] \\
\cline{2-3}
& \multicolumn{2}{c}{vp$/$vp[+rp]} \\
\cline{1-3}  \multicolumn{3}{c}{vp$/$vp[+rp]} 
\end{tabular}
}
\end{equation}

\noindent To rule out these cases, we simply add the information that the
argument of left-directional {\em hebben} is {\sc +phrase}.  Thus, we get:

\begin{equation}
\mbox{\sc
\begin{tabular}[t]{lcccc}
$^*$\ldots {\em zou} & {\em geweigerd}& {\em hebben} & {\em het boek te lezen} 
\\
(vp$/$vp$_i$)/(vp$/$vp$_i$)[-p] & vp$/$vp[+rp] & 
(vp$/$vp$_j$)[-p,+rp]$\backslash$(vp$/$vp$_j$) & 
vp[+p] \\
\cline{2-3}
& \multicolumn{2}{c}{(vp$/$vp[+rp])[+p]} \\
\cline{1-3}  \multicolumn{3}{c}{***} 
\end{tabular}
}
\end{equation}

\noindent Note that since we are using two features to code information about
the phrasal status of constituent, no contradiction arises on the argument of
{\em hebben}.  The correct word order for cases where a left-directional
auxiliary is selected by a VR verb is shown in (\ref{zous}).  The derivation of
the verb cluster is as shown in (\ref{zouoke}).

\begin{equation}
\label{zous}
\mbox{...dat Jan geweigerd zou hebben het contract te tekenen}
\end{equation}

\begin{equation}
\label{zouoke}
\mbox{\sc
\begin{tabular}[t]{ccccc}
{\em geweigerd}& {\em zou} & {\em hebben}  \\
vp$/$vp[+rp] 
&((vp$/$vp)$_j$$\backslash$(vp$/$vp$_i$))/ &  
(vp$/$vp$_k$)[-p,+rp]$\backslash$(vp$/$vp$_k$) \\
& \hspace{1cm}((vp$/$vp)$_j$$\backslash$(vp$/$vp$_i$))[-p] \\
\cline{2-3}
& \multicolumn{2}{c}{(vp$/$vp$_l$)[-p,+rp]$\backslash$(vp$/$vp$_l$)} \\
\cline{1-3}  \multicolumn{3}{c}{(vp$/$vp[+rp])[+p]} 
\end{tabular}
}
\end{equation}

\subsubsection{Seperable Verb Prefixes}
\label{prefixes}

There is a class of verbs in Dutch, consisting of a prefix and a verbal stem,
whose prefix may be seperated from the stem.  Seperation is possible if the verb
is a {\em te}-infinitive and/or if it is selected by a VR verb:

\begin{equation}
\label{sep-prefix2}
\begin{array}[t]{ll@{\hspace{0cm}}l}
a. && \ldots\mbox{dat Jan Marie niet durft aan te spreken}. \\
b. && \ldots\mbox{dat Jan Marie niet aan durft te spreken}. \\
c. && \ldots\mbox{dat Jan Marie  wil aanspreken}. \\
d. && \ldots\mbox{dat Jan Marie aan wil spreken}. \\
e. && \ldots\mbox{dat Jan Marie heeft aan durven spreken}. \\
\end{array}
\end{equation}

\noindent An example such as (\ref{sep-prefix2}e) shows that apart from a
position adjacent to the verb selecting for it or a position on the
left-periphery of the the verb cluster, the prefix may also be positioned inside
the verb cluster.  These facts can be accounted for by assigning the following
category to {\em spreken}:

\begin{equation}
\begin{array}{ll}
\mbox{spreken :} & \mbox{\sc pref[aan]$\backslash$(np[+rp]$\backslash$vp)} 
\end{array}
\end{equation}

The constituent {\em aan te spreken} is not a phrase, and thus can be selected
by a VR verb.  However, via division, a VR verb can also select a verb which has
not yet combine with its prefix.  This leads to the cases where the prefix
occurs to the left of a VR verb.  The derivation of (\ref{sep-prefix2}e) is
shown below.

\begin{equation}
\mbox{\sc
\begin{tabular}[t]{lcccc}
{\em heeft} & {\em aan}& {\em durven} & {\em spreken} \\
(np$_i$$\backslash$vp)/ &  pref & 
(pref$_i$$\backslash$(np$_j$$\backslash$vp))/
& pref$\backslash$(np[+rp]$\backslash$vp) \\
\hspace{0.5cm}(np$_i$$\backslash$vp)[-p] 
&& \hspace{1cm}(pref$_i$$\backslash$(np$_j$$\backslash$vp))[-p] \\
\cline{3-4}
&& \multicolumn{2}{c}{pref$\backslash$(np[+rp]$\backslash$vp)} \\
\cline{2-4}  \multicolumn{3}{c}{np[+rp]$\backslash$vp} \\
\cline{1-3}  \multicolumn{3}{c}{np[+rp]$\backslash$vp} \\
\end{tabular}
}
\end{equation}

We have analyzed instances of partial extraposition by assuming that PE verbs
are subject to the same rule of division as VR verbs.  PE verbs, however, may
not seperate a prefix from the verb selecting for it:

\begin{equation}
\label{sep-prefix-PE}
\begin{array}[t]{ll@{\hspace{0cm}}l}
a. && \ldots\mbox{dat Jan haar  heeft besloten op te bellen}. \\
b. &^*& \ldots\mbox{dat Jan haar op heeft besloten te bellen}. \\
c. &^*& \ldots\mbox{dat Jan haar heeft op besloten te bellen}. 
\end{array}
\end{equation}

\noindent The difference between these cases, and the VR examples in
(\ref{sep-prefix2}) is that PE verbs do not require their argument to be {\sc
-phrase}, but that they do specify their argument as {\sc +rphrase}.  Thus, a
phrase such as {\em schijnt te bellen} is not marked as {\sc +phrase}, whereas
{\em besloten te bellen} is.  This suggests that prefixes may be attached to
nonphrasal constituents, but not to phrasal constituents.\footnote{ Note that
something similar needs to be said to prevent adjuncts from occuring in between
a prefix and its verb:

\[ i. ^*\ldots\mbox{dat Jan Marie op gisteren belde.} \]

\noindent If adjuncts are introduced as arguments, there is principle nothing to
prevent the derivation of such examples.  If adjunct-arguments are marked {\sc
+rphrase}, however, the same mechanism that accounts for PE cases also rules out
this example.}  
We can implement this idea by specifying prefixes as {\sc
-phrase} (a reasonable assumption, given the fact that they can be incorporated
into a root).  To rule out the ungrammatical cases in (\ref{sep-prefix-PE}) we
furthermore must require that in application, the values of {\sc phrase} on
argument and functor must match (\ref{application}).  That is, only nonphrasal 
functors may select a
nonphrasal argument.  Note that the feature {\sc phrase} is not passed on from
functor or argument to value, and that, apart from prefixes, no constituents are
ever marked {\sc -phrase}, so that this solution can be adopted without
interfering with the earlier examples.

\pr
\label{application}
\pred
\head{\rel{application}(\var{Value},\var{Functor},\var{Argument}) \; \prologif }
\body{\var{Functor}\con{:arg} \Leftrightarrow \con{Argument},}
\body{\var{Functor}\con{:val} \Leftrightarrow \con{Value},}
\body{\var{Value}\con{:phrase} \Leftrightarrow \con{Argument}\con{:rphrase},}
\body{\var{Argument}\con{:phrase} \Leftrightarrow \con{Functor}\con{:phrase}.}
\epred
\epr
 
\noindent The matching and specification of the {\sc phrase} and {\sc 
rphrase} features in functor, argument, and value according to the  modified 
definition of {\em application} can be visualized as follows:

\begin{equation}
\mbox{\sc
\begin{tabular}[t]{lcccc}
Functor & Argument \\
$\Downarrow$ & $\Downarrow$ \\
A$/_{\alpha\mbox{\sc phrase}}$B &	B[$\alpha$phrase,$\beta$rphrase] \\
\cline{1-2} 
\multicolumn{2}{c}{A[$\beta$phrase]} \\
\multicolumn{2}{c}{$\Uparrow$} \\
\multicolumn{2}{c}{Value}
\end{tabular}
}
\end{equation}

\noindent The derivation of (\ref{sep-prefix-PE}c) is now correctly ruled out:
\begin{equation}
\mbox{\sc
\begin{tabular}[t]{lcccc}
{\em heeft} & {\em op}& {\em besloten} & {\em te bellen} \\
(np$_i$$\backslash$vp)/ &  pref[-p] & 
(pref$_i$$\backslash$(np$_j$$\backslash$vp))/
& pref$\backslash$(np[+rp]$\backslash$vp) \\
\hspace{0.5cm}(np$_i$$\backslash$vp)[+rp] 
&& \hspace{1cm}(pref$_i$$\backslash$(np$_j$$\backslash$vp))[-p] \\
\cline{3-4}
&& \multicolumn{2}{c}{pref$\backslash$(np[+rp]$\backslash$vp)[+p]} \\
\cline{2-4}  \multicolumn{3}{c}{***} 
\end{tabular}
}
\end{equation}

\subsubsection{Alternative Accounts}
\label{alternatives}

{\bf Using a LEX Feature.}  Gertjan van Noord (p.c.)  has developed an
alternative account in terms of HPSG of Dutch verb clusters.  His analysis
(based on an analysis of German verb clusters developed in
\shortcite{HinrichsNakazawa89}, \shortcite{HinrichsNakazawa93}, and
\shortcite{Nerbonne93}) simply states that all VR verbs require a {\sc +lex}
argument.  All complex phrases are {\sc -lex}.  The {\sc head-complement} rule 
schema allows a
head to combine with any number of its complements, which, together with the
possibility of argument inheritance, leads to completely `flat' {\sc vp}'s. 

The obvious advantage of this solution over the one just discussed, is its  
simplicity. As the use of a rule schema that allows arbitrary branching {\sc 
vp}'s does not seem essential, one might even consider adopting a similar 
approach in a categorial setting. That is, we could simply assume that VR verbs 
select a {\sc +lex} argument and mark all complex phrases as {\sc -lex}. 
This implies, however, that the following
cluster must receive a left-branching analysis:

\begin{equation}
\mbox{\ldots dat Marie [[zou kunnen] komen]}
\end{equation}

\noindent The definition of {\em division} presented earlier indeed makes such
an analysis possible.  The fact that a verb may not combine with any of its
arguments before being selected by a VR verb follows immediately under this
account.

There are only a few minor problems.  First of all, it is not clear how the
possibility of seperable prefixes occurring inside the verb cluster can be
accounted for.  Under this account, one would expect that prefixes that are not
incorporated lexically must always occur on the left periphery of the verb
cluster.\footnote{One might explore the possibility that prefixes are in fact
functors, taking verbs as arguments.}  Second, this proposal forces one to adopt
a left-branching structure for clusters of VR verbs.  In footnote
\ref{against-lb} we observed that this may be problematic.


{\bf Locate PHRASE on value categories.}  Instead of working with two features
 on arguments to control the marking and selection of phrasal constituents, one
 could also try to use only one ({\sc phrase}) feature.  In that case, functors
 must specify on their values whether they give rise to phrasal constituents.
 Thus, we would obtain the following lexical entries:

\begin{equation}
\begin{array}[t]{ll}
\mbox{lezen :} & \mbox{\sc np$\backslash$vp[+phrase]} \\ 
\mbox{denken :}& \mbox{\sc pp$\backslash$vp[+phrase]} \\
\mbox{weigeren :} & \mbox{\sc vp[+phrase]}/ \mbox{\sc vp} \\ 
\mbox{spreken :} & \mbox{\sc pref$\backslash($ np$\backslash$vp[+phrase])}
\end{array}
\end{equation}

The interaction of {\sc phrase} and the lexical rule of {\em division} requires
special attention in this case.  As before, VR verbs must specify that they take
a {\sc -phrase} argument after division has applied.  Also, in the rule of {\em
division} itself, we must ensure that when argument inheritance takes place, the
value of {\sc phrase} of the argument from which inheritance takes place is also
inherited.  This is necessary to rule out cases such as the following:

\begin{equation}
\mbox{\sc
\begin{tabular}[t]{lc@{\hspace{0cm}}ccc}
$^*$&{\em zou} &  {\em het boek} & {\em willen}& {\em lezen}\\
&vp$/$vp[-phrase] & np  & (np$_i\backslash$vp)$/$(np$_i\backslash$vp)[-phrase] & 
np$\backslash$vp[+phrase] \\
\cline{4-5}
&&& \multicolumn{2}{c}{np$\backslash$vp} \\
\cline{3-5} 
&&\multicolumn{2}{c}{vp} \\
\end{tabular}
}
\end{equation}

\noindent If {\em division} is applied as defined previously, the {\sc +phrase}
feature on the value of {\em lezen} is simply lost, and consequently, it is no
longer recognized that {\em een boek willen lezen} is phrasal.  Thus, we need to
modify {\em division} as follows:

\pr
\pred
\head{\rel{division}(\var{In},\var{In}).}
\head{\rel{division}(\var{In},\var{Out}) \; \prologif 
	\mbox{\hspace{2cm} \% rightward disharmonic}}
\body{\var{Out} \Leftrightarrow 
(\var{Z}\backslash\var{X})/(\var{Z}\backslash\var{Y}),}
\body{\var{X}\con{:phrase} \Leftrightarrow \var{Y}\con{:phrase}},
\body{\rel{division}(\var{In},\var{X}/ \var{Y}).}
\head{(\rel{etc.})}
\epred
\epr

\noindent Given these modifications, the basic facts follow as before.  For PE
verbs and the category of the auxiliary in participle inversion, we must assume
that the specification {\sc +phrase} on the value category is added after
division has taken place.  It is not clear, however, how the fact that seperable
prefixes may not appear to the left of PE verbs can be accounted for.

\subsection{Eliminating spurious ambiguity}
\label{sp-ambig}

A grammar employing both application and division is usually in danger of
producing spurious derivations of certain sentences.  This is also the case for
the present fragment.  The feature {\sc phrase}, however, already serves to
disambiguate many cases that would otherwise have two or more derivations.  The
ambiguity that remains can be eliminated by restricting harmonic division to
{\sc +phrase} arguments only.

Consider first the interaction of VR and extraposition verbs. Ignoring {\sc 
phrase}, 
the following example has two derivations:

\begin{equation}
\mbox{\sc
\begin{tabular}[t]{ccccc}
\ldots & {\em moet} & {\em leren}& {\em zich aan te passen} \\
	1. & vp$/$vp[-p] & vp$/$vp[+rp] 	& vp \\
	& \\
	2. & (vp$/$vp)/(vp$/$vp)[-p] & vp$/$vp[+rp] 	& vp \\
\end{tabular}
}
\end{equation}

\noindent We could either simply combine {\em leren} with the {\sc vp} {\em
zich aan te passen} and combine the resulting {\sc vp} with {\em moet} or we
could apply harmonic division to {\em moet} and then derive the constituent {\em
moet leren} before combining with the extraposed {\sc vp}.  The introduction
of {\sc phrase} will rule the first derivation out, as the combination of an
extraposition verb with its complement is phrasal.

Similarly, consider cases of participle inversion.

\begin{equation}
\mbox{\sc
\begin{tabular}[t]{ccccc}
\ldots & {\em het boek} & {\em gelezen}& {\em heeft} \\
1. & np & np[+rp]$\backslash$vp   &   vp[-p]$\backslash$vp  \\
			&&\\
2. & np & np[+rp]$\backslash$vp  & 
(np$\backslash$vp)[-p]$\backslash$(np$\backslash$vp)
\end{tabular}
}
\end{equation}
 
\noindent We might derive the {\sc vp} {\em het boek gelezen} first, and next
combine that with {\em heeft} using just application, but this is blocked by the
fact that {\em hebben} selects a {\sc -phrase} argument.  The only derivation
left is therefore the one on which harmonic division applies to {\em heeft},
which then combines with {\em gelezen} and {\em een boek} in that order.

Two remaining cases of spurious ambiguity are the following:

\begin{equation}
\label{spurious}
\mbox{\sc
\begin{tabular}[t]{ccccc}
\ldots & {\em zou} & {\em kunnen}& {\em komen} \\
1.& 	vp$/$vp[-p] & vp$/$vp[-p] 	& vp \\
	&& \\
2. &	(vp$/$vp$_i$)/(vp$/$vp$_i$)[-p] & vp$/$vp[-p] 	& vp \\
\end{tabular}
}
\end{equation}
\begin{equation}
\label{pref-spu}
\mbox{\sc
\begin{tabular}[t]{ccccc}
\ldots & {\em op} & {\em gebeld}& {\em heeft} \\
1. &	pref[-p] & pref$\backslash$(np$\backslash$vp) &  vp[-p]$\backslash$vp \\
	&& \\
2. &	pref[-p] & pref$\backslash$(np$\backslash$vp) & 
	(pref$\backslash$(np$\backslash$vp))[-p]$\backslash
			$(pref$\backslash$(np$\backslash$vp))
\end{tabular}
}
\end{equation}

\noindent As {\em kunnen komen} is not phrasal, both the left-branching and the
right-branching derivation of (\ref{spurious}) is available.  In the preceding
discussion, we have assumed that the right-branching derivation is correct.  The
left-branching derivation can be ruled out by assuming that in harmonic
division, the inherited argument is always {\sc +phrase} (see
(\ref{div-no-sp})).  This blocks harmonic division in those cases where it is
applied to a VR verb to derive an argument category that is also a VR verb.  The
same constraint will also rule out a left-branching derivation of
(\ref{pref-spu}).

\pr
\label{div-no-sp}
\pred
\head{\rel{division}(\var{In},\var{In}).}
\head{\vdots}
\head{\rel{division}(\var{In},\var{Out}) \; \prologif
	\mbox{\hspace{2cm} \% rightward harmonic}}
\body{\var{Out} \Leftrightarrow 
(\var{X}/ \var{Z})/(\var{Y}/ \var{Z}),}
\body{\var{Z}\con{:phrase} \Leftrightarrow +},
\body{\rel{division}(\var{In},\var{X}/ \var{Y}).}
\head{\rel{division}(\var{In},\var{Out}) \; \prologif
	\mbox{\hspace{2cm} \% leftward harmonic}}
\body{\var{Out} \Leftrightarrow 
(\var{Z}\backslash\var{Y})\backslash(\var{Z}\backslash\var{X}),}
\body{\var{Z}\con{:phrase} \Leftrightarrow +,}
\body{\rel{division}(\var{In},\var{Y}\backslash\var{X}).}
\epred
\epr

\section{Minor Issues}

\subsection{Om}

Most extraposition verbs select for a {\sc vp} which may be optionally prefixed
by the complementizer {\em om} (\ref{om1}a,b).  For other extraposition verbs,
the presence of {\em om} is impossible (\ref{om1}c,d).  With VR verbs, {\em om}
is always impossible (\ref{om1}e,f). With PE verbs, {\em om} is only possible if 
the complement {\sc vp} has been extraposed fully (i.e. in those cases where 
{\em partial extraposition} is in fact full extraposition) (\ref{om1}g-i)

\begin{equation}
\label{om1}
\begin{array}[t]{ll@{\hspace{0cm}}l}
a. && \ldots\mbox{dat Jan beloofde  haar te zullen bellen.} \\
b. && \ldots\mbox{dat Jan beloofde om haar te zullen bellen.} \\
c. && \ldots\mbox{dat Jan zei  haar te zullen bellen.} \\
d. &^*& \ldots\mbox{dat Jan zei om haar te zullen bellen.} \\
e. &^*& \ldots\mbox{dat Jan om het boek schijnt te lezen} \\
f. &^*& \ldots\mbox{dat Jan het boek schijnt om te lezen} \\
g. &^*& \ldots\mbox{dat Jan om de deur beloofde te schilderen} \\
h. &^*& \ldots\mbox{dat Jan de deur beloofde om te schilderen} \\
i. && \ldots\mbox{dat Jan beloofde om de deur te schilderen} \\
\end{array}
\end{equation}

We may account for this fact by assuming that {\em om} is a {\sc vp}-specifier,
with the category indicated below.  Extraposition verbs which optionally take
{\em om}-clauses are are unspecified for the feature {\sc om}, whereas
extraposition verbs which cannot combine with {\em om}-clauses are {\sc -om}.

\begin{equation}
\begin{array}[t]{ll}
\mbox{om :} & \mbox{\sc vp[+om]$/$vp[-om]} \\ 
\mbox{beloven :}& \mbox{\sc vp$/$vp[+rp]} \\
\mbox{zeggen :} & \mbox{\sc vp$/$vp[+rp,-om]} \\ 
\end{array}
\end{equation}

The ungrammaticality of the examples (\ref{om1}e-h) follows from the assumption
that {\em om} takes a {\sc vp} as argument, together with the fact that at no
point in the derivation a {\sc vp} is derived that could function as argument of
{\em om}.  Note that this does not rule out cases where a VR
verb selects a {\em te-infinitive} consisting of a simple intransitive {\sc vp}, 
as in \ref{om2} below.  To rule these out, we must specify for all VR verbs 
(selecting a {\em te-infinitive}) that they select for a {\sc vp} complement that is 
marked {\sc -om}.\footnote{
The reason that VR verbs cannot combine with an {\em om}-clause seems to be that 
a {\sc vp} headed by {\\em om} cannot be part of a verb cluster. One reasonable 
possibility to achieve this, would be to specify the argument of {\em om} as 
{\sc +rprhase}. This does not suffice, however, as it fails to rule out the 
left-branching analysis of (\ref{om2}), i.e. {\em [schijnt om] te verliezen]}.}

\begin{equation}
\label{om2}
^* \ldots\mbox{dat Jan  schijnt om te verliezen.} 
\end{equation}


\subsection{Infinitivus pro Participio}

The auxiliaries {\em hebben} and {\em zijn} normally select a {\sc vp}
headed by a participle (\ref{aux}a,b). If the {\sc vp} is headed by a VR
verb, however, this verb must be in the infinitive form (\ref{aux}c,d). This
phenomenon is known as  {\em Infinitivus pro Participio} ({\sc ipp}). 

\begin{equation}
\label{aux}
\begin{array}[t]{ll@{\hspace{0cm}}l}
a. && \ldots\mbox{dat Jan  is gevallen} \\
b. && \ldots\mbox{dat Jan het boek heeft gelezen.} \\
c. &^*& \ldots\mbox{dat Jan het boek is gaan kopen.} \\
d. && \ldots\mbox{dat Jan het boek heeft moeten lezen.} \\
\end{array}
\end{equation} 

The possibility of {\sc Ipp} is sometimes used as a test to distinguish between
(partial) extraposition and VR verbs.  Only proper VR verbs give rise to the
{\sc ipp}-effect.  Note that this implies that {\em proberen} must be ambiguous
between a VR verb and a PE verb:

\begin{equation}
\begin{array}[t]{ll@{\hspace{0cm}}l}
a. && \ldots\mbox{dat we water hebben verzuimd te drinken} \\
b. &^*& \ldots\mbox{dat we water hebben verzuimen te drinken} \\
c. && \ldots\mbox{dat Jan het boek heeft geprobeerd te lezen} \\
d. && \ldots\mbox{dat Jan het boek heeft proberen te lezen.} \\
\end{array}
\end{equation}

\noindent We can account for these facts either by assuming that there is an
additional entry for auxiliaries, selecting {\sc vp}'s headed by a VR verb in
the infinitive (i.e.  their argument would (minimally) be an {\sc vp[inf,vr]})
or we can assume that the infinitive form of VR verbs can also function as
participle (i.e.  {\em willen} would be {\sc vp[prt]} as well as {\sc vp[inf]}).
We have adopted the latter solution.

\subsection{Participle and Modal Inversion}

Auxiliaries may select a participle to the right as well as to the left. The 
left-directional possibility is not available, however, if the {\sc vp} selected 
by the auxiliary is headed by a VR verb (and is therefore headed by an 
infinitive).

\begin{equation}
\label{ipp-inv}
\begin{array}[t]{ll@{\hspace{0cm}}l}
a. && \ldots\mbox{dat Jan  het boek gelezen heeft.} \\
b. && \ldots\mbox{dat Jan het boek lezen moet.} \\
c. && \ldots\mbox{dat Jan het boek is gaan kopen.} \\
c^\prime. &^*& \ldots\mbox{dat Jan het boek gaan is kopen.} \\
c^{\prime\prime}. &^*& \ldots\mbox{dat Jan het boek gaan kopen is.} \\
d. && \ldots\mbox{dat Jan het boek heeft moeten lezen.} \\
d^\prime. &^*& \ldots\mbox{dat Jan het boek moeten heeft lezen.} \\
d^{\prime\prime}. &^*& \ldots\mbox{dat Jan het boek moeten lezen heeft.} \\
\end{array}
\end{equation} 

We have already presented an account of participle inversion, but, given our
solution of {\sc ipp} which simply treats infinitives of VR verbs as
participles,  we still need to account for the
ungrammaticality of the primed instances of(\ref{ipp-inv}c,d).

A closer look at the examples (\ref{ipp-inv}c$^\prime$,d$^\prime$) shows
that they are in fact not derivable.

\begin{equation}
\label{gaaniskopen}
\mbox{\sc
\begin{tabular}[t]{ccccc}
\ldots & {\em gaan} & {\em is}& {\em kopen} \\
&	tv$/$tv[-p] & (tv$/$tv$_i$)[-p,+rp]$\backslash$(tv$/$tv$_i$) &  tv \\
\cline{2-3}
& \multicolumn{2}{c}{(tv$/$tv[-p])[+p]} \\
\cline{2-4}
& \multicolumn{2}{c}{***} \\
\end{tabular}
}
\end{equation}

The only possibility of deriving the verb cluster {\em gaan is kopen} is by 
combining {\em gaan} and {\em is } first. This gives rise to a phrasal 
category, which is looking for a nonphrasal complement. In section 
\ref{prefixes}, where we discussed the impossibility of seperating a prefix 
form its verb root by a PE verb (as in (\ref{pref-ex})), we have encountered a 
similar situation.

\begin{equation}
\label{pref-ex}
\begin{array}[t]{ll@{\hspace{0cm}}l}
a. && \ldots\mbox{dat Jan haar  heeft besloten op te bellen}. \\
b. &^*& \ldots\mbox{dat Jan haar op heeft besloten te bellen}. \\
c. &^*& \ldots\mbox{dat Jan haar heeft op besloten te bellen}. 
\end{array}
\end{equation}
There, we have proposed a constraint requiring that the {\sc phrase} feature on 
functor and argument must unify. This constraint not only accounts for 
examples such as (\ref{pref-ex}), but also blocks the derivation of 
(\ref{gaaniskopen}). Since {\em gaan is} is a phrasal constituent selecting a 
nonphrasal argument, no combination of this functor with its argument is 
possible.

This still leaves the examples in 
(\ref{ipp-inv}c$^{\prime\prime}$,d$^{\prime\prime}$). These are derivable, as 
the 
verbal constituents {\em gaan kopen} and {\em moeten lezen} are not phrasal and 
are headed by a verb whose form is ambiguous between a participle and an 
infinitive. To rule these out, we stipulate that inverted auxiliaries select a 
{\sc -ipp} argument, and mark all VR verbs as {\sc +ipp}. 

Apart from auxiliaries, modals may also select a verbal complement to the left:

\begin{equation}
\begin{array}[t]{ll@{\hspace{0cm}}l}
a. && \ldots\mbox{dat Jan  komen wil} \\
b. && \ldots\mbox{dat Jan wil komen.} \\
c. && \ldots\mbox{dat de manager de spits wil verbieden het contract te tekenen} \\
d. && \ldots\mbox{dat de manager de spits verbieden wil het contract te tekenen.} \\
\end{array}
\end{equation} 
The possibility of modal inversion is restricted to finite forms of modals:
\begin{equation}
\begin{array}[t]{ll@{\hspace{0cm}}l}
a. && \ldots\mbox{dat Jan  heeft willen komen.} \\
b. &^*& \ldots\mbox{dat Jan heeft komen willen.} \\
c. &^*& \ldots\mbox{dat zou komen heeft willen} 
\end{array} 
\end{equation} 
We can account for this fact by making a left-directional category 
available only for finite modals:
\begin{equation}
\begin{array}{ll}
\mbox{wil:} & \mbox{\sc (vp[-p,+rp]$\backslash$vp)[fin]} \\ 
\mbox{zal:} & \mbox{\sc (vp[-p,+rp]$\backslash$vp)[fin]} \\ 
\end{array}
\end{equation}

Another restriction on modal inversion is that the selected {\sc vp} may not be 
headed by a VR verb:
\begin{equation}
\label{modal-inv}
\begin{array}[t]{ll@{\hspace{0cm}}l}
a. && \ldots\mbox{dat Jan  zal willen komen.} \\
b. &^*& \ldots\mbox{dat Jan willen zal komen.} \\
c. &^*& \ldots\mbox{dat Jan willen komen zal.} 
\end{array} 
\end{equation}

\noindent Thus, modal inversion is subject to the same constraint as participle
inversion.  As was the case with participle inversion, we may accunt for the 
ungrammaticality of the examples in (\ref{modal-inv}) by stipulating that 
inverted modals select a {\sc -ipp} verbal argument. 

The connection between  the ungrammaticality of (\ref{gaaniskopen}) and
(\ref{modal-inv}b,c)  and the (im-)possibility of {\sc ipp} on the governed verb is
further confirmed by the observation that instances of participle inversion and
modal inversion are grammatical when the argument is headed by a PE verb (which 
are not subject to {\sc ipp}):

\begin{equation}
\begin{array}[t]{ll@{\hspace{0cm}}l}
a. && \ldots\mbox{dat Jan  haar besloten heeft te kussen.} \\
b. && \ldots\mbox{dat Jan haar besluiten zou te kussen.} \\
\end{array} 
\end{equation}

\section{Fronting of Verbal Constituents}

In lexicalist accounts of German verb clusters, the possibility of fronting full 
or partial {\sc vp}'s as well parts of a verb cluster has received considerable 
attention \cite{Nerbonne86},\cite{Johnson86},\cite{Nerbonne93}. 

For Dutch, the data are roughly as follows. Although fronting of verbal 
constituents in general appears to be a rather marked phenomenon, it is possible 
in Dutch to topicalize full {\sc vp}'s (\ref{fronting}a,b), as well as 
partial {\sc vp}'s (\ref{fronting}c,d):\footnote{
Examples (\ref{fronting}a-c) are taken 
from the ANS \cite{ans} and (\ref{fronting}d) is from \shortcite{hoeksema-ms})}

\begin{equation}
\label{fronting}
\begin{array}[t]{ll@{\hspace{0cm}}l}
a. && \mbox{Iets gedronken heb je natuurlijk wel?} \\
b. && \mbox{Het proefstuk gauw even afmaken wilde hij niet.} \\
c. && \mbox{Aanvaarden zul je het.} \\
d. && \mbox{Uit kunnen staan heb ik die man nooit} \\
\end{array} 
\end{equation}

The possibility of fronting full constituents is something that requires
explanation, given the fact that the functors selecting for these constituents
are VR verbs, and thus normally combine with a nonphrasal argument.  In
\shortcite{Nerbonne93}, it is suggested that the possibility of fronting full as
well as partial {\sc vp}'s can be accounted for, without introducing spurous
ambiguity elsewhere, by stipulating that the lexical rule which introduces
incomplete constituents (by moving an element from {\sc subcat} to {\sc slash})
removes the specification {\sc +lex} from its argument.  This suggestion is
easily incorporated in the present framework.

The lexical rule that optionally moves an element to {\sc slash} is given below.
The predicate {\em unify\_except} unifies its two first arguments, except for
the features specified in the third argument.

\pr
\pred
\head{\rel{push\_to\_slash}(\var{In},\var{Out}) \; \prologif}
	\body{\var{In} \Leftrightarrow \var{Out}.}
\head{\rel{push\_to\_slash}(\var{In},\var{Out}) \; \prologif}
	\body{\var{Out}\con{:slash} \Leftrightarrow \var{Slash},}
	\body{\rel{push\_to\_slash}(\var{In},\var{Out},\var{Slash}).}
\epred
\pred
\head{\rel{push\_to\_slash}(\var{In},\var{Out},\var{Slash})  ;\ \prologif}
	\body{\var{In}\con{:arg} \Leftrightarrow \var{Arg},}
	\body{\rel{unify\_except}(\var{Arg},\var{Slash},[\con{phrase}]),}
	\body{\var{In}\con{:val} \Leftrightarrow \var{Out}.}
\head{\rel{push\_to\_slash}(\var{In},\var{Out},\var{Slash})  ;\ \prologif}
	\body{\var{In}\con{:arg} \Leftrightarrow \var{Out}\con{:arg},}
	\body{\var{In}\con{:dir} \Leftrightarrow \var{Out}\con{:dir},}
	\body{\var{In}\con{:val} \Leftrightarrow \var{InVal},}
	\body{\var{Out}\con{:val} \Leftrightarrow \var{OutVal},}
	\body{\rel{push\_to\_slash}(\var{InVal},\var{OutVal},\var{Slash}) .}
\epred
\epr

The constraint {\em push\_to\_slash} can be  added to verbal lexical entries:


\pr
\pred
\head{\rel{lex}(\con{willen},\var{Sign}) \; \prologif }
	\body{\var{Sign0} \Leftrightarrow \con{vp}/\con{vp},}
	\body{\rel{division}(\var{Sign0},\var{Sign1}),}
	\body{\rel{push\_to\_slash}(\var{Sign1},\var{Sign}).}
\epred
\pred
\head{\rel{lex}(\con{kussen},\var{Sign}) \; \prologif }
	\body{\var{Sign0} \Leftrightarrow \con{np}\backslash\con{vp},}
	\body{\rel{push\_to\_slash}(\var{Sign0},\var{Sign}).}
\epred
\epr

\noindent By ordering the application of {\em push\_to\_slash} after {\em
division}, we account for the fact that a partial {\sc vp} may be fronted.  That
is, it accounts for the fact that a VR verb may combine with arguments inherited
from the head of its complement, before actually combining with that head.

\subsection{Remaining Issues}
Our treatment of fronting of (partial) {\sc vp}'s predicts that full {\sc vp}'s 
can be fronted and that a verb along with any number of its arguments can be 
fronted in case this verb is selected by a VR verb. There are a number of 
additional constraints on fronting, however, some of which are discussed below.

{\bf Fronting single VR verbs.} In \shortcite{Nerbonne93} it is observed that 
while it is possible to front a 
complex verbal constituent, it is impossible to front a single modal. A Dutch 
example of fronting a verb cluster is given in (\ref{fronting}d). Fronting of a 
single VR verb in Dutch is impossible:

\begin{equation}
\begin{array}[t]{ll@{\hspace{0cm}}l}
a.  &^*& \mbox{Kunnen heb ik die vent nooit uitstaan.} \\
b. &^*& \mbox{Moeten heb ik dit boek lezen}
\end{array} 
\end{equation}

\noindent The ungrammaticality of these examples follows from the fact that {\em
hebben} cannot combine with {\em uitstaan} before combining with the head of its
complement {\em kunnen}.  Remember that in section \ref{sp-ambig} we introduced
a constraint restricting applications of harmonic composition to situations in
which a phrasal argument is inherited.  This constraint is needed to eliminate
the spurious ambiguity caused by the introduction of harminic division.  The
same constraint also precludes the derivation of the ungrammatical examples
above.  Ignoring the effects of verb second, we need a to derive roughly the
following:

\begin{equation}
\mbox{\sc
\begin{tabular}[t]{ccccc}
{\em kunnen} & \ldots & {\em heb} & {\em uitstaan} \\
tv/tv[-p]&	tv$/$tv  &  tv \\
&& $\Downarrow$ \\
&& (tv$/$tv$_i$[+p])$/$(tv$/$tv$_i$[+p]) \\
&& $\Downarrow$ \\
&& (tv$/$tv$_i$[+p])[slash:tv$/$tv$_i$[+p]] \\
\cline{3-4}
&& \multicolumn{2}{c}{tv[slash:tv$/$tv[+p]]} \\
\end{tabular}
}
\end{equation}

\noindent First, harmonic division must be applied to {\em heb}, leading to the
specification {\sc [+phrase]} on the inherited argument.  Next, we can move the
first argument of {\em heb} to {\sc slash}, and combine {\em heb} with {\em
uitstaan}.  The rest of the derivation must fail, however, as there is a
category {\sc tv/tv[+p]} on {\sc slash}, while the fronted element is in fact of
category {\sc tv/tv[-p]}.

{\bf Fronting Participles.}
Fronting of a single participle in general is possible, and can typically 
be found in newspaper prose \cite{ans}:

\begin{equation}
\begin{array}[t]{ll@{\hspace{0cm}}l}
a. && \mbox{Aangenomen wordt dat de voorzitter ontvoerd is.} \\
b. && \mbox{Gedacht wordt aan een verhoging van de olieprijs.} \\
\end{array} 
\end{equation}

\noindent In this somewhat particular construction (note, for instance, that the 
governing verb is always the passive auxiliary), it is also possible to front a 
single extraposition verb:

\begin{equation}
\begin{array}[t]{ll@{\hspace{0cm}}l}
a. && \mbox{Besloten werd het voorstel te accepteren.} \\
b. && \mbox{Voorgesteld werd om Joop een nieuw contract aan te bieden.} \\
\end{array} 
\end{equation}

\noindent In this respect, extrapostion verbs differ from VR verbs, which cannot 
be fronted without the verb they govern. The difference is as prediced, given 
the fact that to front a single extrapostion or VR verb we need to apply 
harmonic division, which is unproblematic if the inherited argument is phrasal. 
This is the case for extraposition verbs, but not for VR verbs.

{\bf Fronting with non-finite VR verbs.} The examples cited at the beginning of 
this section all consist of a fronted verbal constituent which is selected by a 
finite verb in verb second position (i.e. immediately following the fronted 
material). It is difficult to construct examples in which the governing verb is 
non-finite:

\begin{equation}
\label{fronting2}
\begin{array}[t]{ll@{\hspace{0cm}}l}
a. &?& \mbox{Dit voorstel geweigerd zou je natuurlijk wel hebben?} \\
b. && \mbox{Het proefstuk gauw even afmaken zou hij niet willen.} \\
c. && \mbox{Aanvaarden zul je het wel moeten.} \\
d. && \mbox{Uitstaan heb ik die man nooit gekund.} \\
\end{array} 
\end{equation}

The examples (\ref{fronting2}b-d) seem to be just acceptable, whereas 
(\ref{fronting2}a), in which a governing auxiliary is seperated from its 
participle,
is considerably less acceptable. 

{\bf Long distance fronting.} It is not clear to what extent fronting of verbal 
constituents is possible across clause boundaries. The example below does seem 
acceptable. 

\begin{equation}
\mbox{Aanvaarden denk ik dat ik dit voorstel wel zal moeten.} \\
\end{equation}


{\bf Fronting complements of (Partial) Extraposition verbs.}
Our account of fronting, suggests that fronting of full {\sc vp}s that are 
complements of extrapostion verbs should be unproblematic:

\begin{equation}
\begin{array}[t]{ll@{\hspace{0cm}}l}
a. && \mbox{Dit voorstel te weigeren heb ik me voorgenomen} \\
b. && \mbox{Het werkstuk volgende week af te ronden heeft Jan voorgesteld.} \\
\end{array} 
\end{equation} 

\noindent Fronting of partially extraposed verbal complements should also be 
possible, in case the governing verb is a PE verb:

\begin{equation}
\mbox{Een nieuwe taak te geven probeerden ze hem niet.} \\
\end{equation} 



{\bf Stranding Prefixes.}
In \shortcite{hoeksema-ms} it is observed that one cannot front a verb which 
selects for a prefix, while leaving the prefix behind:

\begin{equation}
\label{fronting3}
\begin{array}[t]{ll@{\hspace{0cm}}l}
a. && \mbox{Uitstaan heb ik die man nooit gekund.} \\
b. && \mbox{Uitstaan heb ik die man nooit gekund.} \\ 
c. &^*& \mbox{Staan heb ik die man nooit uit kunnen/gekund.} \\
d. &^*& \mbox{Kunnen staan heb ik die man nooit uit.} \\
\end{array} 
\end{equation}

\noindent The impossibility of (\ref{fronting3}c) in particular needs to be 
explained, as there is nothing that prevents {\em kunnen} to undergo division, 
followed by a movement of the verbal complement to {\sc slash} and finally, to 
combine with {\em uit}. 


{\bf Fronting and IPP.}  In \shortcite{Hoeksema88} it is observed that the {\sc
ipp} effect only occurs if a VR verb does in fact combine with the head of its
complement in the verb cluster.  If this is not the case, i.e.  if the governed
verb or {\sc vp} has been fronted, the VR verb appears in its participle form.

\begin{equation}
\label{fronting-ipp}
\begin{array}[t]{ll@{\hspace{0cm}}l}
a. && \mbox{Stelen heeft ze nooit gewild.} \\
b. && \mbox{Werken heeft Piet nooit gehoeven.}\\
c. && \mbox{Die man uitstaan heb ik nooit gekund.}\\
\end{array}
\end{equation}

\noindent Hoeksema takes this as further evidence for the fact that governors in 
the verbal complex may not be complex (where participles, as opposed to 
infinitives, are considered to be complex). 

It seems difficult to give a formal account of these data. One solution might be 
to enter the participle forms of VR verbs together with some constraint that 
forces the lexical rule {\em push\_to\_slash} to be applied so as to move the 
verbal argument of this verb to  {\sc slash}. While this may be a formal 
possibility, it is rather different from the mechanisms we have used so far. 

Another option might be to consider the fronted material as not verbal at all, 
but as some kind of nominalization. In that case, the participle form is not 
categorized as a VR verb, but as a verb taking a nominal argument. That this 
subcategorization must exist follows from examples such as:

\begin{equation}
\begin{array}[t]{ll@{\hspace{0cm}}l}
a. && \mbox{Dat heeft ze nooit gewild.} \\
b. && \mbox{Dat harde werken heeft ze nooit gewild.}\\
\end{array}
\end{equation} 

Note, however, that if the examples in (\ref{fronting-ipp}) involve 
nominalizations, it is hard to account for partial frontings, if these are 
grammatical:

\begin{equation}
\mbox{Uitstaan heb ik die man nooit gekund.}
\end{equation} 


\section{The Position and Scope of Adjuncts}


\subsection{Adjuncts as arguments} 

The position of verb modifiers in Dutch is not fixed.  Adjuncts can, at least in
principle, occur anywhere to the left of the verb:

\begin{equation}
\label{adj-ex}
\begin{array}[t]{llllllll}
a. & \mbox{dat} & \mbox{waarschijnlijk} & \mbox{niemand} & \mbox{het ongeluk} & 
\mbox{zag}\\
& \mbox{that} & \mbox{probably} & \mbox{nobody} & \mbox{the accident} & 
\mbox{saw} 
\\
& \multicolumn{6}{l}{\mbox{\em that noboby probably saw the accident}} \\
b. & \mbox{dat} & \mbox{Johan} & \mbox{opzettelijk} & \mbox{een ongeluk} & 
\mbox{veroorzaakt}\\
& \mbox{that} & \mbox{J.} & \mbox{deliberately} & \mbox{an accident} & 
\mbox{causes} 
\\
& \multicolumn{6}{l}{\mbox{\em that J. deliberately causes an accident}} \\
c. & \mbox{dat} & \mbox{Johan} & \mbox{Marie} & \mbox{opzettelijk} & \mbox{geen 
cadeau} & \mbox{geeft} \\
& \mbox{that} & \mbox{J.} & \mbox{M.} & \mbox{deliberately} & \mbox{no present} 
& 
\mbox{gives} \\
& \multicolumn{6}{l}{\mbox{\em that J. deliberately gave M. no present}}
\end{array}
\end{equation}

\noindent There are several ways to account for this fact.  One can assign
multiple categories to adjuncts or one can assign a polymorphic category {\sc
x$/$x} to adjuncts, with {\sc x} restricted to `verbal projections'
\cite{bouma:88}.

Alternatively, one can assume that adjuncts are not functors, but arguments of
the verb.  Since adjuncts are optional, can be iterated, and can occur in
several positions, this implies that verbs must be polymorphic.  The constraint
{\em add\_adjuncts} has this effect, as it optionally adds one or more adjuncts
as arguments to the `initial' category of a verb:

\begin{equation}
\label{add-adj}
\begin{array}[t]{l}
\att{lex}(\con{veroorzaken},\var{Sign}) \con{:-} \\
\hspace{1cm}  
\var{Sign0} \Leftrightarrow \con{np}\backslash(\con{np}\backslash\con{s})), \\
\hspace{1cm}  
\att{add\_adjuncts}(\var{Sign0},\var{Sign}). \\
\att{lex}(\con{geven},\var{Sign}) \con{ :-} \\
\hspace{1cm}
\var{Sign0} \Leftrightarrow 
\con{np}\backslash(\con{np}\backslash(\con{np}\backslash\con{s}))), \\
\hspace{1cm}  
\att{add\_adjuncts}(\var{Sign0},\var{Sign}). \\
\\
\att{add\_adjuncts}(\con{In},\con{Out}) \con{:-} \\
\hspace{1cm}  
\var{In} \Leftrightarrow \con{s}, \\
\hspace{1cm}  
\var{In} \Leftrightarrow \var{Out}. \\
\att{add\_adjuncts}( \var{In},\var{Adj}\backslash \var{OutVal}) \con{ :-} \\
\hspace{1cm} \var{Adj} \Leftrightarrow \con{adj}, \\
\hspace{1cm} \var{Adj}\con{:rphrase} \Leftrightarrow +, \\
\hspace{1cm} \att{add\_adjuncts}(\var{In}, \var{OutVal}). \\
\att{add\_adjuncts}(\var{In},\var{Out}) \con{ :-} \\
\hspace{1cm} \var{In}\con{:arg} \Leftrightarrow \var{Out}\con{:arg}, \\
\hspace{1cm} \var{In}\con{:dir} \Leftrightarrow \var{Out}\con{:dir}, \\
\hspace{1cm} \var{In}\con{:val} \Leftrightarrow \var{InVal}, \\
\hspace{1cm} \var{Out}\con{:val} \Leftrightarrow \var{OutVal}, \\
\hspace{1cm} \att{add\_adjuncts}(\var{InVal}, \var{OutVal}). \\
\end{array}
\end{equation} 

\noindent The derivation of (\ref{adj-ex}a) is given below.
{\sc 
\begin{equation}
\mbox{
\begin{tabular}[t]{rcccccc}
\ldots {\em Johan} & {\em opzettelijk} & {\em een ongeluk} & {\em veroorzaakt}\\
        np         &  adj              &  np  &  np$\backslash$(np$\backslash$ 
s) \\
                                                &&&$\Downarrow$ \\
                          &&& np$\backslash$(adj$\backslash$(np$\backslash$ s)) 
\\
\cline{3-4}     && \multicolumn{1}{c}{adj$\backslash$(np$\backslash$ s)} \\
\cline{2-3} & \multicolumn{2}{c}{np$\backslash$ s} \\
\cline{1-3}  \multicolumn{2}{c}{s} \\
\end{tabular}
}
\end{equation}
}

An interesting implication of this analysis is that in a categorial setting the
notion `head' can be equated with the notion `main functor'.  This has been
proposed by \shortcite{barry-pickering}, but they are forced to assign a
category containing Kleene-star operators to verbal elements. The present
proposal is an alternative for such assignments which avoids introducing new
categorial operators.  

The second thing to note is that treating adjuncts as arguments makes it easy to 
account for the fact that adjuncts can be extracted and fronted in {\sc 
wh}-questions, relatives, and topicalization constructions. If the lexical rule 
{\em push\_to\_slash} operates on the output of {\em add\_adjuncts}, this 
follows automatically. 

Finally, this analysis allows us to give a straightforward account of the
distribution and scope of adjuncts in verb phrases headed by a verbal complex.

\subsection{Adjuncts and verb clusters}

In sentences headed by a verb cluster, we find the same pattern as in simple 
clauses, i.e. adjuncts may occur anywhere to the left of the verb 
cluster.\footnote{Adjuncts cannot occur within the verb cluster. This is 
accounted for in the rule which adds adjuncts (\ref{add-adj}) by stipulating 
that the added adjunct argument is {\sc +rphrase}. Note that the same 
stipulation, together with the constraint that requires the phrasal status of 
functor and argument ot be unifiable, also rules out examples such as (i) 
below, in which an adjunct occurs left of a sperable prefix:


\[ i. ^*\ldots\mbox{\em dat Jan Marie op gisteren heeft gebeld.} \]
}


\begin{equation}
\begin{array}[t]{ll@{\hspace{0cm}}l}
a. && \ldots\mbox{dat waarschijnlijk niemand dat ongeluk heeft zien gebeuren.} 
\\
b. && \ldots\mbox{dat Frits iedere dag iemand vis zag eten.} \\
c. && \ldots\mbox{dat Frits hem iedere dag vis zag eten.} \\
d. && \ldots\mbox{dat Frits hem dat boek duidelijk zag stelen.} \\
\end{array} 
\end{equation}

\noindent Note that these examples are an argument against assigning multiple 
categories to adjuncts. As the subcategorization requirements of a verbal 
complex can be arbitrarily complex, it will be impossible to provide an 
exhaustive list of categories for adjuncts modifying such verb complexes. 

Another issue that needs to be accounted for is the fact that adjuncts, when 
they occur to the left of a verb cluster, may take scope over the entire 
cluster, but also over (embedded) parts of it:\footnote{
There seem to be exceptions here, as \shortcite{hoeksema-ms} observes, for 
instance, that the negation {\em niet} always takes wide scope (while noting 
that \shortcite{haegeman} claim otherwise). We might account for this fact by 
assigning {\em niet} a polymorphic category after all, or by letting VR verbs 
select for `non-negative' complements only.}

\begin{equation}
\label{interaction}
\begin{array}[t]{ll}
a. & 
\begin{array}[t]{llllll}
\mbox{dat} & \mbox{Frits} & \mbox{Marie} & \mbox{volgens mij} & \mbox{lijkt} & 
\mbox{te ontwijken.} \\
\mbox{that} & \mbox{F.}& \mbox{M.}& \mbox{to me} & \mbox{seems} & \mbox{to 
avoid} \\
\multicolumn{6}{l}{\mbox{\em It seems to me that F. avoids M.}} 
\end{array} \\
b. & 
\begin{array}[t]{llllll}
\mbox{dat}& \mbox{Frits} & \mbox{Marie} & \mbox{opzettelijk} & \mbox{lijkt} & 
\mbox{te 
ontwijken.} \\
\mbox{that}& \mbox{F.} &\mbox{M.} & \mbox{to me} & \mbox{seems} & \mbox{to 
avoid} \\
\multicolumn{6}{l}{\mbox{\em It seems that F. deliberately avoids M.}} 
\end{array} \\
c. & 
\begin{array}[t]{llllll}
\mbox{dat} & \mbox{Frits} & \mbox{Marie} & \mbox{de laatste tijd} & \mbox{lijkt} 
& 
\mbox{te ontwijken.}  \\
\mbox{that}& \mbox{F.} &\mbox{M.} & \mbox{lately} & \mbox{seems} & \mbox{to 
avoid} \\
\multicolumn{6}{l}{\mbox{\em It seems lately as if  F. avoids M.}} \\
\multicolumn{6}{l}{\mbox{\em It seems as if  F. avoids M. lately}} 
\end{array}
\end{array}
\end{equation}

\noindent In (\ref{interaction}a) the adjunct modifies the entire verb cluster
(in the most likely reading).  However, we also find structurally similar
examples in which the adjunct modifies the governed verb (\ref{interaction}b).
Finally, there are examples that are ambiguous between a wide and narrow scope
reading (\ref{interaction}c).  We take it that the latter case is actually what
needs to be accounted for, i.e.  examples such as (\ref{interaction}a) and
(\ref{interaction}b) are cases in which there is a strong preference for a wide
and narrow scope reading, respectively, but we will remain silent about the
(semantic) factors determining such preferences.

Examples of sentences in which an adjuncts takes narrow scope are in fact quite 
frequent: 
\begin{equation}
\begin{array}[t]{ll@{\hspace{0cm}}l}
a. && \ldots\mbox{dat Frits het artikel morgen probeert af te hebben.} \\
b. && \ldots\mbox{dat van Basten de bal sierlijk in het doel probeerde te 
werken.} \\
\end{array} 
\end{equation}
Nevertheless, these cases are difficult to if one does not assume that adjuncts 
are in fact arguments of the verb they modify. An approach which assigns a 
polymorphic category to verbal adjuncts may get the word order facts right, but 
will not be able to account for the fact that a modifier may take scope over 
only part of the verb cluster it takes as argument. 
 
If one treats adjuncts as arguments, on the other hand, the preceding data can
be accounted for quite naturally.  Assuming the lexical entries for {\em lijken}
en {\em ontwijken} to be as in (\ref{lex}), example (\ref{interaction}c) has two
possible derivations ((\ref{wide}) and (\ref{narrow})).

\begin{equation}
\label{lex}
\begin{array}[t]{l}
lex(lijken,\var{Sign}) \con{:-} \\
\hspace{1cm} \var{Sign0} \Leftrightarrow  \con{vp}/\con{vp}, \\
\hspace{1cm} \att{add\_adjuncts}(\var{Sign0},\var{Sign1}), \\
\hspace{1cm} \att{division}(\var{Sign1},\var{Sign}).\\
lex(ontwijken,\var{Sign}) \con{:-} \\
\hspace{1cm} \var{Sign0} \Leftrightarrow  \con{np}\backslash\con{vp}, \\
\hspace{1cm} \att{add\_adjuncts}(\var{Sign0},\var{Sign} ).
\end{array}
\end{equation}

{\sc 
\begin{equation}
\label{wide}
\mbox{
\begin{tabular}[t]{rcccccc}
\ldots {\em Frits} & {\em Maria} & {\em de laatste tijd} & {\em lijkt}
                                                        & {\em te ontwijken}\\
        np      & np            & adj                   & vp/vp  &  
np$\backslash$vp\\
                                                &&&$\Downarrow$ \\
                                       &&& 
(np$\backslash$vp)$/$(np$\backslash$vp)\\
                                      &&&$\Downarrow$ \\
                       &&& 
(adj$\backslash$(np$\backslash$vp))$/$(np$\backslash$vp) \\
\cline{4-5}     &&& \multicolumn{1}{r}{adj$\backslash$(np$\backslash$vp)} \\
\cline{3-4} && \multicolumn{2}{c}{np$\backslash$vp}
\end{tabular}
}
\end{equation}

\begin{equation}
\label{narrow}
\mbox{
\begin{tabular}[t]{rcccccc}
\ldots {\em Frits} & {\em Maria} & {\em de laatste tijd} & {\em lijkt}
                                                        & {\em te ontwijken}\\
        np      & np            & adj                   & vp$/$vp  &  
np$\backslash$vp\\
                                        &&&$\Downarrow$ & $\Downarrow$\\
        &&& 
(adj$\backslash$(np$\backslash$vp))/(adj$\backslash$(np$\backslash$vp))
                         & adj$\backslash$(np$\backslash$vp)\\
\cline{4-5}     &&& \multicolumn{2}{c}{adj$\backslash$(np$\backslash$vp)} \\
\cline{3-4} && \multicolumn{2}{c}{np$\backslash$vp}
\end{tabular}
}
\end{equation}
}

\noindent Procedurally speaking, the rule that adds adjuncts can be applied
either to the VR verb (after division has taken place) or to the governed verb.
In the latter case, the adjunct is `inherited' by the matrix verb.  Assuming
that adjuncts take scope over the verbs introducing them, this accounts for the
ambiguity observed above.  In the first case, the adjunct is introduced by the
VR verb, and thus, the adjunct will take scope over this verb (and thus, over
the entire verb cluster).  In the second case, the adjunct is introduced by the
governed verb, over which it takes scope.  It does not take scope over the VR
verb, as this verb only inherits the adjunct as argument.

% \section{Clitic Placement}

%\bibliography{gb_references}
 
\section{Implementation}

The fragment discussed in the preceding sections has been fully implemented.\footnote{ in Sicstus Prolog.  Copies of the implementation are available upon request.}

The implemented grammar contains the syntactic rules rightward and leftward
application, and two construction specific rules for topicalization/{\sc
wh}-questions and relatives.  For the latter two constructions, it seems easiest to
use a syntactic rule, eventhough we could have chosen a lexicalist account, using
lexical rules.

The grammar contains lexical rules for division, gap-introduction, the addition 
of adjuncts, and verb second/first. 

Semantic representations are constructed, although we have not been overly 
concerned with this aspect, so that the semantics ignores many subtleties. It 
does account for the scope variation of adjuncts, however.

Lexical entries are computed by a component interfacing the grammar and the lexicon (a list of word forms and associated subcategorization properties and semantics). It applies morphology, builds the appropriate feature structures by combining syntactic and semantic information, adds default information, and applies lexical rules.

The grammar uses feature-structures, that are compiled into Prolog-terms for 
efficiency. For the same reason, lexical entries and syntactic rules are 
partially evaluated. The parser works in two stages. First, it constructs a set 
of possible derivation-trees on the basis of a context-free approximation of the 
actual grammar. Next, it tries to expand each of the context-free derivations 
into a full-fledged derivation on the basis of the actual grammar. Eventhough 
the performance of this parser is still not very impressive, it is in many cases 
substantially faster than using a shift-reduce parser with naive backtracking 
(which is the alternative method we have used). 

\addcontentsline{toc}{section}{\hspace{0.5cm}References}
\bibliography{gb_references}


%{\small
%\subsection{The Grammar}
%\begin{verbatim}
:- ensure_loaded(ubg_utils).
:- ensure_loaded(lexicon).
:- ensure_loaded(parser).
:- ensure_loaded(interface).
:- ensure_loaded(examples).

%%%%%%%%%%%%%%%%%%%%%%%%%%%%  The Grammar %%%%%%%%%%%%%%%%%%%%%%%%%%%%%%%%

%%%%%%%%%%%%%%%%%%%%%%%%%%%  headed grammar rules   %%%%%%%%%%%%%%%%%%%%%%
        
rule(ra, Val, [ Fun, Arg ], 1) :-               % rightward application
        Fun <=> Val/Arg,
        apply_principles(Val,Fun,Arg).
        
rule(la, Val, [ Arg, Fun ], 2) :-               % leftward application
        Fun <=> Arg\Val,
        apply_principles(Val,Fun,Arg).
        
rule(que, Que, [ Filler, Sent ], 2) :-          % topicalization, wh-questions
        Sent <=> s,
        Sent <=> vfirst,
        Sent:gap <=> Filler,            
        Que:gap  <=> empty,             % wh-clauses are extraction islands
        Que <=> s,
        Que <=> vsecond,
        Que:sem <=> Sent:sem,
        Filler:gap <=> empty,
        Filler:rel <=> empty.           % exclude rel. pronouns
        
rule(rel, RelS, [ RelPro, S ], 1) :-            % relative clause formation
        S <=> s,
        S <=> vfinal,
        S:gap <=> RelPro,
        RelS <=> n(_)\n(_),             
        RelS:gap <=> empty,             % relatives are extraction islands
        RelS:mor <=> RelPro:rel:rmor,   
        RelS:mor <=> RelS:arg:mor,
        RelS:sem <=> X^and(NSem,Ssem),
        RelS:arg:sem <=> X^NSem,
        S:sem <=> Ssem,
        RelPro:rel:index <=> X.
        
%%%%% grammar principles

apply_principles(Val,Fun,Arg) :-
        Fun:mor <=> Val:mor,
        Fun:sem <=> Val:sem,
        Val:phrase <=> Arg:rphrase,     
        Fun:phrase <=> Arg:phrase,      % * marie wil op verzuimen te bellen
        non_local(rel,Fun,Arg,Val),
        non_local(gap,Fun,Arg,Val).     % nb gap is not a list-valued feature
                                        % so, no multiple extractions
                                        
non_local(Feat,Dghtr1,Dghtr2,Mthr) :-   % percolation of non-local,
        Dghtr1:Feat <=> D1,             % `sign-valued', features
        Dghtr2:Feat <=> D2,
        Mthr:Feat <=> M,
        non_local(D1,D2,M).
        
:- block non_local(-,-,?).              % delay evaluation of disjunction 

non_local(Dghtr1,Dghtr2,Mthr) :-                
        ( nonvar(Dghtr1) ->                      
          ( Dghtr1 = empty -> Dghtr2 <=> Mthr
          ; Dghtr2 = empty,   Dghtr1 <=> Mthr
          )
        ; Dghtr2 = empty -> Dghtr1 <=> Mthr
        ; Dghtr1 = empty,   Dghtr2 <=> Mthr
        ).
        
%%%%%%%%%%%%%%%%%%%%%%%%%% templates %%%%%%%%%%%%%%%%%%%%%%%%%%%%%%%%%%%%%

%%%% syntax

temp(basic(Cat),X) :-
        X:cat <=> Cat,                  % distinguish basic and complex
        X:arg <=> nil,                  % properly (for delayed stuff)
        X:val <=> nil.
        
temp(np(Case),X) :-
        X <=> basic(np),
        % X:rphrase <=> yes,
        X:mor:case <=> Case.
        
temp(pn,X) :-
        X <=> np(_),
        X:mor:agr <=> sg(3).
        
temp(pronoun(Agr,Case),X) :-
        X <=> np(Case),
        X:mor:agr <=> Agr.

temp(n(Det),X) :-
        X <=> basic(n),
        X:mor:det <=> Det.
                
temp(adj(Agr,Det,Def),X) :-
        X <=> n(_)/n(_),
        X:mor <=> X:arg:mor,
        X:mor:det <=> Det,
        X:mor:def <=> Def,
        X:mor:agr <=> Agr.
                        
temp(det(Agr,Det,Def),X) :-
        X <=> np(_)/n(Det),
        X:mor:agr <=> Agr,
        X:arg:mor:agr <=> Agr,
        X:arg:mor:def <=> Def.
        
temp(s,X) :-
        X <=> basic(sent).
        
temp(iv,X) :-
        X <=> np(nom)\s,
        X <=> rphrasal_arg.
        
temp(tv,X) :-
        X <=> np(acc)\iv,
        X <=> rphrasal_arg.

temp(dtv1,X) :-                         % marie het boek geven
        X <=> np(acc)\tv,
        X <=> rphrasal_arg.
temp(dtv2,X) :-                         % het boek aan marie geven
        X <=> pp(aan)\tv,
        X <=> rphrasal_arg.
        
temp(tvp(PrepFrm),X) :-
        X <=> pp(PrepFrm)\iv,
        X <=> rphrasal_arg.
        
temp(tv_pref(PrefFrm),X) :-
        X <=> prefix(PrefFrm)\tv.

temp(modal,X) :-
        ( X <=> iv/iv
        ; X <=> iv\iv,          % modal inversion
          X <=> finite,         % for finite modals only
          X:arg <=> no_ipp
        ),
        X:arg <=> infinitive.
        
temp(s_raising,X) :-
        X <=> iv/iv,
        X:arg <=> no_om.                % te_infinitive without om
        
temp(perc_verb,X) :-
        X <=> tv/iv,
        X:arg <=> infinitive.
        
temp(perf_aux(AuxFrm),X) :-
        ( X <=> iv/iv
        ; X <=> iv\iv,          % account for participle inversion
          X:arg <=> no_ipp
        ),
        X:arg <=> participle(AuxFrm).

temp(extra_verb(Om),X) :-
        X <=> iv/iv,
        X:arg <=> te_infinitive,
        X:arg <=> Om,
        X <=> rphrasal_arg.
        
temp(trans_extra_verb(Om),X) :-
        X <=> tv/iv,
        X:arg <=> te_infinitive,
        X:arg <=> Om,
        X <=> rphrasal_arg.
        
temp(s_compl_verb,X) :-
        X <=> iv/s,
        X:arg <=> subordinate,
        X <=> rphrasal_arg.
        
         
temp(pp(PrepFrm),X) :-
        X <=> basic(pp),
        X:mor:form <=> PrepFrm.
                
temp(prep1(PrepFrm),X) :-
        X <=> pp(PrepFrm)/np(acc).
        
temp(prep2,X) :-
        X <=> (n(_)\n(_))/np(acc),
        X:mor <=> X:val:arg:mor.
                        
temp(rel_pro(Agr,Det),X) :-
        X <=> np(_),
        X:mor:agr <=> Agr,
        X:rel:rmor:agr <=> Agr,
        X:rel:rmor:det <=> Det.
        
temp(rel_det,X) :-
        X <=> det(_Agr,_Det,def).
        
temp(prefix(Prefix),X) :-
        X <=> basic(pref),
        X <=> nonphrasal,
        X:mor:form <=> Prefix.

temp(adjunct,X) :-
        X <=> basic(adj).
        
%%%% semantics  (mixing fvt and prolog terms ...)

temp(pn_sem(Name),S) :-
        S:sem <=> (Name^Prop)^Prop.
                
temp(n_sem(Name),S) :-
        S:sem <=> X^Prop,
        Prop =.. [Name,X].
                        
temp(adj_sem(Name),S) :-
        S:sem <=> Var^and(NSem,Asem),
        S:arg:sem <=> Var^NSem,
        Asem =.. [Name,Var].
                
temp(det_sem(Name),S) :-
        S:sem <=> (X^Scope)^Formula,
        S:arg:sem <=> X^Restriction,
        Formula =.. [Name,X,Restriction,Scope].
                
temp(iv_sem(Name),S) :-
        S:sem <=> Sem,
        S:arg:sem <=> (X^Pred)^Sem,
        Pred =.. [Name,X].
        
temp(tv_sem(Name),S) :-
        S:sem <=> Sem,
        S:arg:sem <=> (Y^Pred)^Sem0,
        S:val:arg:sem <=> (X^Sem0)^Sem,
        Pred =.. [Name,X,Y].

temp(dtv1_sem(Name),S) :-
        S:sem <=> Sem,
        S:arg:sem <=> (Z^Pred)^Sem0,
        S:val:arg:sem <=> (Y^Sem0)^Sem1,
        S:val:val:arg:sem <=> (X^Sem1)^Sem,
        Pred =.. [Name,X,Y,Z].
        
temp(dtv2_sem(Name),S) :-
        S:sem <=> Sem,
        S:arg:sem <=> (Z^Pred)^Sem0,
        S:val:arg:sem <=> (Y^Sem0)^Sem1,
        S:val:val:arg:sem <=> (X^Sem1)^Sem,
        Pred =.. [Name,X,Z,Y].
        
temp(tv_pref_sem(Name),S) :-
        S:sem <=> Sem,
        S:val:arg:sem <=> (Y^Pred)^Sem0,
        S:val:val:arg:sem <=> (X^Sem0)^Sem,
        Pred =.. [Name,X,Y].
        
temp(modal_sem(Name),S) :-
        S:sem <=> Sem,
        S:arg:arg:sem <=> (X^VPsem)^VPsem,
        S:arg:sem <=> ArgSem,
        Form =.. [Name,X,ArgSem],
        S:val:arg:sem <=> (X^Form)^Sem.
        
temp(s_raising_sem(Name),S) :-
        S:sem <=> Sem,
        Sem =.. [Name,Ssem],
        S:arg:arg:sem <=> (X^VPsem)^VPsem,
        S:arg:sem <=> ArgSem,
        S:val:arg:sem <=> (X^ArgSem)^Ssem.
        
temp(perc_verb_sem(Name),S) :-
        S:sem <=> SubjSem,
        S:arg:arg:sem <=> (X^VPsem)^VPsem,
        S:arg:sem <=> ArgSem,
        Sem =.. [Name,Y,ArgSem],
        S:val:arg:sem <=> (X^Sem)^ObjSem,               % object wide scope 
        S:val:val:arg:sem <=> (Y^ObjSem)^SubjSem.       % wrt perc verb ...

temp(trans_extra_verb_sem(Name),S) :-
        S:sem <=> SubjSem,
        S:arg:arg:sem <=> (X^VPsem)^VPsem,
        S:arg:sem <=> ArgSem,
        Sem =.. [Name,Y,X,ArgSem],
        S:val:arg:sem <=> (X^Sem)^ObjSem,               % object wide scope 
        S:val:val:arg:sem <=> (Y^ObjSem)^SubjSem.       % wrt perc verb ...

temp(s_compl_sem(Name),S) :-
        S:sem <=> SubjSem,
        S:arg:sem <=> ComplSem,
        Sem =.. [Name,X,ComplSem],
        S:val:arg:sem <=> (X^Sem)^SubjSem.
        
temp(prep1_sem,S) :-
        S:sem <=> S:arg:sem.
        
temp(prep2_sem(Name),S)  :-
        S:sem <=> X^and(Nsem,Formula),
        S:arg:sem <=> (Y^Scope)^Formula,
        S:val:arg:sem <=> X^Nsem,
        Scope =..[Name,Y,X].
        
temp(rel_pro_sem,S) :-
        S:sem <=> (Index^Prop)^Prop,
        S:rel:index <=> Index.  
                
temp(rel_det_sem,S) :-
        S:sem <=> (X^Scope)^unique(X,and(Restriction,poss(Index,X)),Scope),
        S:arg:sem <=> X^Restriction,    
        S:rel:index <=> Index.
        
temp(adjunct_sem(Name),S) :-
        S:sem <=> X^Form,
        Form =..[Name,X].
        
%%%%% verbal subcategorization types (combine syntax and semantics) %%%%%%%%

temp(subcat(iv,Sem,nil),Sign) :-
        Sign <=> iv,
        Sign <=> iv_sem(Sem).
temp(subcat(tv,Sem,nil),Sign) :-
        Sign <=> tv,
        Sign <=> tv_sem(Sem).
temp(subcat(dtv,Sem,nil),Sign) :-
        Sign <=> dtv1,
        Sign <=> dtv1_sem(Sem).
temp(subcat(dtv,Sem,nil),Sign) :-
        Sign <=> dtv2,
        Sign <=> dtv2_sem(Sem).
temp(subcat(tvp(PrpFrm),Sem,nil),Sign) :-
        Sign <=> tvp(PrpFrm),
        Sign <=> tv_sem(Sem).
temp(subcat(tv_pref(PrefFrm),Sem,nil),Sign) :-
        Sign <=> tv_pref(PrefFrm),
        Sign <=> tv_pref_sem(Sem).
temp(subcat(modal,Sem,vr),Sign) :-
        Sign <=> modal,
        Sign <=> modal_sem(Sem).
temp(subcat(modal_te,Sem,vr),Sign) :-
        Sign <=> s_raising,
        Sign <=> modal_sem(Sem).
temp(subcat(s_raising,Sem,vr),Sign) :-
        Sign <=> s_raising,
        Sign <=> s_raising_sem(Sem).
temp(subcat(perc_verb,Sem,vr),Sign) :-
        Sign <=> perc_verb,
        Sign <=> perc_verb_sem(Sem).
temp(subcat(perf_aux(AuxFrm),Sem,vr),Sign) :-
        Sign <=> perf_aux(AuxFrm),
        Sign <=> s_raising_sem(Sem).
temp(subcat(extra_verb(Om),Sem,nil),Sign) :-
        Sign <=> extra_verb(Om),
        Sign <=> modal_sem(Sem).
temp(subcat(trans_extra_verb(Om),Sem,nil),Sign) :-
        Sign <=> trans_extra_verb(Om),
        Sign <=> trans_extra_verb_sem(Sem).
temp(subcat(p_extra_verb,Sem,p_extra),Sign) :-
        Sign <=> extra_verb(no_om),
        Sign <=> modal_sem(Sem).        
temp(subcat(s_compl_verb,Sem,nil),Sign) :-
        Sign <=> s_compl_verb,
        Sign <=> s_compl_sem(Sem).
        
        
%%%% morphological stuff

temp(finite,X) :-
        X:mor:form <=> fin(_).
temp(vfirst,X) :-
        X:mor:form <=> fin(1).
temp(vsecond,X) :-
        X:mor:form <=> fin(2).
temp(vfinal,X) :-
        X:mor:form <=> fin(3).
temp(subordinate,X) :-
        X:mor:form <=> fin(sub).
        
temp(infinitive,X) :-                           % needed for ipp effect 
        X:mor:form <=> nonfin(inf,no,_,_).
temp(participle(Frm),X) :-
        X:mor:form <=> nonfin(no,prt,Frm,_).
temp(inf_or_part(Frm),X) :-                     % infinitival form of VR verb
        X:mor:form <=> nonfin(_,_,Frm,ipp).     % must unify with Prt as well
temp(no_ipp,X) :-                               
        X:mor:form <=> nonfin(_,_,_,no_ipp).    
        
temp(te_infinitive,X) :-
        X:mor:form <=> te(_).
temp(om,X) :-
        X:mor:form <=> te(om).
temp(no_om,X) :-
        X:mor:form <=> te(no_om).

        
temp(verbal_morphology(fin(Agr)),X) :-
        X <=> finite,
        sv_agreement(Agr,X).            % ensure X is instantiated
temp(verbal_morphology(non_fin(Frm)),X) :-
        X <=> Frm.
        
temp(singular,X) :-
        X:mor:agr <=> sg(3).
temp(plural,X) :-
        X:mor:agr <=> pl(3).

temp(phrasal,X) :-
        X:phrase <=> yes.
temp(nonphrasal,X) :-
        X:phrase <=> no.
temp(rphrasal_arg,X) :-
        X:arg:rphrase <=> yes.


%%%%%%%%%%% recursive constraints and lexical rules %%%%%%%%%%%%%%%%%%%%%%
                
sv_agreement(Agr,X) :-
        X:arg <=> Arg,
        sv_agreement(Agr,X,Arg).
        
:- block sv_agreement(?,?,-).

sv_agreement(Agr,X,_) :-                % agreement as a recursive constraint
        X <=> np(nom)\s, !,             
        X:arg:mor:agr <=> Agr.
sv_agreement(Agr,X,_) :-
        X:val <=> Val,
        sv_agreement(Agr,Val).
        
        
verb_position(In,Out) :-                % do nothing 
        In <=> Out,
        apply_default(Out <=> vfinal).  % and if Out is finite, it is vfinal
verb_position(In,Out) :-                % apply verb_first lexical rule
        In <=> finite,                  
        In:mor <=> Out:mor,             
        In:gap <=> Out:gap,
        In:sem <=> Out:sem,
        Out <=> vfirst,
        S <=> s,
        verb_first(In,S,Out,_).
        
verb_first(In,Sofar,Out,Dir) :- 
        Out:arg <=> Arg,
        verb_first(In,Sofar,Out,Dir,Arg).
        
:- block verb_first(?,?,?,?,-).

verb_first(In,Mid,Out,_,_) :-           % verb first as a lexical rule   
        In <=> Mid,                     % Out is the verb first version of In
        Out <=> s.
verb_first(In,Sofar,Out,left,_) :-
        Out <=> Val/Arg0,
        Sofar0 <=> Arg1\Sofar,
        unify_except(Arg0,Arg1,[rphrase]),      % hack, needed for
        verb_first(In,Sofar0,Val,_).            % belt jan marie[+rp] op[-p]
verb_first(In,Mid,Out,right,_) :-       % handle vfinal verbs selecting 
        In <=> In0/Arg0,                % arguments to their right
        Out <=> Out0/Arg1,
        unify_except(Arg0,Arg1,[rphrase]),
        verb_first(In0,Mid,Out0,right).
                
gap_intro_lr(In,Out):-                  % gap introduction (aka push_to_slash)
        In <=> Out.                     % as a (optional) lexical rule
gap_intro_lr(In,Out) :-                 
        In:mor <=> Out:mor,
        In:sem <=> Out:sem,
        Out:gap <=> Gap,
        gap_intro(In,Out,Gap).          
        
gap_intro(In,Out,Gap) :-
        Out:arg <=> OutArg,
        In:arg <=> InArg,
        Gap:cat <=> _Any,               % used to trigger non_local/3
        gap_intro(In,Out,Gap,InArg,OutArg).
        
:- block gap_intro(?,?,?,-,-).          % block only if both in and out Arg
                                        % are unknown (to avoid deadlock if
gap_intro(In,Out,Gap,_,_) :-            % a verb is selected by a vraiser)      
        In:arg <=> Arg,                 
        unify_except(Arg,Gap,[phrase]), % remove phrase feature from gap
        In:val <=> Out.                 % to get (partial) vp topicalization
gap_intro(In,Out,Gap,_,_) :-
        In:arg <=> Out:arg,
        In:dir <=> Out:dir,
        In:cat <=> Out:cat,
        In:val <=> InVal,
        Out:val <=> OutVal,
        gap_intro(InVal,OutVal,Gap).
        
verb_raise(In,Out,nil) :-               % not a v-raising verb
        In <=> Out.
verb_raise(In,Out,Type) :-
        In:mor <=> Out:mor,
        In:sem <=> Out:sem,
        In:gap <=> Out:gap,
        In:arg:mor <=> Out:arg:mor,
        ( Type = vr,            
          Out:arg <=> nonphrasal
        ; Type = p_extra,       
          Out:arg:rphrase <=> yes
        ),
        ( In:dir <=> left -> Out <=> rphrasal_arg       % inversion cases
        ; true
        ),
        division(In,Out).
        
division(In,Out) :-
        Out:val:arg <=> Arg,
        division(In,Out,Arg).
        
:- block division(?,?,-).

division(In,Out,_) :-
        In <=> Out.
division(In,Out,_) :-                           % four versions in one
        Out:val:val <=> Mid:val,
        Out:arg:val <=> Mid:arg,
        Out:dir <=> Mid:dir,
        Out:arg:arg <=> Out:val:arg,
        Out:arg:dir <=> Out:val:dir,
        Out:cat <=> nil,
        Out:val:cat <=> nil,
        Out:arg:cat <=> nil,
        Out:arg:dir <=> Dir1,
        Out:dir <=> Dir2,
        Out:arg:arg <=> Arg,
        check_harmonic(Dir1,Dir2,Arg),
        Out:arg:sem <=> Out:arg:val:sem,        % do raising verbs properly
        division(In,Mid).
        
:- block check_harmonic(-,?,?), check_harmonic(?,-,?).

check_harmonic(Dir1,Dir2,Arg) :-                % harmonic division implies
        ( Dir1 = Dir2 -> Arg <=> phrasal        % argument is a non VR verb
        ; true                                  % (avoids sp. ambiguities)
        ).

add_adjuncts_lr(In,Out) :-
        In:gap <=> Out:gap,
        In:mor <=> Out:mor,
        In:sem <=> InSem,
        Out:sem <=> OutSem,                     
        add_adjuncts(In,Out,InSem,OutSem).
        
add_adjuncts(In,Out,InSem,OutSem) :-
        Out:arg <=> Arg,
        add_adjuncts(In,Out,InSem,OutSem,Arg).
        
:- block add_adjuncts(?,?,?,?,-).
        
add_adjuncts(In,Out,Sem,Sem,_) :-
        In <=> Out,
        In <=> s.
add_adjuncts(In,Out,InSem,OutSem,_) :-          
        Out <=> Adj\OutVal,                     
        Adj <=> adjunct,                        % adjuncts take scope over
        Out <=> rphrasal_arg,                   % arguments, with the first 
        Adj:sem <=> InSem^ResultSem,            % adjunct taking widest scope
        add_adjuncts(In,OutVal,ResultSem,OutSem).       
add_adjuncts(In,Out,InSem,OutSem,_) :-
        Out:arg <=> In:arg,
        Out:dir <=> In:dir,
        Out:val <=> OutVal,
        In:val <=> InVal,
        add_adjuncts(InVal,OutVal,InSem,OutSem).
        
%%%%%%%%%%%%%%%%%%%%%%%%% lexicon interface %%%%%%%%%%%%%%%%%%%%%%%%%%%

%%%% apply lexical defaults

lexic(Word,Sign)        :-
        lex0(Word,Sign),
        apply_default(Sign:gap <=> empty),
        apply_default(Sign:rel <=> empty).

%%%% word class specific interface:
%%%% handle morphology, combine syntax and semantics

lex0(Word,Sign) :-
        proper_name(Word,Sem),
        Sign <=> pn,
        Sign <=> pn_sem(Sem).
        
lex0(Word,Sign) :-
        pronoun(Word,Agr,Case,Sem),
        Sign <=> pronoun(Agr,Case),
        Sign <=> pn_sem(Sem).
        
lex0(Word,Sign) :-
        ( noun([Word,_],Det,Sem),
          Sign <=> singular
        ; noun([_,Word],Det,Sem),
          Sign <=> plural
        ),
        Sign <=> n(Det),
        Sign <=> n_sem(Sem).
        
lex0(Word,Sign) :-
        ( adj([Word,_],Sem),
          Sign <=> adj(sg(3),het,indef)
        ; adj([_,Word],Sem),
          ( Sign <=> adj(pl(3),_,_)
          ; Sign <=> adj(sg(3),de,_)
          ; Sign <=> adj(sg(3),het,def)
          )
        ),
        Sign <=> adj_sem(Sem).
        
lex0(Word,Sign) :-
        ( det(Word,Agr,Det,Def,Sem),
          Sign <=> det(Agr,Det,Def)
        ; det([Word,_],Sem),
          Sign <=> det(sg(3),het,indef)
        ; det([_,Word],Sem),
          ( Sign <=> det(pl(3),_,_)
          ; Sign <=> det(sg(3),de,_)
          )
        ),
        Sign <=> det_sem(Sem).
        
lex0(Word,Sign) :-
        verb0(Word,Syn,Sem,Morphology),
        Sign0 <=> subcat(Syn,Sem,VerbType),
        Sign0 <=> verbal_morphology(Morphology),
        verb_raise(Sign0,Sign1,VerbType),       
        add_adjuncts_lr(Sign1,Sign2),
        gap_intro_lr(Sign2,Sign3),
        verb_position(Sign3,Sign).
        
verb0(Word,Syn,Sem,fin(sg1)) :-
        verb([Word|_],Syn,_,Sem).
verb0(Word,Syn,Sem,fin(sg(_))) :-               % 2nd or 3rd person singular
        verb([_,Word,_,_],Syn,_,Sem).
verb0(Word,Syn,Sem,fin(sg(2))) :-               % 2nd person (hebt, bent)
        verb([_,Word,_,_,_],Syn,_,Sem).
verb0(Word,Syn,Sem,fin(sg(3))) :-               % 3nd person (heeft, is)
        verb([_,_,Word,_,_],Syn,_,Sem).
verb0(Word,Syn,Sem,fin(pl(_))) :- 
        ( verb([_,_,Word,_],Syn,_,Sem)
        ; verb([_,_,_,Word,_],Syn,_,Sem)
        ).
verb0(Word,Syn,Sem,Morphology) :-       
        ( verb([_,_,Inf,Prt],Syn,PartFrm,Sem)   
        ; verb([_,_,_,Inf,Prt],Syn,PartFrm,Sem)
        ), 
        (  Inf = Prt                                    % ipp verbs
        -> Word = Inf,
           Morphology = non_fin(inf_or_part(PartFrm))   
        ;  Word = Inf,                                  % others
           Morphology = non_fin(infinitive)
        ;  Word = Prt,
           Morphology = non_fin(participle(PartFrm))
        ).
        
verb0([te,Word],Syn,Sem,non_fin(te_infinitive)) :- 
        ( verb([_,_,Word,_],Syn,_,Sem)
        ; verb([_,_,_,Word,_],Syn,_,Sem)
        ).

        
lex0(Word,Sign) :-
        preposition(Word,Sem),
        ( Sign <=> prep1(Word),
          Sign <=> prep1_sem
        ; Sign <=> prep2,
          Sign <=> prep2_sem(Sem)
        ). 
        
lex0(Word,Sign) :-
        rel_pronoun(Word,Num,Det),
        Sign <=> rel_pro(Num,Det),
        Sign <=> rel_pro_sem.

lex0(Word,Sign) :-
        rel_det(Word),
        Sign <=> rel_det,
        Sign <=> rel_det_sem.

lex0(Word,Sign) :-
        prefix(Word),
        Sign <=> prefix(Word).

lex0(om,Sign) :-
        Sign <=> iv/iv,
        Sign:arg <=> no_om,
        Sign:val <=> om,
        Sign:arg:arg <=> Sign:val:arg,          % `subject raising' semantics
        Sign:arg:sem <=> Sign:sem.              
        
lex0(dat,Sign) :-
        Sign <=> s/s,
        Sign:arg <=> vfinal,
        Sign:val <=> subordinate,
        Sign:arg:sem <=> Sign:sem.              
        
lex0(Word,Sign) :-
        adjunct(Word,Sem),                      
        Sign <=> adjunct,
        Sign <=> adjunct_sem(Sem).      
        
%%%%%%%%%%%%%%%% grammar compilation (for efficiency only) %%%%%%%%%%%%%%%%
        
:- compile_lexicon.

:- compile_rules.
\end{verbatim}


%\subsection{The Lexicon}
%\begin{verbatim}
%%%%%%%%%%%%%%%%%%%%%%% lexicon proper %%%%%%%%%%%%%%%%%%%%%%%%%%%%%%%%%%

proper_name(jan,jan).
proper_name(marie,marie).
proper_name(wie,'??').
proper_name(groningen,groningen).
proper_name(water,water).

pronoun(ik,sg1,nom,pro(sg,1)).
pronoun(mij,sg1,acc,pro(sg,1)).
pronoun(me,sg1,acc,pro(sg,1)).
pronoun(jij,sg(2),nom,pro(sg,2)).
pronoun(jou,sg(2),acc,pro(sg,2)).
pronoun(je,sg(2),acc,pro(sg,2)).
pronoun(hij,sg(3),nom,pro(sg,3,m)).
pronoun(hem,sg(3),acc,pro(sg,3,m)).
pronoun(zij,sg(3),nom,pro(sg,3,f)).
pronoun(ze,sg(3),nom,pro(sg,3,f)).
pronoun(haar,sg(3),acc,pro(sg,3,f)).
pronoun(wij,pl(1),nom,pro(pl,1)).
pronoun(we,pl(1),nom,pro(pl,1)).
pronoun(ons,pl(1),acc,pro(pl,1)).

noun([man,mannen],de,man).
noun([broer,broers],de,broer).
noun([vrouw,vrouwen],de,vrouw).
noun([zuster,zusters],de,zuster).
noun([meisje,meisjes],het,meisje).
noun([jongen,jongens],de,jongen).

noun([boek,boeken],het,boek).

adj([aardig,aardige],aardig).
adj([slim,slimme],slim).

det(de,_,de,_,unique).
det(het,sg(3),het,def,unique).
det(een,sg(3),_,indef,exist).
det(alle,pl(3),_,_,all).
det([welk,welke],which).

verb([slaap,slaapt,slapen,geslapen],iv,heb,slapen).
verb([val,valt,vallen,gevallen],iv,zijn,vallen).
verb([zing,zingt,zingen,gezongen],iv,zijn,zingen).     
  
verb([kus,kust,kussen,gekust],tv,heb,kussen).
verb([ken,kent,kennen,gekend],tv,heb,kennen).
verb([drink,drinkt,drinken,gedronken],tv,heb,drinken).
verb([lees,leest,lezen,gelezen],tv,heb,lezen).

verb([geef,geeft,geven,gegeven],dtv,heb,geven).

verb([houd,houdt,houden,gehouden],tvp(van),heb,houden_van).
verb([bel,belt,bellen,gebeld],tv_pref(op),heb,op_bellen).

verb([wil,wil,willen,willen],modal,heb,willen).         % ipp
verb([moet,moet,moeten,moeten],modal,heb,moeten).       

verb([schijn,schijnt,schijnen,xxxxx],s_raising,xxx,schijnen).

verb([probeer,probeert,proberen,proberen],modal_te,heb,proberen).

verb([zie,ziet,zien,zien],perc_verb,heb,zien).
verb([hoor,hoort,horen,gehoord],perc_verb,heb,horen).


verb([verklaar,verklaart,verklaren,verklaard],extra_verb(no_om),heb,verklaren).
verb([zeg,zegt,zeggen,gezegd],extra_verb(no_om),heb,zeggen).

verb([vraag,vraagt,vragen,gevraagd],trans_extra_verb(_),heb,vragen).

verb([probeer,probeert,proberen,geprobeerd],p_extra_verb,heb,proberen).
verb([verzuim,verzuimt,verzuimen,verzuimd],p_extra_verb,heb,verzuimen).

verb([heb,hebt,heeft,hebben,xxxxx],perf_aux(heb),xxx,perf).
verb([ben,bent,is,zijn,xxxxx],perf_aux(zijn),xxx,perf).

verb([beweer,beweert,beweren,beweerd],s_compl_verb,heb,beweren).
verb([denk,denkt,denken,gedacht],s_compl_verb,heb,denken).

preposition(van,van).
preposition(uit,uit).
preposition(aan,aan).

prefix(op).

rel_pronoun(dat,sg(3),het).             
rel_pronoun(die,sg(3),de).
rel_pronoun(die,pl(3),_).

rel_pronoun(wie,sg(3),de).
rel_pronoun(wie,pl(3),_).

rel_det(wiens).

adjunct(nooit,nooit).
adjunct(altijd,altijd).
adjunct(mogelijk,mogelijk).
\end{verbatim}


%\subsection{Examples}
%\begin{verbatim}
%%% use examples/0 to view examples, parse_ex/1 to parse examples

examples :-
        write('no analyses sentence'),
        format_example(1).
        
format_example(N) :-
        ( example(N,Ana,Sentence) ->
          format('~n~d~3+~d~8+',[N,Ana]),
          pprint_sent(Sentence),
          N1 is N + 1,
          format_example(N1)
        ; true
        ).

pprint_sent([W]) :- format('~a.',[W]).
pprint_sent([W|R]) :- format('~a ',[W]), pprint_sent(R).
        
parse_ex(N) :-
        example(N,_,Sentence),
        nl, pprint_sent(Sentence), nl,
        parse(Sentence).

%%%%%%%%%%%%%%%%%%%%%%%%%% example set %%%%%%%%%%%%%%%%%%%%%%%%%%%%%%%%%%%%%%%

%% example(No,Analyses,Sentence).

%% basics (main and subordinate clauses, relatives, questions)

example(1, 1, [jan,slaapt]).
example(2, 1, [de,mannen,uit,groningen,slapen]).
example(3, 2, [jan,kust,marie]).
example(4, 2, [de,man,die,slaapt,kust,marie]).
example(5, 1, [de,man,van,wie,marie,houdt,slaapt]).
example(6, 1, [de,vrouw,wiens,broer,ik,ken,slaapt]).
example(7, 2, [wie,denk,ik,dat,marie,kust]).

%% vraising, extraposition, and partial extraposition
%% subordinate clauses (probeert is ambiguous)

example(8, 1, [ik,denk,dat,hij,marie,wil,kussen]).
example(9, 0, [ik,denk,dat,hij,wil,marie,kussen]).
example(10, 1, [ik,denk,dat,hij,jan,marie,ziet,kussen]).

example(11, 1, [ik,denk,dat,hij,marie,vraagt,het,boek,te,lezen]).
example(12, 0, [ik,denk,dat,hij,marie,het,boek,vraagt,te,lezen]).
example(13, 1, [ik,denk,dat,hij,verzuimt,marie,te,kussen]).
example(14, 1, [ik,denk,dat,hij,marie,verzuimt,te,kussen]).
example(15, 1, [ik,denk,dat,hij,probeert,marie,te,kussen]).
example(16, 2, [ik,denk,dat,hij,marie,probeert,te,kussen]).

%% phrasal and subphrasal constituents

example(17, 1, [ik,denk,dat,hij,marie,heeft,willen,kussen]).
example(18, 1, [ik,denk,dat,hij,marie,schijnt,op,te,bellen]).
example(19, 1, [ik,denk,dat,hij,marie,heeft,willen,op,bellen]).
example(20, 1, [ik,denk,dat,hij,marie,heeft,op,willen,bellen]).
example(21, 1, [ik,denk,dat,hij,marie,op,heeft,willen,bellen]).
example(22, 1, [ik,denk,dat,hij,marie,heeft,gevraagd,het,boek,te,lezen]).
example(23, 1, [ik,denk,dat,hij,marie,heeft,geprobeerd,te,kussen]).
example(24, 1, [ik,denk,dat,hij,marie,heeft,proberen,te,kussen]).

%% participle inversion 

example(25, 1, [ik,denk,dat,hij,marie,heeft,gekust]).
example(26, 1, [ik,denk,dat,hij,marie,gekust,heeft]).
example(27, 1, [ik,denk,dat,hij,van,marie,gehouden,moet,hebben]).
example(28, 0, [ik,denk,dat,hij,van,marie,moet,gehouden,hebben]).

%% oke with extraposition, but not with vr verbs

example(29, 1, [ik,denk,dat,jan,hem,gevraagd,heeft,marie,op,te,bellen]).
example(30, 0, [ik,denk,dat,hij,marie,willen,heeft,op,bellen]).

%% modal inversion, only with finite modals, complement must be -vr

example(31, 1, [ik,denk,dat,hij,marie,kussen,wil]).
example(32, 0, [ik,denk,dat,hij,marie,kussen,heeft,willen]).
example(33, 1, [ik,denk,dat,jan,hem,vragen,wil,marie,te,kussen]).
example(34, 0, [ik,denk,dat,hij,marie,hebben,wil,gekust]).

%% the scope of adjuncts

example(35, 2, [ik,denk,dat,jan,marie,mogelijk,kussen,wil]).
example(36, 2, [ik,denk,dat,jan,altijd,een,meisje,kussen,wil]).
example(37, 1, [ik,denk,dat,altijd,een,jongen,haar,kussen,wil]).

%% adjunct topicalization (spurious ambiguities arising from interaction of
%% extraction-lr, add-adjuncts rule, and division)

example(38, 7, [nooit,hoor,ik,marie,zingen]).

%% vp topicalization (spurious ambiguities arising from interaction of 
%% v1/v2 and inversion)

example(39, 2, [kussen,wil,jan,marie]).
example(40, 2, [marie,kussen,wil,jan]).
example(41, 1, [willen,kussen,heeft,jan,marie]).
example(42, 0, [willen,heeft,jan,marie,kussen]).
\end{verbatim}


%\subsection{Handling Feature Unification Constraints}
%\begin{verbatim}
:- op(500,xfy,\).
:- op(600, xfx, <=> ).


%%%%%%%%%%%%%%%%%%%%% Feature Unification %%%%%%%%%%%%%%%%%%%%%%%%%%%%%%%%

fvt_unify(A,A).

%%%% fvt unification macros

X <=> Y:- 
        denotes(X, Term), 
        denotes(Y, Term).

denotes(Var, FVT) :- 
        var(Var), !, 
        Var = FVT.
denotes(A/B,sign(nil,A1,right,B1,_Sem,_Mor,_Gap,_Rel,_Phrase,_Rphrase)) :-
        !,
        denotes(A,A1),
        denotes(B,B1).
denotes(B\A,sign(nil,A1,left,B1,_Sem,_Mor,_Gap,_Rel,_Phrase,_Rphrase)) :-
        !,
        denotes(A,A1),
        denotes(B,B1).
denotes(Dag:Path, Value):-
        !,
        pathval(Dag, Path, Value).       
denotes(Temp,FVT) :- 
        temp(Temp,FVT).
denotes(Any, Any) :- 
        \+ temp(Any,_).
        
pathval(sign(Cat,_Val,_Dir,_Arg,_Sem,_Mor,_Phrase,_Rphrase,_Rel,_Gap),cat,Cat).
pathval(sign(_Cat,Val,_Dir,_Arg,_Sem,_Mor,_Phrase,_Rphrase,_Rel,_Gap),val,Val).
pathval(sign(_Cat,_Val,Dir,_Arg,_Sem,_Mor,_Phrase,_Rphrase,_Rel,_Gap),dir,Dir).
pathval(sign(_Cat,_Val,_Dir,Arg,_Sem,_Mor,_Phrase,_Rphrase,_Rel,_Gap),arg,Arg).
pathval(sign(_Cat,_Val,_Dir,_Arg,Sem,_Mor,_Phrase,_Rphrase,_Rel,_Gap),sem,Sem).
pathval(sign(_Cat,_Val,_Dir,_Arg,_Sem,Mor,_Phrase,_Rphrase,_Rel,_Gap),mor,Mor).
pathval(sign(_Cat,_Val,_Dir,_Arg,_Sem,_Mor,Phrase,_Rphrase,_Rel,_Gap),phrase,Phrase).
pathval(sign(_Cat,_Val,_Dir,_Arg,_Sem,_Mor,_Phrase,Rphrase,_Rel,_Gap),rphrase,Rphrase).
pathval(sign(_Cat,_Val,_Dir,_Arg,_Sem,_Mor,_Phrase,_Rphrase,Rel,_Gap),rel,Rel).
pathval(sign(_Cat,_Val,_Dir,_Arg,_Sem,_Mor,_Phrase,_Rphrase,_Rel,Gap),gap,Gap).


pathval(mor(Case,_Form,_Agr,_Det,_Def),case,Case).
pathval(mor(_Case,Form,_Agr,_Det,_Def),form,Form).
pathval(mor(_Case,_Form,Agr,_Det,_Def),agr,Agr).
pathval(mor(_Case,_Form,_Agr,Det,_Def),det,Det).
pathval(mor(_Case,_Form,_Agr,_Det,Def),def,Def).

pathval(rel(Mor,_Index),rmor,Mor).
pathval(rel(_Mor,Index),index,Index).

pathval(sign(_Cat,Val,_Dir,_Arg,_Sem,_Mor,_Phrase,_Rphrase,_Rel,_Gap),val:Path,Value) :-
        pathval(Val,Path,Value).
pathval(sign(_Cat,_Val,_Dir,Arg,_Sem,_Mor,_Phrase,_Rphrase,_Rel,_Gap),arg:Path,Value) :-
        pathval(Arg,Path,Value).
pathval(sign(_Cat,_Val,_Dir,_Arg,_Sem,Mor,_Phrase,_Rphrase,_Rel,_Gap),mor:Path,Value) :-
        pathval(Mor,Path,Value).
pathval(sign(_Cat,_Val,_Dir,_Arg,_Sem,_Mor,_Phrase,_Rphrase,Rel,_Gap),rel:Path,Value) :-
        pathval(Rel,Path,Value).
pathval(sign(_Cat,_Val,_Dir,_Arg,_Sem,_Mor,_Phrase,_Rphrase,_Rel,Gap),gap:Path,Value) :-
        pathval(Gap,Path,Value).

pathval(rel(Mor,_Index),rmor:Path,Value) :-
        pathval(Mor,Path,Value).
        
%%%%%%%%%%%%%%%%%%%%%%%%%% defaulty stuff %%%%%%%%%%%%%%%%%%%%%%%%%%%%%%%

%% add hoc solution for this grammar only

unify_except(sign(Cat,Val,Dir,Arg,Sem,Mor,_,RP,Rel,Gap),
             sign(Cat,Val,Dir,Arg,Sem,Mor,_,RP,Rel,Gap),[phrase]).
unify_except(sign(Cat,Val,Dir,Arg,Sem,Mor,P,_,Rel,Gap),
             sign(Cat,Val,Dir,Arg,Sem,Mor,P,_,Rel,Gap),[rphrase]).
             
        
%%% lexical defaults 

apply_default(Default) :-               
        ( call(Default), !
        ; true
        ).

        
\end{verbatim}


%\subsection{The parser}
%\begin{verbatim}

%% this is a parser for constraint-based grammars in which a cfg is used
%% to produce an approximation of the parse forest. The actual derivations 
%% are constructed by checking for each cf derivation, whether there is a
%% corresponding derivation using the  actual grammar.

:- dynamic item/4, index_item/6.

:- ensure_loaded(cf_grammar).

parse(Sentence) :-
        retractall(item(_,_,_,_)), 
        retractall(index_item(_,_,_,_,_,_)),
        statistics(runtime,_),
        scan(Sentence,0,End),
        write('recovering..'), ttyflush,
        recover(End),
        statistics(runtime,[_,Time]),
        nl, nl, write(Time), write(' msecs').
        
recover(End) :-
        startsymbol(Sign),
        recover(t(Sign,Ds),_,0,End),
        display_tree(t(Sign,Ds),0),
        display_semantics(Sign),
        ttyflush,
        fail.
recover(_).

%%%%%%%%%%%%%%%%%%%%%%%%%%% cf parsing part %%%%%%%%%%%%%%%%%%%%%%%%%%

%%      A bottom-up chart parser for cfg with wf substring table and packing
%%      Grammar may only contain binary rules !

scan([],N,N) :- !.
scan([te,Word|Rest],N,End) :-           %% hack for `te' as verbal prefix 
        N1 is N + 1,
        lex_lookup([te,Word],N,N1),
        scan(Rest,N1,End).
scan([Word|Rest],N,End) :- 
        N1 is N + 1,
        lex_lookup(Word,N,N1),
        scan(Rest,N1,End).
        
lex_lookup(Word,N,N1) :-
        cf_lex(Word,Cat),
        add_item(word,Cat,Word,N,N1),
        fail.
lex_lookup(Word,N,N1) :-
        ( item(_,_,N,N1) -> true
        ; format('unknown word : ~w~n',[Word])
        ).

closure(Index2,Cat2,Mid,End) :-         % completion with compiled rules
        cf_rule(Name,LHS,[Cat1,Cat2]),
        item(Index1,Cat1,Begin,Mid),
        add_item(Name,LHS,[Index1,Index2],Begin,End),
        fail.
closure(_,_,_,_).
        
add_item(Name,LHS,Parsed,B,E) :-
        ( index_item(_,Name,LHS,Parsed,B,E) -> true             
        ; item(OldIndex,LHS,B,E) ->                             
          asserta(index_item(OldIndex,Name,LHS,Parsed,B,E))     % packing
        ; gen_index_sym(Index), 
          asserta(index_item(Index,Name,LHS,Parsed,B,E)),
          asserta(item(Index,LHS,B,E)),
          closure(Index,LHS,B,E)
        ).
        
gen_index_sym(Index) :-
        (  item(Index0,_,_,_)
        -> Index is Index0 + 1
        ;  Index = 1
        ).

%%%%%%%%%%%%%%%%% recovery of full parse trees %%%%%%%%%%%%%%%%%%%%%%%%%%%%%

recover(t(Sign,w(Word)),Index,Begin,End) :-       
        index_item(Index,word,_Cat,Word,Begin,End),
        lex(Word,Sign). 
recover(t(Sign,[t(D1,Ds1),t(D2,Ds2)]),Index,Begin,End) :-
        index_item(Index,Rule,_Cat,[I1,I2],Begin,End),
        rule0(Rule,Sign,[D1,D2],Head),
        ( Head = 1 ->                           % recovery is head-driven
          recover(t(D1,Ds1),I1,Begin,Mid),
          recover(t(D2,Ds2),I2,Mid,End)
        ; recover(t(D2,Ds2),I2,Mid,End),
          recover(t(D1,Ds1),I1,Begin,Mid)
        ).


%% debugging
count_cf(End,Analyses) :-
        findall(Ns,(index_item(Index,_,_,_,0,End),
                    recover_cf_tree(Index,_) ),
                Ns),
        length(Ns,Analyses).

cf_trees(End) :-
        index_item(Index,_,_,_,0,End),
        recover_cf_tree(Index,Tree),
        display_cf_tree(Tree,0),
        fail.
cf_trees(_).

recover_cf_tree(Index,t(Cat,w(Word))) :-
        index_item(Index,word,Cat,Word,_,_), !.
recover_cf_tree(Index,t(Cat,[T1,T2])) :-
        index_item(Index,_Rule,Cat,[I1,I2],_,_),
        recover_cf_tree(I1,T1),
        recover_cf_tree(I2,T2).
        
display_cf_tree(t(Cat,w(Word)),Indent) :-
        nl, tab(Indent), write(Cat), write(' -- '), write(Word).
display_cf_tree(t(Cat,[D|Ds]),Indent) :-
        nl, tab(Indent), write(Cat), 
        NewIndent is Indent + 3,
        display_cf_ds([D|Ds],NewIndent).

display_cf_ds([],_).
display_cf_ds([D|Ds],Indent) :-
        display_cf_tree(D,Indent),
        display_cf_ds(Ds,Indent).


\end{verbatim}


%\subsection{The Context-free Grammar}
%\begin{verbatim}
%%%%%%%%%%%%%%%%%%%%%%%% cf grammar %%%%%%%%%%%%%%%%%%%%%%%%%%%%%%%%%%%

cf_rule(ra,np,[det,n]).
cf_rule(ra,n,[adj,n]).
cf_rule(la,n,[n,rel_s]).
cf_rule(la,n,[n,pp]).

cf_rule(ra,pp,[prep,np]).

cf_rule(rel,rel_s,[rel_pronoun,vsub]).
cf_rule(rel,rel_s,[rel_np,vsub]).
cf_rule(rel,rel_s,[rel_pp,vsub]).
cf_rule(ra,rel_np,[rel_det,n]).
cf_rule(ra,rel_np,[det,rel_n]).
cf_rule(la,rel_n,[n,rel_pp]).
cf_rule(ra,rel_pp,[prep,rel_np]).
cf_rule(ra,rel_pp,[prep,rel_pronoun]).




cf_rule(ra,vsub,[om,vsub]).
cf_rule(ra,vsub,[dat,vsub]).

cf_rule(la,vsub,[np,vsub]).
cf_rule(la,vsub,[adjunct,vsub]).
cf_rule(la,vsub,[pp,vsub]).

cf_rule(ra,vsub,[vc,vsub]).             % 1.
cf_rule(la,vsub,[prefix,vc]).
% cf_rule(ra,vsub,[v,vc]).              % 2. heeft [willen slapen]
%% regel 1. subsumeert regel 2, aangezien iedere v een vc is, en iedere
%% vc een vsub.
cf_rule(la,vsub,[v,vc]).                % gekust [wil hebben]

cf_rule(la,vc,[prefix,vc]).
cf_rule(ra,vc,[v,vc]).                  % heeft [willen slapen]
cf_rule(la,vc,[v,vc]).                  % gekust [wil hebben]


cf_rule(ra,v1,[v1,np]).
cf_rule(ra,v1,[v1,adjunct]).
cf_rule(ra,v1,[v1,pp]).
cf_rule(ra,v1,[v1,prefix]).
cf_rule(ra,v1,[v1,vsub]).               % iedere vc is ook vsub

cf_rule(que,que,[np,v1]).
cf_rule(que,que,[adjunct,v1]).
cf_rule(que,que,[vsub,v1]).
cf_rule(que,que,[pp,v1]).


%%%%%%%%%%%%%%  cf lexicon

cf_lex(Word,np) :- proper_name(Word,_).
cf_lex(Word,np) :- pronoun(Word,_,_,_). 

cf_lex(Word,rel_pronoun) :- rel_pronoun(Word,_,_). 

cf_lex(Word,n) :- noun([Word,_],_,_).
cf_lex(Word,n) :- noun([_,Word],_,_).

cf_lex(Word,adj) :- adj([Word,_],_).
cf_lex(Word,adj) :- adj([_,Word],_).

cf_lex(Word,det) :- det(Word,_,_,_,_).
cf_lex(Word,det) :- det([Word,_],_,_,_,_).
cf_lex(Word,det) :- det([_,Word],_,_,_,_).

cf_lex(Word,rel_det) :- rel_det(Word).



cf_lex(Word,prep) :- preposition(Word,_).

cf_lex(Word,prefix) :- prefix(Word).

cf_lex(Word,adjunct) :- adjunct(Word,_).

cf_lex(om,om).
cf_lex(dat,dat).

cf_lex(Word,v1) :- cf_verb(Word).
cf_lex(Word,vsub) :- cf_verb(Word).
cf_lex(Word,vc) :- cf_verb(Word).
cf_lex(Word,v) :- cf_verb(Word).

cf_verb(Word) :- verb([Word|_],_,_,_).
cf_verb(Word) :- verb([_,Word|_],_,_,_).
cf_verb(Word) :- verb([_,_,Word|_],_,_,_).
cf_verb(Word) :- verb([_,_,_,Word|_],_,_,_).
cf_verb(Word) :- verb([_,_,_,_,Word],_,_,_).
cf_verb([te,Word]) :- verb([_,_,Word,_],_,_,_).
cf_verb([te,Word]) :- verb([_,_,_,Word,_],_,_,_).


\end{verbatim}


%\subsection{The Grammar/Parser Interface}
%\begin{verbatim}
%%%%%%%%%%%%%%%%%%%%%%%%% io etc. %%%%%%%%%%%%%%%%%%%%%%%%%%%%%%%%%

%%%% morphological_prefix/1 called by check_wrd in parser

morphological_prefix(te).

%%%% check_and_display_result/1 is called by the parser   

check_and_display_result(t(Sign,Ds)) :-
        startsymbol(Sign),
        display_tree(t(Sign,Ds),0),
        display_semantics(Sign),
        ttyflush.
        
startsymbol(X) :-
        X <=> s,
        ( X <=> vfirst
        ; X <=> vsecond
        ),
        X:gap <=> empty.

display_tree(t(Sign,w(Word)),Indent) :-
        cat_symbol(Sign,Sym),
        nl, tab(Indent), write(Sym), write(' -- '), write(Word).
display_tree(t(Sign,[D|Ds]),Indent) :-
        cat_symbol(Sign,Sym),
        nl, tab(Indent), write(Sym), 
        NewIndent is Indent + 3,
        display_ds([D|Ds],NewIndent).

display_ds([],_).
display_ds([D|Ds],Indent) :-
        display_tree(D,Indent),
        display_ds(Ds,Indent).
        
cat_symbol(Sign,Sym) :-
        ( Sign:gap <=> empty -> cat_symbol0(Sign,Sym)
        ; cat_symbol0(Sign,Sym0),
          Sign:gap <=> Gap,
          cat_symbol0(Gap,GapSym),
          Sym = Sym0-GapSym
        ).
        
cat_symbol0(Sign,Cat) :-
        Sign:cat <=> Var,
        var(Var),!,
        Var = Cat.
cat_symbol0(Sign,Cat) :-
        ( Sign:cat <=> n
        ; Sign:cat <=> np
        ; Sign:cat <=> pref
        ; Sign:cat <=> pp
        ; Sign:cat <=> adj
        ),
        Sign:cat <=> Cat.
cat_symbol0(Sign,s) :-
        Sign:cat <=> sent.
cat_symbol0(Sign,Symbol) :-
        ( Sign <=> Val/Arg -> Symbol = ValSym/ArgSym
        ; Sign <=> Arg\Val,
          Symbol = ArgSym\ValSym
        ),
        cat_symbol0(Val,ValSym),
        cat_symbol0(Arg,ArgSym).


display_semantics(Sign) :-
        Sign:sem <=> Sem,
        \+ \+ ( prolog:prettyvars(Sem),
                nl, write(Sem)
              ).
              
         
/*portray(fvt(F,V,T)) :-
        cat_symbol(fvt(F,V,T),Cat),
        write(Cat).           
portray(t(Sign,_Ds)) :-
        portray(Sign).
*/      
%%%%%%%%%%%%%%%% grammar compilation (for efficiency only) %%%%%%%%%%%%%%%%
:- dynamic lex/2, rule0/4.

compile_lexicon :-
        retractall(lex(_,_)),
        ( lexic(Word,Sign), assert(lex(Word,Sign)), 
          write(Word), write(', '), ttyflush, fail
        ; true
        ).

compile_rules :-
        retractall(rule0(_,_,_,_)),
        ( rule(Name,Lhs,Rhs,Head), assert(rule0(Name,Lhs,Rhs,Head)), fail
        ; true
        ).

\end{verbatim}

%}
\end{document}

