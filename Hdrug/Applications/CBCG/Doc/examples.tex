\begin{verbatim}
%%% use examples/0 to view examples, parse_ex/1 to parse examples

examples :-
        write('no analyses sentence'),
        format_example(1).
        
format_example(N) :-
        ( example(N,Ana,Sentence) ->
          format('~n~d~3+~d~8+',[N,Ana]),
          pprint_sent(Sentence),
          N1 is N + 1,
          format_example(N1)
        ; true
        ).

pprint_sent([W]) :- format('~a.',[W]).
pprint_sent([W|R]) :- format('~a ',[W]), pprint_sent(R).
        
parse_ex(N) :-
        example(N,_,Sentence),
        nl, pprint_sent(Sentence), nl,
        parse(Sentence).

%%%%%%%%%%%%%%%%%%%%%%%%%% example set %%%%%%%%%%%%%%%%%%%%%%%%%%%%%%%%%%%%%%%

%% example(No,Analyses,Sentence).

%% basics (main and subordinate clauses, relatives, questions)

example(1, 1, [jan,slaapt]).
example(2, 1, [de,mannen,uit,groningen,slapen]).
example(3, 2, [jan,kust,marie]).
example(4, 2, [de,man,die,slaapt,kust,marie]).
example(5, 1, [de,man,van,wie,marie,houdt,slaapt]).
example(6, 1, [de,vrouw,wiens,broer,ik,ken,slaapt]).
example(7, 2, [wie,denk,ik,dat,marie,kust]).

%% vraising, extraposition, and partial extraposition
%% subordinate clauses (probeert is ambiguous)

example(8, 1, [ik,denk,dat,hij,marie,wil,kussen]).
example(9, 0, [ik,denk,dat,hij,wil,marie,kussen]).
example(10, 1, [ik,denk,dat,hij,jan,marie,ziet,kussen]).

example(11, 1, [ik,denk,dat,hij,marie,vraagt,het,boek,te,lezen]).
example(12, 0, [ik,denk,dat,hij,marie,het,boek,vraagt,te,lezen]).
example(13, 1, [ik,denk,dat,hij,verzuimt,marie,te,kussen]).
example(14, 1, [ik,denk,dat,hij,marie,verzuimt,te,kussen]).
example(15, 1, [ik,denk,dat,hij,probeert,marie,te,kussen]).
example(16, 2, [ik,denk,dat,hij,marie,probeert,te,kussen]).

%% phrasal and subphrasal constituents

example(17, 1, [ik,denk,dat,hij,marie,heeft,willen,kussen]).
example(18, 1, [ik,denk,dat,hij,marie,schijnt,op,te,bellen]).
example(19, 1, [ik,denk,dat,hij,marie,heeft,willen,op,bellen]).
example(20, 1, [ik,denk,dat,hij,marie,heeft,op,willen,bellen]).
example(21, 1, [ik,denk,dat,hij,marie,op,heeft,willen,bellen]).
example(22, 1, [ik,denk,dat,hij,marie,heeft,gevraagd,het,boek,te,lezen]).
example(23, 1, [ik,denk,dat,hij,marie,heeft,geprobeerd,te,kussen]).
example(24, 1, [ik,denk,dat,hij,marie,heeft,proberen,te,kussen]).

%% participle inversion 

example(25, 1, [ik,denk,dat,hij,marie,heeft,gekust]).
example(26, 1, [ik,denk,dat,hij,marie,gekust,heeft]).
example(27, 1, [ik,denk,dat,hij,van,marie,gehouden,moet,hebben]).
example(28, 0, [ik,denk,dat,hij,van,marie,moet,gehouden,hebben]).

%% oke with extraposition, but not with vr verbs

example(29, 1, [ik,denk,dat,jan,hem,gevraagd,heeft,marie,op,te,bellen]).
example(30, 0, [ik,denk,dat,hij,marie,willen,heeft,op,bellen]).

%% modal inversion, only with finite modals, complement must be -vr

example(31, 1, [ik,denk,dat,hij,marie,kussen,wil]).
example(32, 0, [ik,denk,dat,hij,marie,kussen,heeft,willen]).
example(33, 1, [ik,denk,dat,jan,hem,vragen,wil,marie,te,kussen]).
example(34, 0, [ik,denk,dat,hij,marie,hebben,wil,gekust]).

%% the scope of adjuncts

example(35, 2, [ik,denk,dat,jan,marie,mogelijk,kussen,wil]).
example(36, 2, [ik,denk,dat,jan,altijd,een,meisje,kussen,wil]).
example(37, 1, [ik,denk,dat,altijd,een,jongen,haar,kussen,wil]).

%% adjunct topicalization (spurious ambiguities arising from interaction of
%% extraction-lr, add-adjuncts rule, and division)

example(38, 7, [nooit,hoor,ik,marie,zingen]).

%% vp topicalization (spurious ambiguities arising from interaction of 
%% v1/v2 and inversion)

example(39, 2, [kussen,wil,jan,marie]).
example(40, 2, [marie,kussen,wil,jan]).
example(41, 1, [willen,kussen,heeft,jan,marie]).
example(42, 0, [willen,heeft,jan,marie,kussen]).
\end{verbatim}
