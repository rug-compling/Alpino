\documentclass[12pt]{article}
\bibliographystyle{plain}
\usepackage{a4}

\begin{document}

\title{Annotation differences between CGN and the Alpino Treebank}

\author{Leonoor van der Beek, Gosse Bouma, Gertjan van Noord}

\maketitle

The Alpino Treebank uses CGN Dependency Structures, as defined in the
CGN annotation guidelines \cite{cgn-annot} and exemplified in the
Leuven Yellow Pages \cite{leuven-yellow-pages}. However, the Alpino
Treebank differs with respect to this annotation convention in a
number of ways. The differences are listed here.

We make a distinction between minor and major differences. We regard a
difference as minor if it appears to be possible to implement a filter
which would transform one representation into the other automatically.
Note that such a filter has not actually been implemented.

\subsection*{Major Differences}

\begin{itemize}
  
\item In the Alpino Treebank obligatory control relations are
  represented explicitly. This affects the treatment of auxiliaries,
  modals, control verbs, passives, etc. Motivation: such control
  relations are not always predictable from lexical specifications.
  
\item In the Alpino Treebank, modifiers in sentences with auxiliaries
  are generally annotated such that the modifier is attached to the
  main verb, not the finite auxiliary verb. Note that in other
  constructions, if the correct attachment is hard to determine, the
  modifier is attached high.
  
\item In the Alpino Treebank, partitives are not treated special.
  Instead, in the Alpino Treebank a noun phrase such as {\tt \'e\'en
    van de drie} is analysed by taking {\tt \'e\'en} as the head,
  which is modified by a {\em PP}. Thus, the dependency relation {\em
    part} is not used in the Alpino Treebank.
  
\item In the Alpino Treebank, all cases of {\em prt} are treated as
  {\em mod}. They are typically not grouped together, either. If they
  are grouped together, then a head-modifier structure results. Thus, 
  the dependency relation {\em prt} is not used in the Alpino Treebank.

\item As in CGN, leaf nodes contain a part of speech label. However,
  the inventory of part of speech labels is much smaller in the Alpino
  Treebank. In addition, the annotation of part of speech labels
  contains many inconsistencies and should be regarded poor quality. 

\item Leaf nodes in the Alpino Treebank contain pointers into the
  string (this is equivalent to CGN), but in addition also contain a
  canonical form of the word(s) --- typically the stem of the
  word. Note however, that the annotation of canonical word forms
  still contains many inconsistencies and should be regarded poor
  quality. 

\item We never use complex heads. If a word-group has the {\em hd}
  relation, then it must be a leaf.

\item Sbar complements are never assigned the {\em obj1} relation; they
  get the {\em vc} relation. As a consequence, there is never a need
  to have multiple {\em obj1} relations for a given head.

\item Idiomatic phrases are assigned the {\em svp} relation, as in
  CGN. However, unlike CGN idiomatic phrases are not analyzed. As a
  consequence, if a word-group has the {\em svp} relation, it must be
  a leaf. \footnote{This was probably not a good decision.}

\end{itemize}

\subsection*{Minor differences}

\begin{itemize}

\item Secondary edges are represented by means of co-indexing in the
  Alpino Treebank.
  
\item In CGN, multi-word-units are represented by a flat tree (for
  each word a node), in the Alpino Treebank represented by a single
  node. NB. There are of course also non-minor differences w.r.t. the
  decision when something is regarded a multi-word-unit or not. As a
  consequence, in the Alpino Treebank you can {\bf never} have
  discontinuous multi-word-units. \footnote{This seems right:
    discontinuous multi-word-units are weird.}
  
\item Root sentences introduced by words such as {\tt want, en, maar}
  get a {\em dlink, nucl} representation in CGN. In the Alpino
  Treebank, these are treated as complementizers, so they receive {\em
    cmp, body} representation. The {\em dlink} label is not used in
  the Alpino Treebank.
  
\item In CGN {\tt te} in a {\em te-infinitive} is treated as a
  complementizer. In the Alpino Treebank, a {\em te-infinitive} is
  treated as a single (multi-word) unit; {\tt te} is treated as
  inflection.

\item In IPP constructions, we use the {\em ppart} category, whereas
  CGN uses {\em inf}.
  
\item In CGN coordination without an explicit coordinator are {\em
    lists} where every conjunct gets the {\em lp} relation. In the
  Alpino Treebank coordinations with or without explicit coordinator
  are treated the same. In the Alpino Treebank, the {\em lp} relation
  is never used; and the {\em list} category is never used.
  
\item In CGN interjections (hesitations, disfluencies) are represented
  as a node without a relation name ({\tt --}). We do not represent
  these parts of the input at all.

\item The category label {\em compp} is not used in the Alpino
  Treebank. Instead, we use category labels such as {\em advp, ap} for
  word groups that are {\em obcomp} with respectively an adverbial or
  adjectival head etc. Apparantly, this is now the case in CGN too.

\item {\em Rangtelwoorden} such as `eerste' and `tweede' are assigned
  the {\em det} role in the Alpino Treebank, but {\em mod} in CGN.

\item In noun phrases such as {\em de rode auto is duurder dan {\bf de
      groene}} CGN would use {\em auto} as the head. In Alpino, we
      treat {\em groene} as a nominalized adjective, and it is treated
      as the head of the noun phrase.

\item In constructions consisting of the word {\em als} followed by a
  noun phrase, CGN treats {\em als} as a complementizer (if it can be
  re-phrased as {\em zoals}), with cmp and body, whereas Alpino always
  assumes {\em als} is a preposition (with hd and obj1).

\item In coordinations, there often is a modifier in front of the
  second conjunct. In Alpino, these modifiers are modifiers of the
  head of this conjunct. In CGN, there typically is a dp-dp structure
  consisting of this modifier and the second conjunct.

\end{itemize}

This list is incomplete.

\bibliography{references}

\end{document}
%%% Local Variables: 
%%% mode: latex
%%% TeX-master: t
%%% End: 
